%%%%%%%% ICML 2019 EXAMPLE LATEX SUBMISSION FILE %%%%%%%%%%%%%%%%%

\documentclass{article}

% Recommended, but optional, packages for figures and better typesetting:
\usepackage{microtype}
\usepackage{graphicx}
\usepackage{subfigure}
\usepackage{booktabs} % for professional tables

% hyperref makes hyperlinks in the resulting PDF.
% If your build breaks (sometimes temporarily if a hyperlink spans a page)
% please comment out the following usepackage line and replace
% \usepackage{icml2019} with \usepackage[nohyperref]{icml2019} above.
\usepackage{hyperref}

% Attempt to make hyperref and algorithmic work together better:
\newcommand{\theHalgorithm}{\arabic{algorithm}}

% Use the following line for the initial blind version submitted for review:
\usepackage{icml2019}

% If accepted, instead use the following line for the camera-ready submission:
%\usepackage[accepted]{icml2019}

% The \icmltitle you define below is probably too long as a header.
% Therefore, a short form for the running title is supplied here:
\icmltitlerunning{On the Generalization Performance of Compressed Word Embeddings}

%Added by Avner
\usepackage{amsmath,amsfonts,amsthm,amssymb}

%\usepackage{graphicx}
%\usepackage{amsmath,amsfonts,amssymb,amsopn,amsbsy,amsthm}
%\usepackage{dsfont,bm,bbm,times,url,verbatim,epstopdf,xspace}
%\usepackage[capitalize]{cleveref}
%\usepackage{placeins}
%\usepackage{amssymb,amsthm}
%\usepackage{bbm,epstopdf}
%\usepackage[group-separator={,}]{siunitx}
%\usepackage[export]{adjustbox}
%\usepackage{hhline}
%\usepackage[hypertexnames=false]{hyperref}
% for footnotes without markers
%\newcommand\blfootnote[1]{%
%  \begingroup
%  \renewcommand\thefootnote{}\footnote{#1}%
%  \addtocounter{footnote}{-1}%
%  \endgroup
%}

%\newcommand{\vfigsp}{\vspace{-0.5em}}
%\newcommand{\vflistsp}{\vspace{-0.25em}}
\newcommand{\vfigsp}{}
\newcommand{\vflistsp}{}

%\newcommand{\tsigma}{\tilde{\sigma}}
%\newcommand{\eps}{\epsilon}
%\newcommand{\tZ}{\tilde{Z}}
%\newcommand{\tA}{\tilde{A}}
%\newcommand{\by}{\bar{y}}
\newcommand{\sm}{\sigma_{\min}}
%\newcommand{\tK}{\tilde{K}}
%\newcommand{\tX}{\tilde{X}}
%\newcommand{\defeq}{:=}
%\newcommand{\eqdef}{=:}
\DeclareMathOperator{\clip}{clip}
\DeclareMathOperator{\Span}{span}
%\DeclareMathOperator{\Dim}{dim}
%\providecommand{\tr}{\mathop{\rm tr}}
%\providecommand{\sup}{\mathop{\rm sup}}
%\newcommand{\E}{\mathbb{E}} % Expectation symbol (ADDED BY TRI)

%\DeclareMathOperator{\eigover}{Eo}
\newcommand{\eigover}{\mathcal{E}}
% New commands added from Nystrom notes.
\newcommand{\Nystrom}{Nystr\"{o}m }
\newcommand{\NystromNS}{Nystr\"{o}m} % NS means ``no space''
\newcommand{\NystromCaps}{NYSTR\"{O}M }
\newcommand{\NystromCapsNS}{NYSTR\"{O}M} % NS means ``no space''
\newcommand{\Ap}{A^\perp}
\newcommand{\mua}{\mu_A}
\newcommand{\muap}{\mu_{\Ap}}
\newcommand{\bphi}{\bar{\phi}}
\newcommand{\by}{\bar{y}}
\newcommand{\bS}{\bar{S}}
\newcommand{\bT}{\bar{T}}
%\newcommand{\bw}{\bar{w}}
%\newcommand{\bv}{\bar{v}}
%\newcommand{\bF}{\bar{F}}
%\newcommand{\bg}{\bar{g}}
%\newcommand{\be}{\bar{\eta}}
%\newcommand{\br}{\bar{\rho}}
%\newcommand{\bU}{\bar{U}}
%\newcommand{\bV}{\bar{V}}
%\newcommand{\bLam}{\bar{\Lambda}}
%\newcommand{\bL}{\bar{L}}
\newcommand{\hx}{\hat{x}}
\newcommand{\hb}{\hat{b}}
%\newcommand{\hX}{\hat{X}}
%\newcommand{\hY}{\hat{Y}}
\newcommand{\hp}{\hat{\phi}}
\newcommand{\hK}{\hat{K}}
%\newcommand{\tp}{\tilde{\phi}}
\newcommand{\tk}{\tilde{k}}
%\newcommand{\tC}{\tilde{C}}
\newcommand{\eps}{\epsilon}
\newcommand{\teps}{\tilde{\epsilon}}
\newcommand{\tS}{\tilde{S}}
\newcommand{\tK}{\tilde{K}}
\newcommand{\tZ}{\tilde{Z}}
\newcommand{\tX}{\tilde{X}}
\newcommand{\tA}{\tilde{A}}
\newcommand{\tf}{\tilde{f}}
\newcommand{\tz}{\tilde{z}}
\newcommand{\tsigma}{\tilde{\sigma}}
\newcommand{\tgamma}{\tilde{\gamma}}
\newcommand{\tlambda}{\tilde{\lambda}}
\newcommand{\hcR}{\widehat{\cR}}
\newcommand{\id}{I}
\newcommand{\sq}{\sqrt{2}}
\newcommand{\ulq}{\underline{q}}
\newcommand{\olq}{\overline{q}}
\newcommand{\ulx}{\underline{x}}
\newcommand{\olx}{\overline{x}}
\newcommand*{\QED}{\hfill\ensuremath{\square}}
\DeclareMathOperator*{\argmin}{arg\,min}
\DeclareMathOperator*{\argmax}{arg\,max}
\DeclareMathOperator*{\rank}{rank}

\newcommand*\conj[1]{\overline{#1}}

% note: I removed the package amsthm because it defines "proof", which is already defined in jmlr2e package
\newcommand{\ie}{i.e.}
\newcommand{\eg}{e.g.}
\newcommand{\etal}{et al.\ }
\newcommand{\etalNS}{et al.}

\def\ddefloop#1{\ifx\ddefloop#1\else\ddef{#1}\expandafter\ddefloop\fi}
%
%% \bbA, \bbB, ...
\def\ddef#1{\expandafter\def\csname bb#1\endcsname{\ensuremath{\mathbb{#1}}}}
\ddefloop ABCDEFGHIJKLMNOPQRSTUVWXYZ\ddefloop
%
%% \bfA, \bfB, ...
%\def\ddef#1{\expandafter\def\csname bf#1\endcsname{\ensuremath{\mathbf{#1}}}}
%\ddefloop ABCDEFGHIJKLMNOPQRSTUVWXYZabcdefghijklmnopqrstuvwxyz\ddefloop
%
%% \bfalpha, \bfbeta, ...,  \bfGamma, \bfDelta, ...,
%\def\ddef#1{\expandafter\def\csname bf#1\endcsname{\ensuremath{\pmb{\csname #1\endcsname}}}}
%\ddefloop {alpha}{beta}{gamma}{delta}{epsilon}{varepsilon}{zeta}{eta}{theta}{vartheta}{iota}{kappa}{lambda}{mu}{nu}{xi}{pi}{varpi}{rho}{varrho}{sigma}{varsigma}{tau}{upsilon}{phi}{varphi}{chi}{psi}{omega}{Gamma}{Delta}{Theta}{Lambda}{Xi}{Pi}{Sigma}{varSigma}{Upsilon}{Phi}{Psi}{Omega}{ell}\ddefloop
%
%% \cA, \cB, ...
\def\ddef#1{\expandafter\def\csname c#1\endcsname{\ensuremath{\mathcal{#1}}}}
\ddefloop ABCDEFGHIJKLMNOPQRSTUVWXYZ\ddefloop
%
%% \mbf0, \mbf1, ...
%\newcommand\mbf{\ensuremath{\mathbf}}
%
%\DeclareMathOperator*{\argmin}{arg\,min}
%\DeclareMathOperator*{\argmax}{arg\,max}
%
%\newcommand\parens[1]{(#1)}
\newcommand\norm[1]{\|#1\|}
%\newcommand\braces[1]{\{#1\}}
%\newcommand\brackets[1]{[#1]}
%\newcommand\ceil[1]{\lceil#1\rceil}
%\newcommand\abs[1]{|#1|}
%\newcommand\ind[1]{\ensuremath{\mathds{1}\{#1\}}}
\newcommand\dotp[1]{\langle #1 \rangle}
%\newcommand\Parens[1]{\left(#1\right)}
%\newcommand\Norm[1]{\left\|#1\right\|}
%\newcommand\Braces[1]{\left\{#1\right\}}
%\newcommand\Brackets[1]{\left[#1\right]}
%\newcommand\Ceil[1]{\left\lceil#1\right\rceil}
%\newcommand\Abs[1]{\left|#1\right|}
%\newcommand\Ind[1]{\mathds{1}\left\{#1\right\}}
\newcommand\Dotp[1]{\left\langle#1\right\rangle}
%
\newcommand{\RR}{\ensuremath{\bbR}} %real numbers
\newcommand{\Nat}{\ensuremath{\bbN}} %natural numbers 
\newcommand{\CC}{\ensuremath{\bbC}} %complex numbers
%
%\newcommand{\risk}{\ensuremath{R}} % risk
%\newcommand{\loss}{\ensuremath{\ell}} % loss
%\newcommand{\logloss}{\ensuremath{\ell_{\operatorname{log}}}} % logistic loss
%
%\newcommand{\rfm}{\ensuremath{z}} % (random) feature map
%\newcommand{\kfm}{\ensuremath{\phi}} % kernel feature map
%\newcommand{\kernel}{\ensuremath{K}} % kernel function
%
%%\setlength{\marginparwidth}{25mm}
%\usepackage[textsize=tiny]{todonotes}
%%\usepackage[disable]{todonotes}
%\newcommand{\djh}[2][]{\todo[color=red!20!white,#1]{DH: #2}}
%\newcommand{\avner}[2][]{\todo[color=red!20!white,#1]{AM: #2}}
\newcommand\new[1]{\textcolor{blue}{#1}}
\newcommand\todo[1]{\textcolor{red}{#1}}
\newcommand\avner[1]{\textcolor{red}{AM: #1}}
\newcommand\tri[1]{\textcolor{red}{TD: #1}}
\newcommand\jian[1]{\textcolor{red}{JZ: #1}}
%
%\newcommand{\numtr}{\ensuremath{n_{\operatorname{tr}}}}
%\newcommand{\numho}{\ensuremath{n_{\operatorname{ho}}}}
%\newcommand{\numte}{\ensuremath{n_{\operatorname{te}}}}
%
%\usepackage{tabularx}
%\newcolumntype{Y}{>{\centering\arraybackslash}X}
%\usepackage{multirow}
%
%\usepackage{algorithm}
%\usepackage{algorithmic}
%\usepackage{pdflscape}
%
%% define vector and matrix symbols
%\newcommand{\vct}[1]{\boldsymbol{#1}} % vector
%\newcommand{\mat}[1]{\boldsymbol{#1}} % matrix
%\newcommand{\cst}[1]{\mathsf{#1}}  % constant
%
%%%%% Special math symbols
%\newcommand{\field}[1]{\mathbb{#1}}
%\newcommand{\R}{\field{R}} % real domain
%\newcommand{\C}{\field{C}} % complex domain
%\newcommand{\F}{\field{F}} % functional domain
%%\newcommand{\T}{^{\top}\!\!} % transpose
%\newcommand{\T}{^{\textrm T}} % transpose
%
%%% operator in linear algebra, functional analysis
%\newcommand{\inner}[2]{#1\cdot #2}
%%\newcommand{\norm}[1]{\|#1\|}
%\newcommand{\twonorm}[1]{\left\|#1\right\|_2^2}
%\newcommand{\onenorm}[1]{\|#1\|_1}
%\newcommand{\Map}[1]{\mathcal{#1}}  % operator in functions, maps such as M: domain1 --> domain 2
%
%% operator in probability: expectation, covariance, 
\newcommand{\E}{\mathbb{E}} % Expectation symbol (ADDED BY TRI)
\newcommand{\Prob}{\mathbb{P}}
\newcommand{\ProbOpr}[1]{\mathbb{#1}}
%% independence
%\newcommand\independent{\protect\mathpalette{\protect\independenT}{\perp}}
%\def\independenT#1#2{\mathrel{\rlap{$#1#2$}\mkern2mu{#1#2}}}
%%\newcommand{\ind}[2]{{#1} \independent{#2}}
%\newcommand{\cind}[3]{{#1} \independent{#2}\,|\,#3} % conditional independence
\newcommand{\expect}[2]{%
\ifthenelse{\equal{#2}{}}{\ProbOpr{E}_{#1}}
{\ifthenelse{\equal{#1}{}}{\ProbOpr{E}\left[#2\right]}{\ProbOpr{E}_{#1}\left[#2\right]}}} % Expectation: syntax: E{1}{2} = E_1[2], E{}{2}=E[2], E{1}{} = E_1
\newcommand{\var}[2]{%
\ifthenelse{\equal{#2}{}}{\ProbOpr{VAR}_{#1}}
{\ifthenelse{\equal{#1}{}}{\ProbOpr{VAR}\left[#2\right]}{\ProbOpr{VAR}_{#1}\left[#2\right]}}} 
% Expectation: syntax: V{1}{2} = V_1[2], V{}{2}=V[2], V{1}{} = V_1
%\newcommand{\cndexp}[2]{\ProbOpr{E}\,[ #1\,|\,#2\,]}  % conditional expectation
%
%% operator in optimization
%\DeclareMathOperator{\argmax}{arg\,max}
%\DeclareMathOperator{\argmin}{arg\,min}
%
%% special functions
%\newcommand{\trace}[1]{\operatornamewithlimits{tr}\left\{#1\right\}}
\newcommand{\diag}{\operatornamewithlimits{diag}}
%\newcommand{\sign}{\operatornamewithlimits{sign}}
%\newcommand{\const}{\operatornamewithlimits{const}}
%
%% special display
%\newcommand{\parde}[2]{\frac{\partial #1}{\partial  #2}}
%
%
%% environment
\newtheorem{thm}{Theorem}
\newtheorem{theorem}{Theorem}
\newtheorem{definition}{Definition}
\newtheorem{lemma}[theorem]{Lemma}
\newtheorem{conjecture}[theorem]{Conjecture}
\newtheorem{proposition}[theorem]{Proposition}
\newtheorem{claim}{Claim}
\newtheorem{corollary}{Corollary}[theorem]

\newtheorem{innercustomthm}{Theorem}
\newenvironment{customthm}[1]
{\renewcommand\theinnercustomthm{#1}\innercustomthm}
{\endinnercustomthm}

\newtheorem{innercustomprop}{Proposition}
\newenvironment{customprop}[1]
{\renewcommand\theinnercustomprop{#1}\innercustomprop}
{\endinnercustomprop}


%\newtheorem{thm}{Theorem}
%\newtheorem{theorem}{Theorem}
%\newtheorem{claimcor}{Corollary}[claim]
%\newtheorem{lemma}{Lemma}[theorem]



%% DPP symbols
%\newcommand{\Xt}{{\mathcal{X}_{(t)}}} % X_t
%\newcommand{\Yt}{{Y_{(t)}}} % Y_t
%\newcommand{\Ystart}{{Y^{\star}_{(t)}}} % Y*_t
%\newcommand{\ground}{{\mathcal{Y}}} % ground set
%\newcommand{\groundX}{{\mathcal{X}}} % ground set
%
%
%
%% shorthand
%\newcommand{\vtheta}{\vct{\theta}}
%\newcommand{\vmu}{\vct{\mu}}
%\newcommand{\valpha}{\vct{\alpha}}
%\newcommand{\vc}{\vct{c}}
%\newcommand{\vp}{\vct{p}}
%\newcommand{\vq}{\vct{q}}
%\newcommand{\vx}{{\vct{x}}}
%\newcommand{\vy}{\vct{y}}
%\newcommand{\vz}{{\vct{z}}}
%\newcommand{\vu}{\vct{u}}
%\newcommand{\vo}{{\vct{o}}}
%\newcommand{\va}{\vct{a}}
%\newcommand{\vb}{\vct{b}}
%\newcommand{\vr}{\vct{r}}
%\newcommand{\vt}{\vct{t}}
%\newcommand{\vs}{\vct{s}}
%\newcommand{\vv}{\vct{v}}
%\newcommand{\vw}{\vct{w}}
%\newcommand{\ones}{\vct{1}}
%\newcommand{\mU}{\mat{U}}
%\newcommand{\mA}{\mat{A}}
%\newcommand{\mB}{\mat{B}}
%\newcommand{\mC}{\mat{C}}
%\newcommand{\mW}{\mat{W}}
%\newcommand{\mH}{\mat{H}}
%\newcommand{\mS}{\mat{S}}
%\newcommand{\mJ}{\mat{J}}
%\newcommand{\mM}{\mat{M}}
%\newcommand{\mT}{\mat{T}}
%\newcommand{\mZ}{\mat{Z}}
%\newcommand{\mO}{\mat{O}}
%\newcommand{\mY}{\mat{Y}}
%%\newcommand{\cN}{\cst{N}}
%%\newcommand{\cQ}{\cst{Q}}
%%\newcommand{\cD}{\cst{D}}
%%\newcommand{\cL}{\cst{L}}
%%\newcommand{\cK}{\cst{K}}
%%\newcommand{\cH}{\cst{H}}
%\newcommand{\sQ}{\mathcal{Q}}
%\newcommand{\sS}{\mathcal{S}}
%\newcommand{\mL}{\mat{L}}
%\newcommand{\mI}{\mat{I}}
%\newcommand{\mK}{\mat{K}}
%\newcommand{\mSigma}{\mat{\Sigma}}
%\newcommand{\sF}{\mathcal{F}}
%\newcommand{\vzero}{\vct{0}}
%\newcommand{\vf}{\vct{f}}
%\newcommand{\vF}{\vct{F}}
%\newcommand{\vP}{\vct{P}}
%
%
%\newcommand{\vh}{\vct{h}}
%\newcommand{\vg}{\vct{g}}
%\newcommand{\vphi}{\vct{\phi}}
%\newcommand{\vpsi}{\vct{\psi}}
%\newcommand{\vdelta}{\vct{\delta}}
%\newcommand{\vomega}{\vct{\omega}}
%\newcommand{\mOmega}{\mat{\Omega}}
%
%% 
%\newcommand{\nx}{\tilde{x}}
%\newcommand{\vnx}{{\tilde{\vx}}}
%\newcommand{\vnz}{{\tilde{\vz}}}
%%\newcommand{\deltavx}{(\vnx-\vx)}
%\newcommand{\deltavx}{\delta_\vx}
%\newcommand{\vmx}{\bar{\vx}}
%\newcommand{\vmz}{\bar{\vz}}
%\newcommand{\sigmax}{\mSigma_{\vx}}
%\newcommand{\sigmaz}{\mSigma_{\vz}}
%\newcommand{\no}{\tilde{o}}
%\newcommand{\vno}{{\tilde{\vo}}}
%\newcommand{\nell}{\tilde{\ell}}
%\newcommand{\jacob}{\mat{J}}
%\newcommand{\hess}{\mat{H}}
%\newcommand{\mloss}{\hat{\ell}}
%\newcommand{\vDD}{\vct{\Delta}}
%%
%\newcommand{\eat}[1]{}
%
%\newcommand{\FS}[1]{{\color{blue}FS:#1}}
%
%\newcommand\id{\ensuremath{\mathbbm{1}}}
%\newcommand*{\QED}{\hfill\ensuremath{\square}}
\newcommand{\defeq}{:=}
\newcommand{\eqdef}{=:}
%\newcommand{\var}{{\rm Var}} % Variance
\providecommand{\tr}{\mathop{\rm tr}}
\newcommand\numberthis{\addtocounter{equation}{1}\tag{\theequation}}
\begin{document}
	
\twocolumn[
\icmltitle{On the Generalization Performance of Compressed Word Embeddings}

% It is OKAY to include author information, even for blind
% submissions: the style file will automatically remove it for you
% unless you've provided the [accepted] option to the icml2019
% package.

% List of affiliations: The first argument should be a (short)
% identifier you will use later to specify author affiliations
% Academic affiliations should list Department, University, City, Region, Country
% Industry affiliations should list Company, City, Region, Country

% You can specify symbols, otherwise they are numbered in order.
% Ideally, you should not use this facility. Affiliations will be numbered
% in order of appearance and this is the preferred way.
\icmlsetsymbol{equal}{*}

\begin{icmlauthorlist}
	\icmlauthor{Random Name}{equal,stan}
\end{icmlauthorlist}

\icmlaffiliation{stan}{Department of Computer Science, Stanford University, Stanford, California, USA}

\icmlcorrespondingauthor{Random Name}{random@stanford.edu}

% You may provide any keywords that you
% find helpful for describing your paper; these are used to populate
% the "keywords" metadata in the PDF but will not be shown in the document
\icmlkeywords{Machine Learning, ICML}

\vskip 0.3in
]

% this must go after the closing bracket ] following \twocolumn[ ...

% This command actually creates the footnote in the first column
% listing the affiliations and the copyright notice.
% The command takes one argument, which is text to display at the start of the footnote.
% The \icmlEqualContribution command is standard text for equal contribution.
% Remove it (just {}) if you do not need this facility.

%\printAffiliationsAndNotice{}  % leave blank if no need to mention equal contribution
\printAffiliationsAndNotice{\icmlEqualContribution} % otherwise use the standard text.


\begin{abstract}
Word embedding compression is critical to the deployment in memory-constrained devices such as mobile phones. State-of-the-art embedding compression methods attains strong empirical performance by learning compressed representations. In this paper, we work on a simple and fast embedding compression method based on uniform fixed-point quantization. 
%This method first determines a threshold value to clip extreme values in the embedding matrix using simple golden section search; it then uniformly quantizes the clipped embedding into fixed-point representation. 
Without computationally expensive training, the uniform quantization approach can empirically match the performance of state-of-the-art compression methods across different compression rates. 
Theoretically, we perform analysis to answer how to optimize down stream task performance with fixed memory budgets for quantized embeddings. Our analysis suggests quantizing higher dimensional uncompressed embedding to lower precision, until it reaches the optimal dimensionality for uncompressed embedding.
	

%approach and solution
%Contribution 1: empirically match 
%Contribution 2: 
\todo{theory paragraph in intro can be shorten} 
\end{abstract}

\section{Introduction}
\label{sec:intro}
%\subsection{Joint Version}
In recent years, \textit{word embeddings} \citep{word2vec13,glove14,fasttext18} have brought large improvements to a wide range of applications in natural language processing (NLP) \citep{collins16,drqa17}.
By encoding words as low-dimensional dense vectors, word embeddings allow NLP tasks to be framed as pattern recognition problems over dense continuous spaces, which can then be tackled using the powerful machinery of neural networks.
However, these word embeddings can occupy a very large amount of memory, making it expensive to deploy them in data centers, and impractical to use them in memory-constrained environments like smart phones.
Thus, compressing these embeddings is an important problem.


Designing simple and fast embedding compression methods with strong empirical performance is challenging.
There have been numerous important works proposing compression schemes for word embeddings and studying their empirical performance \citep{sparse16,andrews16,dccl17,kway18};
for example, current state-of-the-art methods train neural network models to represent each word as a sum of vectors from learned dictionaries \citep{dccl17,kway18}.
Although these methods can attain strong performance on a range of natural language processing tasks such as language modeling \citep{mikolov10} and machine translation \citep{bahdanau15}, designing and training the neural network for compression can be time-consuming.

%Designing embedding compression methods with both strong empirical performance and provable generalization bounds for downstream tasks is challenging.
%There have been numerous important works proposing compression schemes for word embeddings and studying their empirical performance \citep{sparse16,andrews16,dccl17,kway18};
%for example, current state-of-the-art methods train neural network models to represent each word as a sum of vectors from learned dictionaries \citep{dccl17}.
%Although these methods can attain large reductions in memory while performing well on tasks such as language modeling \citep{mikolov10} and machine translation \citep{bahdanau15}, to the best of our knowledge there exists no analysis describing the effect compression has on generalization performance for these methods.

%In this work, we show that a simple compression method based on uniform quantization can empirically compete with the state-of-the-art methods, and we present generalization bounds for the quantized embeddings for linear regression models.
%The compression method has two parts: first, it determines the optimal threshold at which to clip the extreme values in the word embedding matrix using golden section search \citep{golden53}.
%It then uniformly quantizes the clipped embeddings within the clipped interval.
%This simple approach can compress large embeddings in seconds on a single CPU core, without computationally expensive training or hyperparameter tuning.

In this paper, we work on a simple compression method based on uniform quantization. We show that it can empirically compete with the state-of-the-art methods, and we analysis theoretically to understand how to optimize down steam task performance under a fixed memory budget for the quantized embedding.
Our simple compression method has two components: first, it determines the optimal threshold at which to clip the extreme values in the word embedding matrix using golden section search \citep{golden53}.
It then uniformly quantizes the clipped embeddings to fixed-point representations within the clipped interval.
This simple approach can compress large embeddings in seconds on a single CPU core, without computationally expensive training or hyperparameter tuning.


Empirically, we demonstrate that this uniform quantization method can match the performance of the state-of-the-art compression schemes on question answering, sentiment analysis, and word analogy and similarity tasks, for both GloVe \citep{glove14} and fastText \citep{fasttext18} embeddings.
Using the DrQA model \citep{drqa17} on the Stanford Question Answering Dataset (SQuAD) \citep{squad16}, for example, it attains a compression ratio of 32x with an average F1 score only 0.47\% absolute below the uncompressed GloVe embeddings, while the state-of-the-art deep compositional code learning method \cite{dccl17} is 0.43\% below.
%\todo{Discuss other tasks}

Theoretically, we work on understanding a critical question regarding embedding quantization given a fixed memory budget to store embedding matrices --- {to optimize downstream task performance, should we quantize a higher dimensional uncompressed embedding to lower precision, or should we quantize a lower dimensional uncompressed embedding to higher precision.} Specifically, we analyze generalization bounds in the case of linear ridge regression using word embeddings.
Our analysis reveals that one can optimize down stream performance under the fixed memory budget following a two-steps principle. In the first step, we show an optimal dimensionality exists as a consequence of the bias-variance trade-off in generalization performance. In the second step, we show quantization can have negligible impact on the generalization bound when the uncompressed embedding has a slow-decaying spectrum (empirically observed to be true). Thus under limit memory budgets, one should use the lowest quantization precision until the quantized embedding reaches the optimal dimensionality for the uncompressed counter-part. 



%Theoretically, we show in the context of linear ridge regression that quantization has a negligible effect on generalization performance when the quantization noise is small relative to the regularization parameter.
%Furthermore, we show that when the spectrum of the word embedding matrix decays slowly (which we empirically observe to be true), strong regularization has minimal impact on generalization performance.
%Combined, these two results imply that the uniformly quantized embeddings can have similar generalization performance to the uncompressed embeddings at high compression rates.
%
%1. Theoretically, we analyze the trade-off between dimensionality and precision. From this analysis, we extract guidelines how to optimize for generalization performance given a fixed memory budget: there exists an optimal dimension without 
%2. Specifically, we analyze in the linear regression setting.
%3. We first go in renosante, show there is an optimal without memory constraint.
%3. We then show quantization has minimal effect in generalizaiton
%These two understanding combines and lead to our stuff.
%
%
%
%Theoretically, we show in the context of linear ridge regression that quantization has a negligible effect on generalization performance when the quantization noise is small relative to the regularization parameter.
%Furthermore, we show that when the spectrum of the word embedding matrix decays slowly (which we empirically observe to be true), strong regularization has minimal impact on generalization performance.
%Combined, these two results imply that the uniformly quantized embeddings can have similar generalization performance to the uncompressed embeddings at high compression rates.


The rest of this paper is organized as follows:
In Section~\ref{sec:relwork} we review related work, including existing methods for compressing word embeddings.
In Section~\ref{sec:uniform} we present our uniform quantization method for compressing embeddings.
We discuss our large-scale experiments in Section~\ref{sec:experiments}, and the theoretical analysis for our method in Section~\ref{sec:theory}.
We conclude in Section~\ref{sec:conclusion}.

%\subsection{Jian Version}
%In recent years, \textit{word embeddings} \citep{word2vec13,glove14,fasttext18} have brought large improvements to a wide range of applications in natural language processing (NLP) \citep{collins16,drqa17}.
%By encoding words as low-dimensional dense vectors, word embeddings allow NLP tasks to be framed as pattern recognition problems over dense continuous spaces, which can then be tackled using the powerful machinery of neural networks.
%However, these word embeddings can occupy a very large amount of memory, making it expensive to deploy in the data center, and impractical to deploy them to memory-constrained environments like smart phones.
%Thus, it is important to decrease the amount of memory occupied by the embeddings for massive deployment.
%
%Designing embedding compression methods with both strong empirical performance and provable bounds on generalization performance is challenging.
%For example, the current state-of-the-art methods for compressing word embeddings are deep learning-based approaches, such as deep compositional code \citep{dccl17} and K-way discrete code \citep{chen2018learning}. In these methods, the compression are achieved by training a neural network architectures to represent each word as a sum of vectors from learned dictionaries.
%
%
%In this work, we revisit a compression method based on simple uniform quantization, and analyze the impact of quantization on generalization performance in the case of linear regression models.
%The quantization-based approach has two parts: By using golden section search, we find the optimal threshold at which to clip the extreme values in the word embedding matrix. We then uniformly quantize the clipped embeddings within the clipped interval. Based on these two steps, this simple approach can achieve fast compression without computationally expensive training or extensive hyperparameter tuning.
%
%Empirically, we demonstrate that uniform quantization can match the performance of the more complex baselines across a variety of tasks, embedding types, and compression rates.
%On the machine reading comprehension task, we evaluate different embeddings with DrQA model \cite{drqa17} using the Stanford Question Answering Dataset (SQuAD) \citep{squad16}. Uniform quantization can attain 32x compression for GloVe and fastText embeddings, while on average attaining F1 scores within 0.45\% and \todo{XX\%} absolute of the full-precision embeddings, respectively. To evaluate on more NLP applications with various model architecture, we also show that quantized embeddings matche the generalization performance of embeddings compressed with state-of-the-art deep learning-based methods, across 7 sentiment analysis, 2 word similarity and 2 analogy tasks. 
%
%We theoretically show that the quantization has negligible effect on generalization in important schemes when quantization error is relatively small comparing to regularization strength. Specifically we work in the case of linear regression, which is a simplified model of regression and ranking-based NLP tasks such as sentiment analysis. Our analysis reveals that when the spectrum of the word embedding matrix decays slowly (which we empirically observe to be true), strong regularization has minimally impact on generalization performance. Thus uniformly quantized embeddings can have similar generalization performance to that of uncompressed full precision embeddings.
%\todo{Discuss how clipping fits into theory.?}
%
%
%
%\subsection{Avner Version}
%In recent years, \textit{word embeddings} \citep{word2vec13,glove14,fasttext18} have brought large improvements to a wide range of applications in natural language processing (NLP) \citep{collins16,drqa17}.
%By encoding words as low-dimensional dense vectors, word embeddings allow NLP tasks to be framed as pattern recognition problems over dense continuous spaces, which can then be tackled using the powerful machinery of neural networks.
%However, these word embeddings can occupy a very large amount of memory, making it impractical to deploy them to memory-constrained environments like smart phones.
%Our goal in this work is to dramatically decrease the amount of memory occupied by the embeddings, while provably retaining strong performance on downstream tasks.
%
%Designing a compression scheme which is capable of retaining strong performance at low memory budgets, while also being amenable to theoretical analysis, is a challenging problem.
%There have been numerous important works proposing compression schemes for word embeddings and studying their empirical performance, for example using deep dictionary learning or sparse representations \citep{sparse16,andrews16,dccl17}.
%Although these methods can attain large reductions in memory usage while performing well on tasks such as language modeling \citep{mikolov10} and machine translation \citep{bahdanau15}, to the best of our knowledge there exists no analysis describing the effect compression has on generalization performance for these methods.
%
%%Designing powerful and principled compression schemes is challenging because it is unclear a priori how to model how compression affects performance.
%%This is particularly true when the compression schemes being considered are complex, and when large non-convex models are employed for the downstream tasks.
%%For example, the current state-of-the-art method for compressing word embeddings, called deep compositional code learning (DCCL), uses a deep architecture to represent each word as a sum of vectors from learned dictionaries \citep{dccl17}.
%
%In this work, we show that a simple compression method based on uniform quantization can compete with the above-mentioned methods, and we bound this method's impact on the generalization performance of linear regression models.
%This method has two parts: first, it determines the optimal threshold at which to clip the extreme values in the word embedding matrix.
%It then uniformly quantizes the clipped embeddings within the clipped interval.
%This algorithm is easy to implement, has no hyperparameters, and runs in seconds on a single CPU core.
%
%Empirically, we demonstrate that this method can match the performance of the state-of-the-art compression schemes on question answering, sentiment analysis, and word analogy and similarity tasks, for both GloVe \citep{glove14} and fastText \citep{fasttext18} embeddings.
%On the Stanford Question Answering Dataset (SQuAD) \citep{squad16}, for example, it attains a compression ratio of 32x with an average F1 score only 0.47\% absolute below the full-precision GloVe embeddings, while the deep dictionary learning method of \citet{dccl17} is 0.43\% below.
%
%Theoretically, we show in the context of linear ridge regression that quantization has a negligible effect on generalization performance when the quantization noise is small relative to the regularization parameter.
%Furthermore, we show that when the spectrum of the word embedding matrix decays slowly (which we empirically observe to be true), a large regularizer, and thus low-precision, will perform similarly to the full-precision embeddings.
%Importantly, our analysis is not specific to word embeddings; it directly extends to any setting in which a linear model is being learned over a uniformly quantized representation.
%%\todo{Discuss how clipping fits into theory. Motivate why regression matters (as opposed to classification) for NLP.}
%%Our theoretical analysis builds on recent work analyzing the generalization performance of low-precision representations in the context of kernel ridge regression \citep{lprff18}.
%
%Our main contributions:
%\begin{itemize}
%	\item We show that a simple compression scheme based on uniform quantization can compete with state-of-the-art word embedding compression methods on a variety of tasks.
%	\item We prove that this compression method's impact on the generalization performance of linear ridge regression models is minimal when the quantization noise is small relative to the regularization parameter.
%\end{itemize}



\section{Background: Embedding Compression Methods and Quality Metrics}
\label{sec:prelim}
In this paper, we focus on gaining a deeper understanding of what intrinsic characteristics of compressed embeddings determine their downstream performance.
We begin by reviewing existing compression methods (\S\ref{subsec:existing_methods}) as well as existing metrics (\S\ref{subsec:existing_metrics}) which measure the quality of a compressed embedding relative to the uncompressed embedding.
% NOTE: I DON'T THINK WE NEED TO REPEAT PAPER OUTLINE, SINCE WE JUST DID THIS IN PARAGRAPH ABOVE (END OF INTRO)
%We then show in Section~\ref{sec:challenge} that these metrics do not align well with downstream performance, leading us to propose a new metric in Section~\ref{sec:new_metric} which aligns much better.

%In this paper, we focus on gaining a deeper understanding on what intrinsic characteristic of a compressed embedding determines its downstream task performance. To setup this connection, we first discuss preliminaries in different word embedding compression methods. We then present existing compression quality metrics, which are intrinsic characteristics of a compressed embedding matrix. In Section~\ref{sec:challenge}, we demonstrate that these existing metrics can poorly correlates with the downstream task performance of embeddings compressed by different methods; this leads to a new metric we propose in Section~\ref{sec:new_metric} which correlates well with downstream performance.

\subsection{Embedding Compression Methods}
\label{subsec:existing_methods}
We now review a number of existing compression methods for word embeddings;
these are the compression methods which we will be considering in the remainder of the paper.

\paragraph{Deep Compositional Code Learning (DCCL)} 
The DCCL method \citep{dccl17}\footnote{This idea is also independently adopted by \citet{kway18}.} uses a dictionary learning approach to represent a large number of word vectors using a much smaller number of basis vectors.
These basis vectors are organized into multiple dictionaries, and each word is represented as a sum which includes one basis vector per dictionary.
These dictionaries are trained using an autoencoder-style architecture to minimize the embedding vector reconstruction error.
%\todo{(we should pull prelim as a separate paragraph)}To deploy word embedding to memory constraint settings in data center and edge devices, many methods has been proposed to compress the word embedding. \todo{(we should consider whether we can say kmeans is stoa.)}. Among these methods, state-of-the-art \emph{DCCL approach} \citep{dccl17}\footnote{This idea is also independently adopted by \citet{kway18}.} adopts dictionary learning to represent a large number of word vectors with fewer basis vectors. These basis vectors are organized into multiple dictionaries, and each word is represented as a hash code to discretely combine basis from the dictionaries. In particular, the discrete combination hash code is attained by training an autoencoder with Gumbel-softmax reparameterization \citep{maddison2016concrete,jang2016categorical} to minimize the embedding vector reconstruction error. 

\paragraph{K-means Compression}
The k-means clustering algorithm can be used to compress word embeddings by first running 1-dimensional clustering on the full list of values in the word embedding matrix, and then replacing each value with the index of the closest centroid \citep{andrews16}.
Using $k=2^b$ allows for storing each matrix entry using only $b$ bits.

%Built on a significantly different machinery, embeddings compressed by a \emph{k-mean method} also demonstrates strong empirical performance in downstream tasks \citep{andrews16}. This k-means method assigns embedding matrix entry values into clusters, and each embedding entries is the approximated by the corresponding cluster center. In this way, one can achieve compression by only storing the cluster id instead of the original value for each embedding matrix entry.

\paragraph{Uniform Quantization}
Uniform quantization is a classic compression method which divides an interval into a fixed number of sub-intervals of equal size, and then rounds the values in each sub-interval to one of the boundaries of the sub-interval \citep{quant77} (for a review of uniform quantization, see Appendix \todo{XX}).
We apply this method to compress word embeddings as follows:
We first determine the optimal threshold at which to clip the extreme values in the word embedding matrix, and we then uniformly quantize the clipped embeddings within the clipped interval.
This algorithm is summarized in Algorithm~\ref{alg:smallfry}, where we denote by $Q_{b,r}$ the function which quantizes the interval $[-r,r]$ using $b$-bits per value.
Note that in our experiments, we find the optimal clipping threshold $r^*$ to within a specified tolerance $\eps > 0$ using the golden-section search algorithm \citep{golden53}.


%In contrast to the DCCL and k-mean approach, \emph{uniform quantization} is a simple method requiring no computationally expensive training. This method, as shown in at a high-level in Algorithm~\ref{alg:smallfry}, first clips the embedding matrix into the range $[-r,r]$, and then uniformly quantizes the entries of the clipped matrix; we choose the value of $r$ which minimizes the reconstruction error of the quantized matrix.
%Because our algorithm uses uniform quantization, we first review uniform quantization, and then describe in more detail how our algorithm uses it to compress word embeddings.


%\begin{itemize}
%	\item Briefly describe DCCL and k-means approach
%	\item Present uniform quantization with Pseudo-code
%\end{itemize}
%At a high-level, our algorithm clips the embedding matrix into the range $[-r,r]$, and then uniformly quantizes the entries of the clipped matrix;
%we choose the value of $r$ which minimizes the reconstruction error of the quantized matrix.
%Because our algorithm uses uniform quantization, we first review uniform quantization, and then describe in more detail how our algorithm uses it to compress word embeddings.

\begin{algorithm}[tb]
   \caption{Uniform quantization for word embeddings}
   \label{alg:smallfry}
\begin{algorithmic}[1]
	\STATE {\bfseries Input:}  Embedding matrix $X \in \RR^{n \times d}$, quantization function $Q_{b,r}$, clipping function $\clip_r\colon\RR\rightarrow[-r,r]$.
	\STATE {\bfseries Output:} Quantized embeddings $\hat{X}$.
	\STATE $r^* \defeq \argmin_{r \in [0,\max(|X|)} \|Q_{b,r}(\clip_r(X))-X\|_F$.
	%Search for $r^* \in [0,\max(|X|)]$ minimizing $\|Q_{b,r}(\clip_r(X))-X\|_F$.
	\STATE {\bfseries Return:} $Q_{b,r^*}(\clip_{r^*}(X))$.
\end{algorithmic}
\end{algorithm}


\subsection{Matrix Approximation Error and Generalization}
\label{subsec:existing_metrics}
\todo{Do we need more top-down here? What to say?}
Here we review a number of existing metrics which we use in order to evaluate the quality of a compressed word embedding relative to the uncompressed embedding.
Several of these metrics are based on comparing the pairwise inner product (Gram) matrices of the compressed vs.\ uncompressed embeddings.
The Gram matrices of embeddings are natural to consider for a couple reasons:
First, the loss function for training word embeddings typically only considers dot-products between embedding vectors \citep{word2vec13,glove14}.
Second, one can view word embedding training as implicit matrix factorization \citep{levy2014neural}, and thus comparing the Gram matrices of two embedding matrices is similar to comparing the matrices these embeddings are implicitly factoring.
\todo{Do we need this much detail?}

%Recently, there has been substantial progress in understanding the matrix reconstruction error and generalization for models derived from matrix factorization. Particularly in our empirical evaluation, we consider the \emph{Pointwise Inner Product (PIP)} loss \citep{yin18} and \emph{spectral approximation error} \citep{avron17,lprff18} as the metrics for word embedding compression quality.

When defining the metrics below, we will denote by $X \in \RR^{n\times d}$ the uncompressed embedding, and $\tX \in \RR^{n \times k}$ the compressed embedding, where $n$ is the vocabulary size, and $d$, $k$ are the compressed and uncompressed word embedding dimensions, respectively.
We now review the metrics:


%Many main stream word embedding generation problem can be casted matrix decomposition . For example, Skip-gram Word2Vec \citep{word2vec13} implicitly factorizes the Pointwise Mutual Information (PMI) matrix while GloVe \citep{glove14} decomposes the word co-occurrence matrix. In this paper, given an uncompressed embedding $X\in\mathbb{R}^{n\times d}$ and its compressed version $\tilde{X}\in\mathbb{R}^{n\times \tilde{d}}$, we consider the reconstruction error between Gram matrices $G = XX^T$ and $\tilde{G} = \tilde{X}\tilde{X}^T$ as the proxy metrics of compression quality. Recently, there has been substantial progress in understanding the matrix reconstruction error and generalization for models derived from matrix factorization. Particularly in our empirical evaluation, we consider the \emph{Pointwise Inner Product (PIP)} loss \citep{yin18} and \emph{spectral approximation error} \citep{avron17,lprff18} as the metrics for word embedding compression quality.

\paragraph{Word Embedding Reconstruction Error}
The first and simplest way of comparing two embeddings $X$ and $\tX$ is simply to measure the reconstruction error $\|X-\tX\|_F^2$.
In fact, many embedding compression methods \citep{andrews16,dccl17} use this as the loss function for compression.
Note that in order to be able to use this metric, $X$ and $\tX$ must have the same dimension.
This limits the settings in which this metric can be used, relative to the metrics which depend only on the Gram matrices.

\paragraph{Pairwise Inner Product (PIP) Loss}
Given $XX^T$ and $\tX\tX^T$, the Gram matrices of the uncompressed and compressed embeddings, the \textit{Pairwise Inner Product (PIP} Loss) is defined as $\|XX^T -\tX\tX^T\|_F^2$ \citep{yin18}.
This metric was recently proposed in order to explain the existence of an optimal dimension for word embeddings;
specifically, this work shows that increasing the embedding dimension decreases a bias term but increases a variance term in the PIP loss.
Though the PIP loss does not explicitly consider the generalization performance of the embeddings on downstream tasks, \citep{yin18} show that selecting the embedding dimension to minimize the PIP loss can help attain strong downstream NLP task performance.

%It is first proposed to reveal the bias-variance trade-off when selecting the optimal embedding dimensionality. Though PIP loss does not explicitly consider the generalization performance of downstream tasks,  \citet{yin18} show that select embedding dimensionality by PIP loss minimization can attain strong downstream NLP task performance.

\paragraph{Spectral Approximation Error}
Recent work on understanding the generalization performance of kernel approximation methods has proposed a way of comparing Gram matrices in terms of their spectral properties \citep{avron17,lprff18}.
Specifically, a gram matrix $\tX\tX^T$ is said to be \textit{$(\Delta_1,\Delta_2)$-spectral approximation} of another Gram matrix $XX^T$ if it satisfies $(1-\Delta_1) XX^T \preceq \tX\tX^T \preceq (1+\Delta_2)XX^T$.
In order to make this metric more robust, $\lambda I$ is typically added to each of the Gram matrices before computing the minimum $\Delta_1$ and $\Delta_2$ values satisfying the above equation;
in the supervised learning setting, the $\lambda$ value corresponds to the regularization parameter used during training.
\citet{lprff18} show that if $\tX\tX^T+\lambda I$ is a $(\Delta_1,\Delta_2)$-spectral approximation of $XX^T + \lambda$, then the linear model trained using $\tX$ will attain similar generalization performance to the linear model trained using $X$.

Although all of these metrics have strong theoretical foundations, in Section~\ref{subsec:hard_explain} we show that when these metrics are measured on compressed word embeddings, the resulting values align poorly with the downstream performance of the compressed embeddings.
This suggests that these metrics are poor indicators of the quality of the compressed embeddings.

%Building on recent theoretical work \citep{avron17}, \citet{lprff18} propose the notion of \textit{$(\Delta_1,\Delta_2)$-spectral approximation} between Gram matrices to understand the generalization performance of kernel approximation methods on supervised learning problems.
%This metric is defined as 
%A Gram matrix $\tX\tX^T$
%In the context of uncompressed embedding $X$ and compressed embeddings $\tilde{X}$, the Gram matrix $\tilde{G} = \tilde{X}\tilde{X}^T$ is a $(\Delta_1,\Delta_2)$-spectral approximation to $G = XX^T$ if it satisfies 
%\[(1-\Delta_1) G \preceq \tilde{G} \preceq (1-\Delta_2) G.\]
%Though PIP loss and $(\Delta_1,\Delta_2)$-spectral approximation can roughly imply the generalization performance by examining the matrix reconstruction error in previous works \citep{avron17,yin18,lprff18}, we observe in Section\ref{subsec:hard_explain} these two metrics, as a measure of compression quality, can poorly correlate with the downstream task performance across different compressed embedding types.
%\paragraph{Matrix Approximation Error and Generalization}
%\label{subsec:error_gen}
%	\begin{itemize}
%		\item PIP Loss: introduce pip loss
%		\item Delta 1 and Delta 2: introduce Deltas
%	\end{itemize}
	
%\subsection{Hardness in Explaining Generalization}
%\label{subsec:hard_explain}
%	\begin{itemize}
%		\item Results and claim 1: uniform quantization method can compete well with the state-of-the-art, and it can match uncompressed embedding with high compression rate.
%		\begin{itemize}
%			\item show the generalization performance for quantized embedding and other baselines at different bit rate. (DCCL, kmeans, dimension reduction and uniform quantization)
%			\item Figures on 4 tasks: DrQa, one sentiment, one word analogy and one word similarity
%			\item Tables on tasks performance and time
%		\end{itemize}
%		\item Results and claim 2: Existing matrix reconstruction errors (PIP, spectral approximation error--delta1, delta2) do not correlate well with the generalization performance of compressed embeddings.
%		\begin{itemize}
%			\item Show correlation between performance and Frob norm (PIP) / Deltas (maybe the combination of delta1 and delta2) across a few tasks.
%			\item We can optionally do R2 measurements reliably
%		\end{itemize}
%	\end{itemize}

\section{Challenges in Understanding Compressed Embedding Performance}
\label{sec:challenge}
\subsection{Matrix Approximation Error and Generalization}
\label{subsec:error_gen}
	\begin{itemize}
		\item PIP Loss: introduce pip loss
		\item Delta 1 and Delta 2: introduce Deltas
	\end{itemize}
	
\subsection{Hardness in Explaining Generalization}
\label{subsec:hard_explain}
	\begin{itemize}
		\item Empirically demonstrate existing matrix approximation error does not correlate well with the generalization performance of compressed embeddings.
	\end{itemize}

\section{A New Metric for Compression Quality}
\label{sec:new_metric}
%\subsection{Eigenspace Overlap}
%\label{subsec:eigen_overlap}
%	\begin{itemize}
%		\item Introduce the eigenspace overlap metric.
%%		\item Present the generalization bound.
%	\end{itemize}
%	
%\subsection{Generalization of Compressed Embeddings}
%\label{subsec:revisit}
%	\begin{itemize}
%		\item Empirically demonstrate the correlation with respect to generalization performance
%		\item Discuss why it explains better than Deltas (if have not discuss the relation to Deltas, we should discuss here also).
%	\end{itemize}

\begin{table*}
	\caption{The Spearman rank correlation coefficient $\rho$ between compression quality metrics and downstream performance. The larger the absolute value of the coefficient, the stronger the correlation is.
	Within each entry in the table, the correlations are presented in terms of `GloVe (Wiki'14) $\rho$ \;/\; GloVe (Wiki'17) $\rho$ \;/\; fastText $\rho$'.
	}
	\small
	\begin{tabular}{c | c | c | c | c}
		\toprule
		& QA & Sentiment & Analogy & Similarity \\
		\midrule
		Embed. reconst. error &  $-0.61/-0.22/-0.84$  &  $-0.40/-0.25/-0.69$  &  $\mathbf{-0.98/-1.00}/-0.88$  &  $-0.19/0.77/-0.36$  \\ 
		PIP loss &  $-0.49/-0.32/-0.74$  &  $-0.29/-0.27/-0.61$  &  $-0.82/-0.67/-0.82$  &  $0.05/-0.07/-0.23$  \\  
		$1/(1-\Delta_1)$ &  $-0.62/-0.83/-0.77$  &  $-0.42/-0.79/-0.56$  &  $-0.90/-0.98/-0.90$  &  $-0.22/-0.69/-0.31$  \\  
		$\Delta_2$ &  $-0.46/-0.87/0.18$  &  $-0.39/-0.87/0.30$  &  $-0.70/-0.82/-0.02$  &  $-0.43/-0.72/-0.15$  \\  
		$1 - \mathcal{E}$ & $\mathbf{-0.81/-0.97/-0.90}$  &  $\mathbf{-0.72/-0.96/-0.73}$  &  $-0.79/-0.91/\mathbf{-0.96}$  &  $\mathbf{-0.62/-0.88/-0.67}$  \\  
		%%	\midrule
		%Embed. reconst. error & & & & \\
		%%	\midrule
		%PIP loss & & & & \\
		%%	\midrule
		%$1/(1-\Delta_1)$ & & & & \\
		%%	\midrule
		%$\Delta_2$ & & & & \\
		%%	\midrule
		%$\mathcal{E}$ & & & & \\
		\bottomrule
	\end{tabular}
	\label{tab:sp_rank}
\end{table*}

\begin{figure*}
	\footnotesize
	\begin{tabular}{@{\hskip -0.0in}c@{\hskip -0.0in}c@{\hskip -0.0in}c@{\hskip -0.0in}c@{\hskip -0.0in}}
		\includegraphics[width=.245\linewidth]{figures/glove400k_qa_best-f1_vs_subspace-eig-overlap_linx.pdf} &
		\includegraphics[width=.245\linewidth]{figures/glove-wiki400k-am_qa_best-f1_vs_subspace-eig-overlap_linx.pdf} &
		\includegraphics[width=.245\linewidth]{figures/glove400k_sentiment_trec_test-acc_vs_subspace-eig-overlap_linx.pdf} &
		\includegraphics[width=.245\linewidth]{figures/glove-wiki400k-am_sentiment_trec_test-acc_vs_subspace-eig-overlap_linx.pdf}	\\
		(a) GloVe (Wiki'14), QA & \;\;\;\;(b) GloVe (Wiki'17), QA  & \;\;\;\;\;\;(c) GloVe (Wiki'14), sentiment & \;\;\;\;\;(d) GloVe (Wiki'14)
	\end{tabular}
	\caption{Our proposed compression quality metric eigenspace overlap correlates strongly with the downstream task performance of compressed embeddings.  On both the question answering and sentiment analysis task, we demonstrate that eigenspace overlap consistently achieves strong correlation 1) across different compression methods (shown in (a), (c)) and 2) across different precision for the uniform quantization methods (shown in (b), (d)). In other words, embeddings compressed by different methods using different configurations can performs similarly in downstream tasks, when they have similar eigenspace overlap with the same reference uncompressed embedding.}
	\label{fig:good_correlation}
\end{figure*}

In this section, we introduce the \textit{eigenspace overlap metric} to measure the quality of a compressed embedding relative to the uncompressed embedding.
We prove average-case generalization bounds for this metric in the context of linear regression, and show that this metric indeed aligns very well with the downstream performance of the compressed embeddings.

\subsection{Eigenspace Overlap and Generalization}
\label{subsec:eigen_overlap}
\begin{definition}
Given two embedding matrices $X \in \RR^{n \times d}$, $\tX \in \RR^{n \times k}$, whose Gram matrices have eigendecompositions $XX^T = USU^T$, $\tX\tX^T = VRV^T$ for $U \in \RR^{n\times d}$, $V\in \RR^{n \times k}$, we define the eigenspace overlap metric $\eigover(X,\tX) \defeq  \frac{1}{\max(d,k)}\|U^T V\|_F^2$.
\end{definition}

This metric measures the degree to which the span of the eigenvectors with nonzero eigenvalue of $\tX\tX^T$ agrees with that of $XX^T$.
In particular, assuming $k\leq d$, it measures the ratio between the squared Frobenius norm of $U$ before and after being projected onto $V$.
As an example, if the span $\Span(V)$ of the columns of $V$ is a subspace of $\Span(U)$, then $\eigover(X,\tX) = \frac{1}{d}\|U^T V\|_F^2 = \frac{1}{d}\|UU^T V\|_F^2 = \frac{1}{d}\|V\|_F^2 = \frac{k}{d}$.
In contrast, if $\Span(V)$ is orthogonal to $\Span(U)$, then $\eigover(X,\tX) = 0$.

We now show that this metric is closely related to the generalization performance of the linear regression model trained using the compressed embeddings $\tX$ in place of $X$.
For simplicity, we will consider here the noiseless fixed design regression setting;
In this setting, the training loss is equal to the generalization loss.
Letting $y\in\RR^n$ denote the vector of labels and using the closed form solution for the optimal parameters $w^* = (X^T X)^{-1}X^Ty$, we can derive that the generalization performance is equal to $\cR(X) = \|Xw^* - y\|^2 = \|y\|^2 - \|U^T y\|^2$.

To expose the influence of the eigenspace overlap on the generalation performance, it will be necessary for us to considering an average-case analysis.
This is necessary because in worst-case analysis, if there exists a single direction in $\Span(U)$ orthogonal to $\Span(V)$ (which always occurs when $\dim(V) < \dim(U)$) the label vector $y$ can simply be equal to this direction.
In this case $\cR(X) = \|y\|^2 - \|U^T y\|^2 = 0$, while $\cR(\tX) = \|y\|^2 - \|V^T y\|^2 = \|y\|^2$.
Thus, the setting we will consider instead is one where $y$ is a random vector in $\Span(U)$.
We consider the setting $y \in \Span(U)$ (which implies $\cR(X) = 0$) for simplicity because we are most interested in the situation where we know the uncompressed embedding matrix $X$ performs well, and we would like to understand how well $\tX$ will do.
We now present our average-case result (see Appendix~\ref{app:theory} for proof):

\begin{proposition}
If $y = Uz$ for a random vector $z \in \RR^d$ with zero mean and identity covariance, then
\begin{eqnarray}
\expect{y}{\cR(\tX) - \cR(X)} &=& d\cdot(1 - \eigover(X,\tX))
\end{eqnarray}
\end{proposition}

This proposition reveals that a larger eigenspace overlap value results in better generalization performance for the compressed embedding.



\subsection{Revisit Performance of Compressed Embeddings}
\label{subsec:revisit}
We now demonstrate that unlike the metrics we discussed in Section~\ref{sec:prelim}, this metric empirically aligns very well with the performance of compressed embeddings on downstream tasks.
In Figure~\ref{fig:good_correlation} we show scatter plots of downstream performance vs.\ eigenspace overlap, and see that in general higher eigenspace overlap corresponds to better performance.
Thus, even though our analysis is for a linear regression setting, we can see that this metric predicts performance well on a variety of downstream tasks which use neural networks for training.

%\todo{Discuss performance relative to $\Delta_1$,$\Delta_2$?}




\section{Generalization performance of Uniform Quantization }
\label{sec:quant_embed}
To further demonstrate how the eigenspace overlap metric can be used to better understand the performance of compressed embeddings, in this section we show that we can upper bound the eigenspace overlap for uniformly quantized embeddings.
Given the above theoretical and empirical connections between eigenspace overlap and generalization performance, these bounds help explain why the uniformly quantized embeddings perform so well.
To prove these bounds, we leverage the classic Davis-Kahan $\sin(\Theta)$ theorem from matrix perturbation analysis \citep{sintheta70}.
Because we know exactly what the noise structure of the uniform quantization method is, we can use this knowledge to bound how much the eigenspace of the compressed embeddings can differ from the uncompressed embeddings.

We now present the result (see Appendix~\ref{app:theory} for proof):
\begin{theorem}
	Let $X \in \RR^{n\times d}$ be a bounded embedding matrix with $X_{ij} \in [-\frac{1}{\sqrt{d}},\frac{1}{\sqrt{d}}]$ with largest and smallest singular values $\sigma_{\max}$ and $\sigma_{\min}$.
	Then the eigenspace overlap of the corresponding $b$-bit uniformly quantized embedding matrix can be lower bounded as follows:
	\begin{eqnarray*}
		\eigover(X,\tX) &\geq& 1 - \frac{16n}{d(2^b-1)^2} \Bigg( \frac{\sigma_{\max} + \frac{\sqrt{n}}{2^b-1} }{\sigma_{\min}^2} \Bigg)^2
	\end{eqnarray*}
\label{thm1}
\end{theorem}

We can further simplify this expression using the fact that $\sigma_{\max} = \|X\|_2 \leq \sqrt{n}$; using this fact, we get the following corollary:
\begin{corollary}
If $b \geq \log_2\bigg(\frac{8n}{\sigma_{\min}^2 \sqrt{\rho \, d} }} + 1\bigg)$, then the $b$-bit uniformly quantized embedding matrix $\tX$ satisfies $\eigover(X,\tX) \geq 1-\rho$.
\end{corollary}

The corollary shows that one can attain an eigenspace overlap arbitrarily close to $1$ if a sufficient number of bits $b$ are used.

\paragraph{Empirical Validation of Theory}
We now validate the above theory by showing how the eigenspace overlap grows as the precision of the uniformly quantized embeddings is increased.
For this experiment, we generate a random matrix $X \in \RR^{10^5 \times 300}$, with each entry drawn uniformly at random in the interval $[-\frac{1}{\sqrt{300}}, \frac{1}{\sqrt{300}}]$.
We then measure the eigenspace overlap of quantized versions of $X$, for precisions $b \in \{1,2,4,8,16,32\}$.
As we can see in Figure~\ref{fig:micro_eigoverlap_vs_prec}, the eigenspace overlap grows quite quickly as the precision grows, attaining a value of approximately $0.82$ at $b=2$, and $0.99$ at $b=4$.
This helps explain the strong empirical performance of the uniformly quantized embeddings which we observed at low precisions on the real word embedding experiments (Figure~\ref{fig:perf_comp}).

%\todo{Add section on clippings effect on the eigenspace overlap?}

\begin{figure}
	\begin{tabular}{c}
		\includegraphics[width=\linewidth]{figures/micro_eig_overlap_vs_precision.pdf} 
\end{tabular}
\caption{Empirical Validation of Theorem~\ref{thm1}. We measure the eigenspace overlap of uniformly quantized embeddings with the uncompressed embedding for various precisions ($b\in\{1,2,4,8,16,32\}$), and observe the eigenspace overlap grows quickly as the precision grows.  For the uncompressed embedding, we generate a random matrix $X \in \RR^{10^5 \times 300}$, with each entry drawn uniformly at random in $[-\frac{1}{\sqrt{300}}, \frac{1}{\sqrt{300}}]$.}
\label{fig:micro_eigoverlap_vs_prec}
\end{figure}

%\subsection{Theory validation}
%	\begin{itemize}
%		\item Impact of quantization on overlap 
%			\begin{itemize}
% 				\item Exp 1: overlap vs precision for different dimensionality. Expectation: overlap increases with higher precision.
%				\item Exp2: overlap vs dimensionality for different precision. Expectation: overlap increases with dimensionality. This explains that under fix memory budget, using lower bits quantization can be beneficial
%			\end{itemize}
%		\item The impact of clipping on eigen-subspace overlap
%			\begin{itemize}
%				\item Simulation based experiments on subspace overlap as a function of different clipping threshold and precision. 
%				\item The way we introduce this in: in our main theorem, we assume the dynamic range is $O(1/\sqrt{d})$ as a consequence of the automatic clipping. We want to show here this is the case in practice and then discuss the specific way clipping influence eigenspace overlap.
%			\end{itemize}
%		\item Loop-back discussion on the large scale empirical experiments in Section~\ref{subsec:hard_explain}
%	\end{itemize}


\section{Related Work}
\label{sec:relwork}
Compressing word embeddings has recently become an active area of research in NLP.
One important line of work uses sparse dictionary learning in order to compress word embedding matrices while maintaining strong empirical performance.
In this framework, words are represented as sparse combinations of vectors in a learned dictionary.
One approach to learning these sparse representations is to use non-negative sparse matrix factorization \citep{murphy12,sparse16}.
Recently, deep compositional code learning (DCCL) \cite{dccl17} and K-way D-dimensional discrete code learning \cite{kway18} independently proposed using deep architectures in order to learn a set of dictionaries, where each word is represented as a sum of one vector from each dictionary.
This approach to compressing word embeddings attains high compression rates while matching the state-of-the-art performance on tasks such as machine translation and sentiment analysis.
An alternative compression approach proposed by \citet{andrews16} is to run k-means clustering on the entries of the embedding matrix, and then represent each entry by the index of the closest centroid.
In this paper, we compare the performance of these existing compression methods with a simple and fast method based on uniform quantization.
We show this simple approach can match the empirical performance of the state-of-the-art approaches, and does not require expensive training or hyperparameter tuning.

Our analysis on how quantization impacts generalization performance builds on recent work on kernel ridge regression \citep{avron17, lprff18}.
In those papers, the authors study the impact of approximating the kernel matrix on the performance of the trained model.
Similarly, in this work we study the impact of approximating the embedding matrix with uniform quantization, and analyze the impact this approximation has on generalization performance.

%\begin{itemize}
%	\item Existing compression methods: DCCL paper \citep{dccl17}, K-way code \citep{chen2018learning}, k-means \citep{andrews16}, and sparse representations \citep{sparse16}
%	\item Our paper \citep{lprff18} and Avron's paper \citep{avron17}
%	\item Recently there are works with good empirical performances: Sparsity and quantization. Quantization: k means method. Sparsity: based on variants of dictionary learning: dictionary based learning approaches (including Murphy)--- 1) murphy/chen16 uses NNSE , k way approaches, As another sparsity approach, Andrew also tested with per-dimensional sparsity using auto encoder. K means quantization . These method improves iterative training process, in this paper we use no training
%	\item Our generalization builds on matrix spectral approximation techniques and matrix concentration bounds. We use Bernstein to understand the guarantees on the generalizatin performance of blabla.
%\end{itemize}

\section{Conclusion}
\label{sec:conclusion}
\input{conclusion}


%\section{Experiments}
%\label{sec:exp}
%We now evaluate the performance of our word embedding compression algorithm on a variety of NLP tasks and embeddings types.
We show that we are consistently able to match the performance of more complex baselines (DCCL, k-means), while dramatically outperforming more naive baselines (dimensionality reduction).
We additionally present results in which we maintain the memory budget fixed by  jointly varying the dimension and precision of the compressed embeddings;
we observe that in this memory-constrained setting, the best compressed embeddings attain \todo{XX\%} to \todo{YY\%} better performance than the full-precision embeddings.
This demonstrates the importance of considering low-precision when trying to attain the best possible performance under a memory budget.

\subsection{Large-scale evaluation}
\begin{itemize}
	\item \textbf{Embeddings}: GloVe ($d\in\{50,100,200,300\}$, $n=400k$) and fastText ($d=300$, $n=10^6$) pre-trained.
	\item \textbf{Baselines}: k-means, DCCL, dim. reduction (GloVe)
	\item \textbf{Tasks}: DrQA, sentiment, intrinsics, synthetics.
	\item \textbf{Compression ratios}: 8x-32x.
	\item \textbf{Number of random seeds}: 5
\end{itemize}

In Figure~\ref{fig:glove400k_drqa} we show that our uniform quantization method performs comparably to k-means and DCCL, while performing significantly better than the dimensionality reduction baseline.

\begin{figure}
\begin{center}
\centerline{\includegraphics[width=\columnwidth]{figures/glove400k_drqa_vs_compression.pdf}}
\caption{Question answering performance (DrQA) for a number of compression methods at various compression ratios, on pre-trained GloVe embedding.  Uniform quantization performs similarly to k-means and DCCL, while significantly outperforming the dimensionality reduction baseline. \todo{TODO: Include fastText results (though there won't be a dim. reduction baseline, since they don't release pre-trained embeddings at smaller dimensions.)}}
\label{fig:glove400k_drqa}
\end{center}
\end{figure}

\subsection{Fixed budget experiments}
\begin{itemize}
	\item \textbf{Embeddings}: We train GloVe embeddings on a Wiki 2017 dump, for $n=400k$ and $d \in \{25,50,100,200,400,800\}$.
	\item \textbf{Bitrates}: We compress embeddings for $b \in \{1,2,4,8,16,32\}$.
	\item \textbf{Tasks}: DrQA, sentiment, intrinsics, synthetics.
	\item \textbf{Number of random seeds}: 5
\end{itemize}

In Figure~\ref{fig:glove400k_dim_vs_prec}, we show that when considering a fixed memory budget, it is best to use low-precision and high-dimensions (\todo{up to a point? Exact results TBD}).

\begin{figure}
	\begin{center}
		\centerline{\includegraphics[width=\columnwidth]{figures/glove400k_dim_vs_prec.pdf}}
		\caption{We compress GloVe embeddings for dimensions $d\in\{25,50,100,200,400,800\}$ at precision values $b\in\{1,2,4,8,16,32\}$, and measure performance on question-answering (DrQA).  \todo{TODO: Include real results and discuss them.}}
		\label{fig:glove400k_dim_vs_prec}
	\end{center}
\end{figure}

%\section{Uniform Quantization of Word Embeddings}
%\label{sec:uniform}
%To further demonstrate how the eigenspace overlap metric can be used to better understand the performance of compressed embeddings, in this section we show that we can upper bound the eigenspace overlap for uniformly quantized embeddings.
Given the above theoretical and empirical connections between eigenspace overlap and generalization performance, these bounds help explain why the uniformly quantized embeddings perform so well.
To prove these bounds, we leverage the classic Davis-Kahan $\sin(\Theta)$ theorem from matrix perturbation analysis \citep{sintheta70}.
Because we know exactly what the noise structure of the uniform quantization method is, we can use this knowledge to bound how much the eigenspace of the compressed embeddings can differ from the uncompressed embeddings.

We now present the result (see Appendix~\ref{app:theory} for proof):
\begin{theorem}
	Let $X \in \RR^{n\times d}$ be a bounded embedding matrix with $X_{ij} \in [-\frac{1}{\sqrt{d}},\frac{1}{\sqrt{d}}]$ with largest and smallest singular values $\sigma_{\max}$ and $\sigma_{\min}$.
	Then the eigenspace overlap of the corresponding $b$-bit uniformly quantized embedding matrix can be lower bounded as follows:
	\begin{eqnarray*}
		\eigover(X,\tX) &\geq& 1 - \frac{16n}{d(2^b-1)^2} \Bigg( \frac{\sigma_{\max} + \frac{\sqrt{n}}{2^b-1} }{\sigma_{\min}^2} \Bigg)^2
	\end{eqnarray*}
\label{thm1}
\end{theorem}

We can further simplify this expression using the fact that $\sigma_{\max} = \|X\|_2 \leq \sqrt{n}$; using this fact, we get the following corollary:
\begin{corollary}
If $b \geq \log_2\bigg(\frac{8n}{\sigma_{\min}^2 \sqrt{\rho \, d} }} + 1\bigg)$, then the $b$-bit uniformly quantized embedding matrix $\tX$ satisfies $\eigover(X,\tX) \geq 1-\rho$.
\end{corollary}

The corollary shows that one can attain an eigenspace overlap arbitrarily close to $1$ if a sufficient number of bits $b$ are used.

\paragraph{Empirical Validation of Theory}
We now validate the above theory by showing how the eigenspace overlap grows as the precision of the uniformly quantized embeddings is increased.
For this experiment, we generate a random matrix $X \in \RR^{10^5 \times 300}$, with each entry drawn uniformly at random in the interval $[-\frac{1}{\sqrt{300}}, \frac{1}{\sqrt{300}}]$.
We then measure the eigenspace overlap of quantized versions of $X$, for precisions $b \in \{1,2,4,8,16,32\}$.
As we can see in Figure~\ref{fig:micro_eigoverlap_vs_prec}, the eigenspace overlap grows quite quickly as the precision grows, attaining a value of approximately $0.82$ at $b=2$, and $0.99$ at $b=4$.
This helps explain the strong empirical performance of the uniformly quantized embeddings which we observed at low precisions on the real word embedding experiments (Figure~\ref{fig:perf_comp}).

%\todo{Add section on clippings effect on the eigenspace overlap?}

\begin{figure}
	\begin{tabular}{c}
		\includegraphics[width=\linewidth]{figures/micro_eig_overlap_vs_precision.pdf} 
\end{tabular}
\caption{Empirical Validation of Theorem~\ref{thm1}. We measure the eigenspace overlap of uniformly quantized embeddings with the uncompressed embedding for various precisions ($b\in\{1,2,4,8,16,32\}$), and observe the eigenspace overlap grows quickly as the precision grows.  For the uncompressed embedding, we generate a random matrix $X \in \RR^{10^5 \times 300}$, with each entry drawn uniformly at random in $[-\frac{1}{\sqrt{300}}, \frac{1}{\sqrt{300}}]$.}
\label{fig:micro_eigoverlap_vs_prec}
\end{figure}

%\subsection{Theory validation}
%	\begin{itemize}
%		\item Impact of quantization on overlap 
%			\begin{itemize}
% 				\item Exp 1: overlap vs precision for different dimensionality. Expectation: overlap increases with higher precision.
%				\item Exp2: overlap vs dimensionality for different precision. Expectation: overlap increases with dimensionality. This explains that under fix memory budget, using lower bits quantization can be beneficial
%			\end{itemize}
%		\item The impact of clipping on eigen-subspace overlap
%			\begin{itemize}
%				\item Simulation based experiments on subspace overlap as a function of different clipping threshold and precision. 
%				\item The way we introduce this in: in our main theorem, we assume the dynamic range is $O(1/\sqrt{d})$ as a consequence of the automatic clipping. We want to show here this is the case in practice and then discuss the specific way clipping influence eigenspace overlap.
%			\end{itemize}
%		\item Loop-back discussion on the large scale empirical experiments in Section~\ref{subsec:hard_explain}
%	\end{itemize}

%
%\section{Experiments}
%\label{sec:experiments}
%We now evaluate the performance of our word embedding compression algorithm on a variety of NLP tasks and embeddings types.
We show that we are consistently able to match the performance of more complex baselines (DCCL, k-means), while dramatically outperforming more naive baselines (dimensionality reduction).
We additionally present results in which we maintain the memory budget fixed by  jointly varying the dimension and precision of the compressed embeddings;
we observe that in this memory-constrained setting, the best compressed embeddings attain \todo{XX\%} to \todo{YY\%} better performance than the full-precision embeddings.
This demonstrates the importance of considering low-precision when trying to attain the best possible performance under a memory budget.

\subsection{Large-scale evaluation}
\begin{itemize}
	\item \textbf{Embeddings}: GloVe ($d\in\{50,100,200,300\}$, $n=400k$) and fastText ($d=300$, $n=10^6$) pre-trained.
	\item \textbf{Baselines}: k-means, DCCL, dim. reduction (GloVe)
	\item \textbf{Tasks}: DrQA, sentiment, intrinsics, synthetics.
	\item \textbf{Compression ratios}: 8x-32x.
	\item \textbf{Number of random seeds}: 5
\end{itemize}

In Figure~\ref{fig:glove400k_drqa} we show that our uniform quantization method performs comparably to k-means and DCCL, while performing significantly better than the dimensionality reduction baseline.

\begin{figure}
\begin{center}
\centerline{\includegraphics[width=\columnwidth]{figures/glove400k_drqa_vs_compression.pdf}}
\caption{Question answering performance (DrQA) for a number of compression methods at various compression ratios, on pre-trained GloVe embedding.  Uniform quantization performs similarly to k-means and DCCL, while significantly outperforming the dimensionality reduction baseline. \todo{TODO: Include fastText results (though there won't be a dim. reduction baseline, since they don't release pre-trained embeddings at smaller dimensions.)}}
\label{fig:glove400k_drqa}
\end{center}
\end{figure}

\subsection{Fixed budget experiments}
\begin{itemize}
	\item \textbf{Embeddings}: We train GloVe embeddings on a Wiki 2017 dump, for $n=400k$ and $d \in \{25,50,100,200,400,800\}$.
	\item \textbf{Bitrates}: We compress embeddings for $b \in \{1,2,4,8,16,32\}$.
	\item \textbf{Tasks}: DrQA, sentiment, intrinsics, synthetics.
	\item \textbf{Number of random seeds}: 5
\end{itemize}

In Figure~\ref{fig:glove400k_dim_vs_prec}, we show that when considering a fixed memory budget, it is best to use low-precision and high-dimensions (\todo{up to a point? Exact results TBD}).

\begin{figure}
	\begin{center}
		\centerline{\includegraphics[width=\columnwidth]{figures/glove400k_dim_vs_prec.pdf}}
		\caption{We compress GloVe embeddings for dimensions $d\in\{25,50,100,200,400,800\}$ at precision values $b\in\{1,2,4,8,16,32\}$, and measure performance on question-answering (DrQA).  \todo{TODO: Include real results and discuss them.}}
		\label{fig:glove400k_dim_vs_prec}
	\end{center}
\end{figure}
%
%\section{Theory}
%\label{sec:theory}
%%\input{theory_old}
%%\input{jian_theory}
%In this section, we analyze the impact of uniform quantization on the generalization performance of linear regression models trained on top of the word embeddings.
We present two results:
First, we show that when the matrix is quantized to $b$ bits, and the linear ridge regression model is trained with regularization parameter $\lambda$, we attain good generalization bounds for the quantized features relative to the full-precision features when $2^b \lambda$ is large.
Second, we show that if the singular values of the embedding matrix decay slowly, then training the regression model with a large regularization parameter $\lambda$ cannot perform much worse than training with a smaller one.
Combining this theoretical result with our empirical observation that the singular values of embedding matrices typically decay slowly provides a principled explanation low-precision embeddings perform well on various tasks.

We begin, however, by reviewing generalization bounds for fixed design linear ridge regression.

\subsection{Background: Generalized Bounds for Fixed Design Linear Ridge Regression}
In fixed design linear regression, one is given a dataset $\{(x_i,y_i)\}_{i=1}^n$, for $x_i \in \RR^d$ and $y_i = \by_i + \eps_i \in \RR$, where the $\eps_i$ are independent random perturbations of the ``true labels'' $\by_i$ satisfying $\expect{}{\eps_i} = 0$ and $\var{}{\eps_i} = \sigma^2 < \infty$.
The goal is to design a training algorithm which takes as input the noisy dataset $\{(x_i,y_i)\}_{i=1}^n$ and outputs a model $f(x) = w^T x$ for which $\expect{}{\frac{1}{n}\sum_i (f(x_i) -\by_i)} \eqdef \cR(f)$ is small.
Linear ridge regression selects $w^* = \argmin_w \sum_i (w^T x_i - y_i)^2 + \lambda\|w\|_2^2$.
Letting $X \in \RR^{n\times d}$ be the matrix whose rows are $x_i$, $y \defeq (y_1,...,y_n) \in \RR^d$, and $I_d$ be the $d$-dimensional identity matrix, the minimizer of this optimization problem is $w^* = ( X^T X + \lambda I_d)^{-1}X^Ty$.
It is easy to show \citep{alaoui15} that the expected generalization error for the model $f_X(x) = w^{*T}x$ is
\begin{eqnarray*}
\cR(f_X) &=& \frac{\lambda^2}{n} \by^T(XX^T + \lambda I_n)^{-2}\by \; +\\ && \frac{\sigma^2}{n}\tr\Big((XX^T)^2(XX^T + \lambda I_n)^{-2}\Big).
\end{eqnarray*}

The question we ask in this work is: when we replace $X$ by an approximation $\tX$, how close will $\cR(f_{\tX})$ be to $\cR(f_X)$?
To answer this question, we intuitively require two things: a notion of distance between $X$ and $\tX$, and an upper bound for $\cR(f_{\tX})$ in terms of $\cR(f_X)$ and the distance between $X$ and $\tX$.
For both of these components, we leverage the recent work of \citep{lprff18}.
In that work, they define the following notion of distance between two matrices:

\begin{definition}{\citep{lprff18}}
	\label{def:specdist}
	For $\Delta_1, \Delta_2 \geq 0$, a symmetric matrix $A$ is a \emph{$(\Delta_1, \Delta_2)$-spectral approximation} of another symmetric matrix $B$ if $(1-\Delta_1)B \preceq A \preceq (1+\Delta_2)B$. 
\end{definition}

By applying this definition to $\tX\tX^T + \lambda I_n$ and $XX^T + \lambda I_n$, the authors prove the following generalization bound:
\begin{proposition}{(Adapted from \citep{lprff18})}
	Let $K \defeq XX^T$ and $\tK \defeq \tX\tX^T$, and suppose $\tK + \lambda I_n$ is $(\Delta_1, \Delta_2)$-spectral approximation of $K+\lambda I_n$, for $\Delta_1 \in [0,1)$, $\Delta_2 \geq 0$.
	Let $d$ denote the rank of $\tX$, and let $f_{X}$ and $f_{\tX}$ be the ridge regression estimators learned using these matrices, with regularizing constant $\lambda \geq 0$ and label noise variance $\sigma^2 < \infty$. Then
	\begin{equation}
	\cR(f_{\tX}) \leq \frac{1}{1-\Delta_1} \hcR(f_X) +  \frac{\Delta_2}{1+\Delta_2}\frac{d}{n}\sigma^2,
	\label{eq:risk_bound}
	\end{equation}
	where 
	\begin{eqnarray*}
	\hcR(f_X) &\defeq& \frac{\lambda}{n} \by^T(K+\lambda I)^{-1}\by + \frac{\sigma^2}{n}\tr\Big(K(K+\lambda I)^{-1}\Big) \\
	&\geq& \cR(f_X).
%	\label{eq:avron_rhat}
	\end{eqnarray*}
	\label{prop:genbound}
\end{proposition}

\subsection{Theoretical Results}
We now show that our compression algorithm with high probability produces an embedding matrix which is a close spectral approximation of the full-precision matrix, in terms of $(\Delta_1,\Delta_2)$.
Combining this with the Proposition~\ref{prop:genbound} yields a generalization bound for the compressed embeddings.
A consequence of our bound is that when the regularization parameter is large, lower precision can be used for the embeddings without affecting $\Delta_1$ and $\Delta_2$.
We show that in the case of word embeddings with slowly-decaying singular values, a large regularization parameter $\lambda$ (and thus, a low-precision $b$) can be used without significantly affecting generalization performance.
Our empirical observation that embedding matrices have slowly decaying singular values thus helps explain why our compression algorithm is able to attain strong generalization performance at very low levels of precision.

We now present our main theoretical result, deferring all proofs to the Appendix:
\begin{theorem}
	\label{thm:main}
	Let $X \in \RR^{n\times d}$ be an embedding matrix with corresponding Gram matrix $K \defeq XX^T$, where we assume all entries $X_{ij} \in [-\frac{1}{\sqrt{d}},\frac{1}{\sqrt{d}}]$; let $\tX\defeq X+C$ denote a $b$-bit quantization of $X$, with $\tK \defeq \tX\tX^T$ the kernel matrix of the quantized data matrix. Here, $C$ denotes the quantization noise, with $\expect{}{C_{ij}} = 0$ and $\var{}{C_{ij}} \leq \delta_b^2/d \;\;\forall i,j$, where $\delta_b^2 \defeq (2^b-1)^{-2}$, and $b$ is the number of bits used per feature.
	Then for any $\Delta_1 \geq 0, \Delta_2 \geq \delta^2_b/\lambda$,
	\begin{eqnarray*}
	&&\hspace{-0.37in}\Prob\Big[(1- \Delta_1) (K + \lambda I_n) \preceq \tK + \lambda I_n \preceq (1 + \Delta_2) (K + \lambda I_n)
	\Big] 
	\\ &\geq& 1 - 
	n \exp \bigg(\frac{-\Delta_1^2}{2dL^2 + (2L/3)\Delta_1}\bigg) \\
	&&- n \exp \bigg(\frac{-(\Delta_2-\delta_b^2/\lambda)^2}{2dL^2 + (2L/3)(\Delta_2-\delta_b^2/\lambda)}\bigg),
	\end{eqnarray*}
	for $L \defeq 5 \cdot \frac{2^b \cdot \delta_b^2}{\lambda}\cdot \frac{n}{d}$.
\end{theorem}
Note that in this theorem, we assume the embedding matrix is bounded, whereas in the actual implementation of our compression algorithm (Alg.~\ref{alg:smallfry}) we enforce this constraint artificially by searching for the optimal threshold at which to clip the matrix entries.
To better understand the implications of the above theorem, we present the following corollary:
\begin{corollary}
	\label{cor:main}
	If $\Delta_1 \geq \frac{\log(n/\rho)L}{3}\Big(1+\sqrt{1+\frac{18d}{\log(n/\rho)}}\Big) \approx \frac{5n}{2^b \lambda}\sqrt{\frac{2\log(n/\rho)}{d}}$,
	then $\Prob\big[(1 - \Delta_1) (K + \lambda I_n) \preceq \tK + \lambda I_n \big] \geq  1 - \rho$. 
	Similarly, if $\Delta_2 \geq \frac{\delta_b^2}{\lambda} +  \frac{\log(n/\rho)L}{3}\Big(1+\sqrt{1+\frac{18d}{\log(n/\rho)}}\Big) \approx \frac{1}{2^{2b}\lambda} + \frac{5n}{2^b \lambda}\sqrt{\frac{2\log(n/\rho)}{d}}$,
	then $\Prob\big[\tK + \lambda I_n \preceq (1 + \Delta_2) (K + \lambda I_n)\big] \geq  1 - \rho$. 
\end{corollary}

This corollary makes clear that if $2^b\lambda$ is large, the $b$-bit compressed embedding matrix will be a close spectral approximation of the full-precision matrix, for small $\Delta_1$ and $\Delta_2$.
In the following theorem, we show that when the smallest singular value of $X^TX$ is large, using a large regularization parameter $\lambda$ will not significantly harm the generalization performance of $f_X$, relative to using a smaller $\lambda$.
Combining this result with Proposition~\ref{prop:genbound} and Theorem~\ref{thm:main}, we can conclude that the generalization performance of a linear model trained on low-precision features with a large regularizer $\lambda$ will not be much worse than the performance of a model trained on the full-precision features with any $\lambda' \leq \lambda$.
We now present the result:

\begin{theorem}
	\label{thm:large_lambda}
	Let $X$ be an embedding matrix, and $\by$ be the corresponding vector of labels. Let $\sm$ be the smallest eigenvalue of $X^T X$, and let $\lambda_1, \lambda_2$ be two scalars such that $0 \leq \lambda_1 \leq \lambda_2 \leq a\cdot \sm$, for some $a \in [0,1]$. Letting $\cR_{\lambda}(K)$ denote the expected loss when training with regularizer $\lambda$, Gram matrix $K = XX^T$, and label noise $\sigma^2$, we get that:
	\begin{equation}
	R_{\lambda_2}(XX^T) - R_{\lambda_1}(XX^T) \leq a^2\|y\|^2/n
	\label{eq1}
	\end{equation}
\end{theorem}
Intuitively, this theorem shows that if there are no directions in the input space with very small variance, then using a large regularizer will not hurt performance significantly relative to using a smaller regularizer.
This makes sense because the directions of small variance are the ones which are effectively ignored when strong regularization is used.
For the purposes of this work, this theorem shows that if the embedding matrix has a large smallest eigenvalue, it should be possible to attain strong generalization performance with low-precision and a large regularizer.
In practice, we observe that it is common for embedding matrices to have slowly decaying spectra, and thus have a large smallest eigenvalue;
we present plots of the singular values for fastText and Glove embeddings in Figure~\ref{fig:real_spectra}.

\subsection{Empirical Validation of Theory}
In this section, we empirically validate two important predictions made by the above theoretical results:
\begin{enumerate}
	\item When $2^b \lambda$ is large, $\Delta_1$ and $\Delta_2$ are small (Figure~\ref{fig:micro_d1d2}).
	\item When $X^T X$ has a large smallest eigenvalue, generalization performance is not significantly harmed by using a large regularizer
	(Figure~\ref{fig:micro_large_sigma_min}).
\end{enumerate}

In addition, we show that in general word embedding matrices have a relatively large smallest singular value (Figure~\ref{fig:real_spectra}).

\begin{figure*}
	\centering
	\begin{tabular}{c c}
		%		\begin{tabular}{@{\hskip -0.0in}c@{\hskip -0.0in}c@{\hskip -0.0in}c@{\hskip -0.0in}}
		\includegraphics[width=0.4\linewidth]{figures/micro_uniform_nonadapt_delta1_vs_2_b_lambda.pdf} &	
		\includegraphics[width=0.4\linewidth]{figures/micro_uniform_nonadapt_delta2_vs_2_b_lambda.pdf}
	\end{tabular}
	\caption{We plot $\Delta_1$ (left) and $\Delta_2$ (right) as a function of $2^b\lambda$, on a randomly generated matrix $X\in\RR^{1000\times 30}$ ($X_{ij}\sim U([-\frac{1}{\sqrt{30}},\frac{1}{\sqrt{30}}])$), for various precisions $b$ and $\lambda$ values.  We show that $2^b \lambda$ largely determines the values of $\Delta_1$ and $\Delta_2$ after compression, as predicted by our theoretical results. We plot results for both deterministic quantization and stochastic quantization, and see that they perform quite similarly by these metrics (though deterministic does perform slightly better on $\Delta_2$). We additionally plot the bounds for $\Delta_1$ and $\Delta_2$ from Corollary~\ref{cor:main}, and see that while the bounds are not tight, their trajectory is matched by the real data.}
	\label{fig:micro_d1d2}
\end{figure*}

\begin{figure}
	\begin{center}
		\centerline{\includegraphics[width=0.8\columnwidth]{figures/micro_large_sigma_min.pdf}}
		\caption{We show that the ratio $a=\frac{\lambda}{\sigma_{min}}$ between the regularization parameter $\lambda$ and the smallest eigenvalue $\sigma_{min}$ of $X^T X$ is very predictive of the degradation in generalization performance $\|\by - y_{pred}\|^2/\|\by\|^2$, as predicted by Theorem~\ref{thm:large_lambda}.
		In particular, we generate an i.i.d. Gaussian matrix $X\in \RR^{1000\times 2}$ for various standard deviations $\sigma$, and consider the noiseless model $\by_i = [0,1]^T x_i$.
		We then solve for the optimal ridge regression model for various values of $\lambda$, and plot $a$ vs.\ the normalized degradation in generalization performance of this model. \todo{TODO: Perhaps plot a more realistic example?}
		}
		\label{fig:micro_large_sigma_min}
	\end{center}
\end{figure}

\begin{figure*}
	\centering
	\begin{tabular}{c c}
		%		\begin{tabular}{@{\hskip -0.0in}c@{\hskip -0.0in}c@{\hskip -0.0in}c@{\hskip -0.0in}}
		\includegraphics[width=0.4\linewidth]{figures/glove400k_spectra.pdf} &	
		\includegraphics[width=0.4\linewidth]{figures/fasttext1m_spectra.pdf}
	\end{tabular}
	\caption{We plot the real spectra of the pre-trained GloVe ($d \in \{50,100,200,300\}$) and fastText ($d=300$) embedding matrices.
	The smallest singular values are generally only 1 or 2 orders of magnitude smaller than the largest.}
	\label{fig:real_spectra}
\end{figure*}

\subsection{Effect of Clipping on $(\Delta_1,\Delta_2)$}
One way to understand why it is important for us to search for the optimal clipping value in Algorithm~\ref{alg:smallfry} is by understanding the way clipping and quantization together impact $\Delta_1$ and $\Delta_2$.
This perspective reveals that there is a fundamental trade-off between $\Delta_1$ and $\Delta_2$ when choosing the clipping value:
if the clip value is too large, then $\Delta_2$ becomes very large, while if the clip value is too small, then $\Delta_1$ becomes very large.
We demonstrate this in Figure~\ref{fig:deltas_vs_clip_quant}
\begin{figure}
	\begin{center}
		\centerline{\includegraphics[width=0.8\columnwidth]{figures/deltas_vs_clip_and_quant.pdf}}
		\caption{On random Gaussian data $X \in \RR^{1000\times 30}$, we demonstrate the joint effect of clipping and quantizing on $\Delta_1$ and $\Delta_2$, where we take $\lambda = \sigma_{min}(X^T X)/10$.
		A large clipping value gives large $\Delta_2$, whereas a small clipping values gives large $\Delta_1$.
		This reveals the importance of choosing a clipping value which balances these two considerations.
		}
		\label{fig:deltas_vs_clip_quant}
	\end{center}
\end{figure}


%\section{Discussion}
%\label{sec:discussion}
%\begin{itemize}
	\item Compare results to kernel paper.
	\item Discuss how this work provides insight on the question of word embedding optimal dimension.
\end{itemize}

% Acknowledgements should only appear in the accepted version.
%\section*{Acknowledgements}
%
%\textbf{Do not} include acknowledgements in the initial version of
%the paper submitted for blind review.

%\subsection{Figures}
% EXAMPLE FIGURE
%\begin{figure}[ht]
%\vskip 0.2in
%\begin{center}
%\centerline{\includegraphics[width=\columnwidth]{icml_numpapers}}
%\caption{Historical locations and number of accepted papers for International
%Machine Learning Conferences (ICML 1993 -- ICML 2008) and International
%Workshops on Machine Learning (ML 1988 -- ML 1992). At the time this figure was
%produced, the number of accepted papers for ICML 2008 was unknown and instead
%estimated.}
%\label{icml-historical}
%\end{center}
%\vskip -0.2in
%\end{figure}


%\subsection{Algorithms}
% EXAMPLE ALGORITHM
%\begin{algorithm}[tb]
%   \caption{Bubble Sort}
%   \label{alg:example}
%\begin{algorithmic}
%   \STATE {\bfseries Input:} data $x_i$, size $m$
%   \REPEAT
%   \STATE Initialize $noChange = true$.
%   \FOR{$i=1$ {\bfseries to} $m-1$}
%   \IF{$x_i > x_{i+1}$}
%   \STATE Swap $x_i$ and $x_{i+1}$
%   \STATE $noChange = false$
%   \ENDIF
%   \ENDFOR
%   \UNTIL{$noChange$ is $true$}
%\end{algorithmic}
%\end{algorithm}

%\subsection{Tables}
% EXAMPLE TABLE
%\begin{table}[t]
%\caption{Example table.}
%\label{sample-table}
%\vskip 0.15in
%\begin{center}
%\begin{small}
%\begin{sc}
%\begin{tabular}{lcccr}
%\toprule
%Data set & Naive & Flexible & Better? \\
%\midrule
%Cleveland & 83.3$\pm$ 0.6& 80.0$\pm$ 0.6& $\times$\\
%Glass2    & 61.9$\pm$ 1.4& 83.8$\pm$ 0.7& $\surd$ \\
%\bottomrule
%\end{tabular}
%\end{sc}
%\end{small}
%\end{center}
%\vskip -0.1in
%\end{table}


\bibliography{ref}
\bibliographystyle{icml2019}


%%%%%%%%%%%%%%%%%%%%%%%%%%%%%%%%%%%%%%%%%%%%%%%%%%%%%%%%%%%%%%%%%%%%%%%%%%%%%%%
%%%%%%%%%%%%%%%%%%%%%%%%%%%%%%%%%%%%%%%%%%%%%%%%%%%%%%%%%%%%%%%%%%%%%%%%%%%%%%%
% DELETE THIS PART. DO NOT PLACE CONTENT AFTER THE REFERENCES!
%%%%%%%%%%%%%%%%%%%%%%%%%%%%%%%%%%%%%%%%%%%%%%%%%%%%%%%%%%%%%%%%%%%%%%%%%%%%%%%
%%%%%%%%%%%%%%%%%%%%%%%%%%%%%%%%%%%%%%%%%%%%%%%%%%%%%%%%%%%%%%%%%%%%%%%%%%%%%%%
\onecolumn
\appendix
\section{Background}
\label{app:background}
\input{app_background}

\section{Experiments}
\label{app:experiments}
\subsection{Uniform Quantization Review}
A \textit{$b$-bit uniform quantization} $Q_{b,r}(x)$ of a real number $x \in [-r,r]$ is computed as follows:
First, the interval $[-r,r]$ is divided into $2^b - 1$ sub-intervals of equal size.
Then, $x$ is rounded to either the top or bottom of the sub-interval $[\ulx,\olx]$ containing $x$, where $\ulx = r + j\frac{2r}{2^b-1}$ and $\olx = r + (j+1)\frac{2r}{2^b-1}$, for $j\in\{0,1,\ldots,2^b-2\}$.
Given this rounded value, one can simply store the $b$-bit integer $j$ or $j+1$ in place of the real-valued $x$, depending on whether $x$ was rounded to $\ulx$ and $\olx$ respectively.
In this work, we will consider a deterministic rounding scheme which rounds $x$ to the nearest value, and a stochastic rounding scheme which rounds $x$ up or down in such a way that the expected value is equal to $x$.\footnote{
	This stochastic scheme rounds $x$ to $\ulx$ with probability $\frac{\olx-x}{\olx-\ulx}$ and to $\olx$ with probability $\frac{x-\ulx}{\olx-\ulx}$.
}
In particular, our analysis will focus on the stochastic rounding scheme, while our experiments will include results with both schemes.

%Note that we can upper bound the variance of these rounding schemes using the fact that a bounded random variable in an interval of length $c$ has variance at most $c^2/4$;
%using this fact, we can see that the variances of these rounding schemes are at most $\frac{1}{4} \cdot \Big(\frac{2r}{(2^b-1)}\Big)^2 = \frac{4r^2}{(2^b-1)^2}$.


We are now ready to present the uniform quantization compression algorithm, which we express in pseudo-code in Algorithm~\ref{alg:smallfry}.
The input to the algorithm is an embedding matrix $X \in \RR^{n\times d}$, where $n$ is the size of the vocabulary, and $d$ is the dimension of the embeddings.
We define the function $\clip_r(x) = \max(\min(x,r),-r)$ for any non-negative $r$; when matrices are passed in as inputs to this function, it clips the entries in an element-wise fashion.
The first step in our algorithm is to find the value of $r \in [0,\max(|X|)]$ which minimizes the reconstruction error of the quantized embeddings after $X$ is clipped to $[-r,r]$.
More formally, we let $r^* \defeq \argmin_{r \in [0,\max(|X|)} \|Q_{b,r}(\clip_r(X))-X\|_F$.
We then use this value $r^*$ to clip $X$, and then quantize the clipped embeddings to $b$ bits per entry.

In our experiments, we find $r^*$ to within a specified tolerance $\eps > 0$ using the golden-section search algorithm \citep{golden53}.
To avoid stochasticity impacting the search process, we always use deterministic rounding in the search for $r^*$, even if we use stochastic quantization in the final quantization.

\subsection{Experiment Details}
\label{app:experiment_details}


%this choice also allows us to more cleanly compare deterministic vs.\ stochastic rounding, since they will always use the same value of $r^*$.

\subsubsection{Details about Uncomppressed Word Embeddings We Used}
We use the Wikipedia 2014 + Gigaword 5 GloVe embeddings available at \url{http://nlp.stanford.edu/data/glove.6B.zip}, and the 300-dimensional fastText embeddings trained on Wikipedia 2017, UMBC webbase corpus and statmt.org news dataset, available at \url{https://s3-us-west-1.amazonaws.com/fasttext-vectors/wiki-news-300d-1M.vec.zip}.
For the GloVe embeddings which we trained, we used full English Wikimedia dump on Dec. 4, 2017 which was pre-processed by a fastText script~\footnote{https://github.com/facebookresearch/fastText/blob/master/get-wikimedia.sh} while keeping the letter cases and digits.
This corpus has 4.5 billion tokens (vocab size of $400k$).


\subsubsection{Empirical Comparison of Compression Methods}

\subsubsection{Dimension vs. Precision Trade-off}

\subsection{Extended Results}
\label{app:experiment_results}
In Figure~\ref{fig:all_sentiment} we present all our sentiment analysis results, for our different embedding types and the different sentiment analysis datasets.
\todo{I commented out large figure}
%\begin{figure}
%	\centering
%	\begin{tabular} {c c c}
%	% MPQA
%	\includegraphics[width=0.28\linewidth]{figures/fasttext1m_mpqa_test-err_vs_compression.pdf} &
%	\includegraphics[width=0.28\linewidth]{figures/glove400k_mpqa_test-err_vs_compression.pdf} &
%	\includegraphics[width=0.28\linewidth]{figures/glove-wiki400k-am_mpqa_test-err_vs_compression.pdf} \\[-0.5em]
%	% TREC
%	\includegraphics[width=0.28\linewidth]{figures/fasttext1m_trec_test-err_vs_compression.pdf} &
%	\includegraphics[width=0.28\linewidth]{figures/glove400k_trec_test-err_vs_compression.pdf} &
%	\includegraphics[width=0.28\linewidth]{figures/glove-wiki400k-am_trec_test-err_vs_compression.pdf} \\[-0.5em]
%	% SST
%	\includegraphics[width=0.28\linewidth]{figures/fasttext1m_sst_test-err_vs_compression.pdf} &
%	\includegraphics[width=0.28\linewidth]{figures/glove400k_sst_test-err_vs_compression.pdf} &
%	\includegraphics[width=0.28\linewidth]{figures/glove-wiki400k-am_sst_test-err_vs_compression.pdf} \\[-0.5em]
%	% CR
%	\includegraphics[width=0.28\linewidth]{figures/fasttext1m_cr_test-err_vs_compression.pdf} &
%	\includegraphics[width=0.28\linewidth]{figures/glove400k_cr_test-err_vs_compression.pdf} &
%	\includegraphics[width=0.28\linewidth]{figures/glove-wiki400k-am_cr_test-err_vs_compression.pdf} \\[-0.5em]
%	% SUBJ
%	\includegraphics[width=0.28\linewidth]{figures/fasttext1m_subj_test-err_vs_compression.pdf} &
%	\includegraphics[width=0.28\linewidth]{figures/glove400k_subj_test-err_vs_compression.pdf} &
%	\includegraphics[width=0.28\linewidth]{figures/glove-wiki400k-am_subj_test-err_vs_compression.pdf} \\[-0.5em]
%	% MR
%	\includegraphics[width=0.28\linewidth]{figures/fasttext1m_mr_test-err_vs_compression.pdf} &
%	\includegraphics[width=0.28\linewidth]{figures/glove400k_mr_test-err_vs_compression.pdf} &
%	\includegraphics[width=0.28\linewidth]{figures/glove-wiki400k-am_mr_test-err_vs_compression.pdf} \\[-0.5em]
%	\end{tabular}
%	\caption{
%		Sentiment analysis results for different embeddings methods (pre-trained fastText and GloVe embeddings, and GloVe embeddings trained from scratch), on different sentiment analysis datasets (MPQA, TREC, SST, CR, SUBJ, MR).
%	}
%	\label{fig:all_sentiment}
%\end{figure}
%
%\begin{figure*}
%	\centering
%	%	\begin{tabular}{c c c c}
%	\begin{tabular}{@{\hskip -0.0in}c@{\hskip -0.0in}c@{\hskip -0.0in}c@{\hskip -0.0in}c@{\hskip -0.0in}}
%		EIG-OVERLAP & . & . & . \\
%		\includegraphics[width=.245\linewidth]{figures/glove400k_qa_best-f1_vs_subspace-eig-overlap_linx.pdf} &
%		\includegraphics[width=.245\linewidth]{figures/glove400k_sentiment_trec_test-acc_vs_subspace-eig-overlap_linx.pdf} &
%		\includegraphics[width=.245\linewidth]{figures/glove400k_intrinsics_analogy-avg-score_vs_subspace-eig-overlap_linx.pdf} &
%		\includegraphics[width=.245\linewidth]{figures/glove400k_intrinsics_similarity-avg-score_vs_subspace-eig-overlap_linx.pdf} \\
%		
%		EIG-DISTANCE & . & . & . \\
%		\includegraphics[width=.245\linewidth]{figures/glove400k_qa_best-f1_vs_subspace-eig-distance_linx.pdf} &
%		\includegraphics[width=.245\linewidth]{figures/glove400k_sentiment_trec_test-acc_vs_subspace-eig-distance_linx.pdf} &
%		\includegraphics[width=.245\linewidth]{figures/glove400k_intrinsics_analogy-avg-score_vs_subspace-eig-distance_linx.pdf} &
%		\includegraphics[width=.245\linewidth]{figures/glove400k_intrinsics_similarity-avg-score_vs_subspace-eig-distance_linx.pdf} \\
%		
%		FROBENIUS ERROR & . & . & . \\		
%		\includegraphics[width=.245\linewidth]{figures/glove400k_qa_best-f1_vs_gram-large-dim-frob-error_linx.pdf} &
%		\includegraphics[width=.245\linewidth]{figures/glove400k_sentiment_trec_test-acc_vs_gram-large-dim-frob-error_linx.pdf} &
%		\includegraphics[width=.245\linewidth]{figures/glove400k_intrinsics_analogy-avg-score_vs_gram-large-dim-frob-error_linx.pdf} &
%		\includegraphics[width=.245\linewidth]{figures/glove400k_intrinsics_similarity-avg-score_vs_gram-large-dim-frob-error_linx.pdf} \\
%		
%		RECONSTRUCTION ERROR (FROB) & . & . & . \\		
%		\includegraphics[width=.245\linewidth]{figures/glove400k_qa_best-f1_vs_embed-frob-error_linx.pdf} &
%		\includegraphics[width=.245\linewidth]{figures/glove400k_sentiment_trec_test-acc_vs_embed-frob-error_linx.pdf} &
%		\includegraphics[width=.245\linewidth]{figures/glove400k_intrinsics_analogy-avg-score_vs_embed-frob-error_linx.pdf} &
%		\includegraphics[width=.245\linewidth]{figures/glove400k_intrinsics_similarity-avg-score_vs_embed-frob-error_linx.pdf} \\
%		
%		
%		%DELTA1 (Lambda = sigma min/100) & . & . & . \\
%		%\includegraphics[width=.245\linewidth]{figures/glove400k_qa_best-f1_vs_gram-large-dim-delta1-0-trans_linx.pdf} &
%		%\includegraphics[width=.245\linewidth]{figures/glove400k_sentiment_trec_test-acc_vs_gram-large-dim-delta1-0-trans_linx.pdf} &
%		%\includegraphics[width=.245\linewidth]{figures/glove400k_intrinsics_analogy-avg-score_vs_gram-large-dim-delta1-0-trans_linx.pdf} &
%		%\includegraphics[width=.245\linewidth]{figures/glove400k_intrinsics_similarity-avg-score_vs_gram-large-dim-delta1-0-trans_linx.pdf} \\
%		
%		DELTA1 (Lambda = sigma min) & . & . & . \\
%		\includegraphics[width=.245\linewidth]{figures/glove400k_qa_best-f1_vs_gram-large-dim-delta1-2-trans_linx.pdf} &
%		\includegraphics[width=.245\linewidth]{figures/glove400k_sentiment_trec_test-acc_vs_gram-large-dim-delta1-2-trans_linx.pdf} &
%		\includegraphics[width=.245\linewidth]{figures/glove400k_intrinsics_analogy-avg-score_vs_gram-large-dim-delta1-2-trans_linx.pdf} &
%		\includegraphics[width=.245\linewidth]{figures/glove400k_intrinsics_similarity-avg-score_vs_gram-large-dim-delta1-2-trans_linx.pdf} \\		
%		
%		DELTA1 (lambda = sigma max) & . & . & . \\
%		\includegraphics[width=.245\linewidth]{figures/glove400k_qa_best-f1_vs_gram-large-dim-delta1-6-trans_linx.pdf} &
%		\includegraphics[width=.245\linewidth]{figures/glove400k_sentiment_trec_test-acc_vs_gram-large-dim-delta1-6-trans_linx.pdf} &
%		\includegraphics[width=.245\linewidth]{figures/glove400k_intrinsics_analogy-avg-score_vs_gram-large-dim-delta1-6-trans_linx.pdf} &
%		\includegraphics[width=.245\linewidth]{figures/glove400k_intrinsics_similarity-avg-score_vs_gram-large-dim-delta1-6-trans_linx.pdf} \\		
%		
%		\;\;\;\;\;(a) & \;\;\;\;\;\;(b) & \;\;\;\;\;\;(c) & \;\;\;\;\;\;(d)
%	\end{tabular}
%	\caption{GLOVE400k: Performance vs. metrics. (a) QA, (b) Sentiment (TREC), (c) Analogy average, (d) Similarity average.
%	}
%	\label{fig:glove400k_comparison_results}
%\end{figure*}
%
%
%\begin{figure*}
%	\centering
%	%	\begin{tabular}{c c c c}
%	\begin{tabular}{@{\hskip -0.0in}c@{\hskip -0.0in}c@{\hskip -0.0in}c@{\hskip -0.0in}c@{\hskip -0.0in}}
%		EIG-OVERLAP & . & . & . \\
%		\includegraphics[width=.245\linewidth]{figures/glove-wiki400k-am_qa_best-f1_vs_subspace-eig-overlap_linx.pdf} &
%		\includegraphics[width=.245\linewidth]{figures/glove-wiki400k-am_sentiment_trec_test-acc_vs_subspace-eig-overlap_linx.pdf} &
%		\includegraphics[width=.245\linewidth]{figures/glove-wiki400k-am_intrinsics_analogy-avg-score_vs_subspace-eig-overlap_linx.pdf} &
%		\includegraphics[width=.245\linewidth]{figures/glove-wiki400k-am_intrinsics_similarity-avg-score_vs_subspace-eig-overlap_linx.pdf} \\
%		
%		EIG-DISTANCE & . & . & . \\
%		\includegraphics[width=.245\linewidth]{figures/glove-wiki400k-am_qa_best-f1_vs_subspace-eig-distance_linx.pdf} &
%		\includegraphics[width=.245\linewidth]{figures/glove-wiki400k-am_sentiment_trec_test-acc_vs_subspace-eig-distance_linx.pdf} &
%		\includegraphics[width=.245\linewidth]{figures/glove-wiki400k-am_intrinsics_analogy-avg-score_vs_subspace-eig-distance_linx.pdf} &
%		\includegraphics[width=.245\linewidth]{figures/glove-wiki400k-am_intrinsics_similarity-avg-score_vs_subspace-eig-distance_linx.pdf} \\
%		
%		FROBENIUS ERROR & . & . & . \\		
%		\includegraphics[width=.245\linewidth]{figures/glove-wiki400k-am_qa_best-f1_vs_gram-large-dim-frob-error_linx.pdf} &
%		\includegraphics[width=.245\linewidth]{figures/glove-wiki400k-am_sentiment_trec_test-acc_vs_gram-large-dim-frob-error_linx.pdf} &
%		\includegraphics[width=.245\linewidth]{figures/glove-wiki400k-am_intrinsics_analogy-avg-score_vs_gram-large-dim-frob-error_linx.pdf} &
%		\includegraphics[width=.245\linewidth]{figures/glove-wiki400k-am_intrinsics_similarity-avg-score_vs_gram-large-dim-frob-error_linx.pdf} \\
%		
%		
%		RECONSTRUCTION ERROR (FROB) & . & . & . \\		
%		\includegraphics[width=.245\linewidth]{figures/glove-wiki400k-am_qa_best-f1_vs_embed-frob-error_linx.pdf} &
%		\includegraphics[width=.245\linewidth]{figures/glove-wiki400k-am_sentiment_trec_test-acc_vs_embed-frob-error_linx.pdf} &
%		\includegraphics[width=.245\linewidth]{figures/glove-wiki400k-am_intrinsics_analogy-avg-score_vs_embed-frob-error_linx.pdf} &
%		\includegraphics[width=.245\linewidth]{figures/glove-wiki400k-am_intrinsics_similarity-avg-score_vs_embed-frob-error_linx.pdf} \\
%		%		DELTA1 (Lambda = sigma min/100) & . & . & . \\
%		%		\includegraphics[width=.245\linewidth]{figures/glove-wiki400k-am_qa_best-f1_vs_gram-large-dim-delta1-0-trans_linx.pdf} &
%		%		\includegraphics[width=.245\linewidth]{figures/glove-wiki400k-am_sentiment_trec_test-acc_vs_gram-large-dim-delta1-0-trans_linx.pdf} &
%		%		\includegraphics[width=.245\linewidth]{figures/glove-wiki400k-am_intrinsics_analogy-avg-score_vs_gram-large-dim-delta1-0-trans_linx.pdf} &
%		%		\includegraphics[width=.245\linewidth]{figures/glove-wiki400k-am_intrinsics_similarity-avg-score_vs_gram-large-dim-delta1-0-trans_linx.pdf} \\		
%		
%		DELTA1 (Lambda = sigma min) & . & . & . \\
%		\includegraphics[width=.245\linewidth]{figures/glove-wiki400k-am_qa_best-f1_vs_gram-large-dim-delta1-2-trans_linx.pdf} &
%		\includegraphics[width=.245\linewidth]{figures/glove-wiki400k-am_sentiment_trec_test-acc_vs_gram-large-dim-delta1-2-trans_linx.pdf} &
%		\includegraphics[width=.245\linewidth]{figures/glove-wiki400k-am_intrinsics_analogy-avg-score_vs_gram-large-dim-delta1-2-trans_linx.pdf} &
%		\includegraphics[width=.245\linewidth]{figures/glove-wiki400k-am_intrinsics_similarity-avg-score_vs_gram-large-dim-delta1-2-trans_linx.pdf} \\		
%		
%		DELTA1 (lambda = sigma max) & . & . & . \\
%		\includegraphics[width=.245\linewidth]{figures/glove-wiki400k-am_qa_best-f1_vs_gram-large-dim-delta1-6-trans_linx.pdf} &
%		\includegraphics[width=.245\linewidth]{figures/glove-wiki400k-am_sentiment_trec_test-acc_vs_gram-large-dim-delta1-6-trans_linx.pdf} &
%		\includegraphics[width=.245\linewidth]{figures/glove-wiki400k-am_intrinsics_analogy-avg-score_vs_gram-large-dim-delta1-6-trans_linx.pdf} &
%		\includegraphics[width=.245\linewidth]{figures/glove-wiki400k-am_intrinsics_similarity-avg-score_vs_gram-large-dim-delta1-6-trans_linx.pdf} \\		
%		
%		\;\;\;\;\;(a) & \;\;\;\;\;\;(b) & \;\;\;\;\;\;(c) & \;\;\;\;\;\;(d)
%	\end{tabular}
%	\caption{glove-wiki400k-am: Performance vs. metrics. (a) QA, (b) Sentiment (TREC), (c) Analogy average, (d) Similarity average.
%	}
%	\label{fig:glove_wiki400k_am_comparison_results}
%\end{figure*}
%
%
%\begin{figure*}
%	\centering
%	%	\begin{tabular}{c c c c}
%	\begin{tabular}{@{\hskip -0.0in}c@{\hskip -0.0in}c@{\hskip -0.0in}c@{\hskip -0.0in}c@{\hskip -0.0in}}
%		EIG-OVERLAP & . & . & . \\
%		\includegraphics[width=.245\linewidth]{figures/fasttext1m_qa_best-f1_vs_subspace-eig-overlap_linx.pdf} &
%		\includegraphics[width=.245\linewidth]{figures/fasttext1m_sentiment_trec_test-acc_vs_subspace-eig-overlap_linx.pdf} &
%		\includegraphics[width=.245\linewidth]{figures/fasttext1m_intrinsics_analogy-avg-score_vs_subspace-eig-overlap_linx.pdf} &
%		\includegraphics[width=.245\linewidth]{figures/fasttext1m_intrinsics_similarity-avg-score_vs_subspace-eig-overlap_linx.pdf} \\
%		
%		EIG-DISTANCE & . & . & . \\
%		\includegraphics[width=.245\linewidth]{figures/fasttext1m_qa_best-f1_vs_subspace-eig-distance_linx.pdf} &
%		\includegraphics[width=.245\linewidth]{figures/fasttext1m_sentiment_trec_test-acc_vs_subspace-eig-distance_linx.pdf} &
%		\includegraphics[width=.245\linewidth]{figures/fasttext1m_intrinsics_analogy-avg-score_vs_subspace-eig-distance_linx.pdf} &
%		\includegraphics[width=.245\linewidth]{figures/fasttext1m_intrinsics_similarity-avg-score_vs_subspace-eig-distance_linx.pdf} \\
%		
%		FROBENIUS ERROR & . & . & . \\		
%		\includegraphics[width=.245\linewidth]{figures/fasttext1m_qa_best-f1_vs_gram-large-dim-frob-error_linx.pdf} &
%		\includegraphics[width=.245\linewidth]{figures/fasttext1m_sentiment_trec_test-acc_vs_gram-large-dim-frob-error_linx.pdf} &
%		\includegraphics[width=.245\linewidth]{figures/fasttext1m_intrinsics_analogy-avg-score_vs_gram-large-dim-frob-error_linx.pdf} &
%		\includegraphics[width=.245\linewidth]{figures/fasttext1m_intrinsics_similarity-avg-score_vs_gram-large-dim-frob-error_linx.pdf} \\
%		
%		
%		RECONSTRUCTION ERROR (FROB) & . & . & . \\		
%		\includegraphics[width=.245\linewidth]{figures/fasttext1m_qa_best-f1_vs_embed-frob-error_linx.pdf} &
%		\includegraphics[width=.245\linewidth]{figures/fasttext1m_sentiment_trec_test-acc_vs_embed-frob-error_linx.pdf} &
%		\includegraphics[width=.245\linewidth]{figures/fasttext1m_intrinsics_analogy-avg-score_vs_embed-frob-error_linx.pdf} &
%		\includegraphics[width=.245\linewidth]{figures/fasttext1m_intrinsics_similarity-avg-score_vs_embed-frob-error_linx.pdf} \\
%		%		DELTA1 (Tiny lambda) & . & . & . \\
%		%		\includegraphics[width=.245\linewidth]{figures/fasttext1m_qa_best-f1_vs_gram-large-dim-delta1-0-trans_linx.pdf} &
%		%		\includegraphics[width=.245\linewidth]{figures/fasttext1m_sentiment_trec_test-acc_vs_gram-large-dim-delta1-0-trans_linx.pdf} &
%		%		\includegraphics[width=.245\linewidth]{figures/fasttext1m_intrinsics_analogy-avg-score_vs_gram-large-dim-delta1-0-trans_linx.pdf} &
%		%		\includegraphics[width=.245\linewidth]{figures/fasttext1m_intrinsics_similarity-avg-score_vs_gram-large-dim-delta1-0-trans_linx.pdf} \\		
%		
%		DELTA1 (Tiny lambda) & . & . & . \\
%		\includegraphics[width=.245\linewidth]{figures/fasttext1m_qa_best-f1_vs_gram-large-dim-delta1-2-trans_linx.pdf} &
%		\includegraphics[width=.245\linewidth]{figures/fasttext1m_sentiment_trec_test-acc_vs_gram-large-dim-delta1-2-trans_linx.pdf} &
%		\includegraphics[width=.245\linewidth]{figures/fasttext1m_intrinsics_analogy-avg-score_vs_gram-large-dim-delta1-2-trans_linx.pdf} &
%		\includegraphics[width=.245\linewidth]{figures/fasttext1m_intrinsics_similarity-avg-score_vs_gram-large-dim-delta1-2-trans_linx.pdf} \\		
%		
%		DELTA1 (lambda = sigma max) & . & . & . \\
%		\includegraphics[width=.245\linewidth]{figures/fasttext1m_qa_best-f1_vs_gram-large-dim-delta1-6-trans_linx.pdf} &
%		\includegraphics[width=.245\linewidth]{figures/fasttext1m_sentiment_trec_test-acc_vs_gram-large-dim-delta1-6-trans_linx.pdf} &
%		\includegraphics[width=.245\linewidth]{figures/fasttext1m_intrinsics_analogy-avg-score_vs_gram-large-dim-delta1-6-trans_linx.pdf} &
%		\includegraphics[width=.245\linewidth]{figures/fasttext1m_intrinsics_similarity-avg-score_vs_gram-large-dim-delta1-6-trans_linx.pdf} \\		
%		
%		\;\;\;\;\;(a) & \;\;\;\;\;\;(b) & \;\;\;\;\;\;(c) & \;\;\;\;\;\;(d)
%	\end{tabular}
%	\caption{fasttext1m: Performance vs. metrics. (a) QA, (b) Sentiment (TREC), (c) Analogy average, (d) Similarity average.
%	}
%	\label{fig:fasttext1m_comparison_results}
%\end{figure*}
%
%
%
%% SENTIMENT ANALYSIS
%
%\begin{figure*}
%	\centering
%	%	\begin{tabular}{c c c c}
%	\begin{tabular}{@{\hskip -0.0in}c@{\hskip -0.0in}c@{\hskip -0.0in}c@{\hskip -0.0in}c@{\hskip -0.0in}c@{\hskip -0.0in}}
%		EIG-OVERLAP & . & . & . & .\\
%		\includegraphics[width=.2\linewidth]{figures/glove400k_sentiment_mr_test-acc_vs_subspace-eig-overlap_linx.pdf} &
%		\includegraphics[width=.2\linewidth]{figures/glove400k_sentiment_subj_test-acc_vs_subspace-eig-overlap_linx.pdf} &
%		\includegraphics[width=.2\linewidth]{figures/glove400k_sentiment_cr_test-acc_vs_subspace-eig-overlap_linx.pdf} &
%		\includegraphics[width=.2\linewidth]{figures/glove400k_sentiment_sst_test-acc_vs_subspace-eig-overlap_linx.pdf} &
%		\includegraphics[width=.2\linewidth]{figures/glove400k_sentiment_mpqa_test-acc_vs_subspace-eig-overlap_linx.pdf} \\
%		
%		EIG-DISTANCE & . & . & . & .\\
%		\includegraphics[width=.2\linewidth]{figures/glove400k_sentiment_mr_test-acc_vs_subspace-eig-distance_linx.pdf} &
%		\includegraphics[width=.2\linewidth]{figures/glove400k_sentiment_subj_test-acc_vs_subspace-eig-distance_linx.pdf} &
%		\includegraphics[width=.2\linewidth]{figures/glove400k_sentiment_cr_test-acc_vs_subspace-eig-distance_linx.pdf} &
%		\includegraphics[width=.2\linewidth]{figures/glove400k_sentiment_sst_test-acc_vs_subspace-eig-distance_linx.pdf} &
%		\includegraphics[width=.2\linewidth]{figures/glove400k_sentiment_mpqa_test-acc_vs_subspace-eig-distance_linx.pdf} \\
%		
%		
%		FROBENIUS ERROR & . & . & . & .\\
%		\includegraphics[width=.2\linewidth]{figures/glove400k_sentiment_mr_test-acc_vs_gram-large-dim-frob-error_linx.pdf} &
%		\includegraphics[width=.2\linewidth]{figures/glove400k_sentiment_subj_test-acc_vs_gram-large-dim-frob-error_linx.pdf} &
%		\includegraphics[width=.2\linewidth]{figures/glove400k_sentiment_cr_test-acc_vs_gram-large-dim-frob-error_linx.pdf} &
%		\includegraphics[width=.2\linewidth]{figures/glove400k_sentiment_sst_test-acc_vs_gram-large-dim-frob-error_linx.pdf} &
%		\includegraphics[width=.2\linewidth]{figures/glove400k_sentiment_mpqa_test-acc_vs_gram-large-dim-frob-error_linx.pdf} \\
%		
%		RECONSTRUCTION ERROR (FROB) & . & . & . & .\\
%		\includegraphics[width=.2\linewidth]{figures/glove400k_sentiment_mr_test-acc_vs_embed-frob-error_linx.pdf} &
%		\includegraphics[width=.2\linewidth]{figures/glove400k_sentiment_subj_test-acc_vs_embed-frob-error_linx.pdf} &
%		\includegraphics[width=.2\linewidth]{figures/glove400k_sentiment_cr_test-acc_vs_embed-frob-error_linx.pdf} &
%		\includegraphics[width=.2\linewidth]{figures/glove400k_sentiment_sst_test-acc_vs_embed-frob-error_linx.pdf} &
%		\includegraphics[width=.2\linewidth]{figures/glove400k_sentiment_mpqa_test-acc_vs_embed-frob-error_linx.pdf} \\
%		
%		%		DELTA1 (Lambda = sigma min/100) & . & . & . & .\\
%		%		\includegraphics[width=.2\linewidth]{figures/glove400k_sentiment_mr_test-acc_vs_gram-large-dim-delta1-0-trans_linx.pdf} &
%		%		\includegraphics[width=.2\linewidth]{figures/glove400k_sentiment_subj_test-acc_vs_gram-large-dim-delta1-0-trans_linx.pdf} &
%		%		\includegraphics[width=.2\linewidth]{figures/glove400k_sentiment_cr_test-acc_vs_gram-large-dim-delta1-0-trans_linx.pdf} &
%		%		\includegraphics[width=.2\linewidth]{figures/glove400k_sentiment_sst_test-acc_vs_gram-large-dim-delta1-0-trans_linx.pdf} &
%		%		\includegraphics[width=.2\linewidth]{figures/glove400k_sentiment_mpqa_test-acc_vs_gram-large-dim-delta1-0-trans_linx.pdf} \\
%		
%		DELTA1 (Lambda = sigma min) & . & . & . & .\\
%		\includegraphics[width=.2\linewidth]{figures/glove400k_sentiment_mr_test-acc_vs_gram-large-dim-delta1-2-trans_linx.pdf} &
%		\includegraphics[width=.2\linewidth]{figures/glove400k_sentiment_subj_test-acc_vs_gram-large-dim-delta1-2-trans_linx.pdf} &
%		\includegraphics[width=.2\linewidth]{figures/glove400k_sentiment_cr_test-acc_vs_gram-large-dim-delta1-2-trans_linx.pdf} &
%		\includegraphics[width=.2\linewidth]{figures/glove400k_sentiment_sst_test-acc_vs_gram-large-dim-delta1-2-trans_linx.pdf} &
%		\includegraphics[width=.2\linewidth]{figures/glove400k_sentiment_mpqa_test-acc_vs_gram-large-dim-delta1-2-trans_linx.pdf} \\
%		
%		DELTA1 (Lambda = sigma max) & . & . & . & .\\
%		\includegraphics[width=.2\linewidth]{figures/glove400k_sentiment_mr_test-acc_vs_gram-large-dim-delta1-6-trans_linx.pdf} &
%		\includegraphics[width=.2\linewidth]{figures/glove400k_sentiment_subj_test-acc_vs_gram-large-dim-delta1-6-trans_linx.pdf} &
%		\includegraphics[width=.2\linewidth]{figures/glove400k_sentiment_cr_test-acc_vs_gram-large-dim-delta1-6-trans_linx.pdf} &
%		\includegraphics[width=.2\linewidth]{figures/glove400k_sentiment_sst_test-acc_vs_gram-large-dim-delta1-6-trans_linx.pdf} &
%		\includegraphics[width=.2\linewidth]{figures/glove400k_sentiment_mpqa_test-acc_vs_gram-large-dim-delta1-6-trans_linx.pdf} \\	
%		\;\;\;\;\;(a) & \;\;\;\;\;\;(b) & \;\;\;\;\;\;(c) & \;\;\;\;\;\;(d) & \;\;\;\;\;\;(e)
%	\end{tabular}
%	\caption{GLOVE400k: Performance vs. metrics, five sentiment tasks (MR, SUBJ, CR, SST, MPQA).
%	}
%	\label{fig:glove400k_sent_comparison_results}
%\end{figure*}
%
%
%
%\begin{figure*}
%	\centering
%	%	\begin{tabular}{c c c c}
%	\begin{tabular}{@{\hskip -0.0in}c@{\hskip -0.0in}c@{\hskip -0.0in}c@{\hskip -0.0in}c@{\hskip -0.0in}c@{\hskip -0.0in}}
%		EIG-OVERLAP & . & . & . & .\\
%		\includegraphics[width=.2\linewidth]{figures/glove-wiki400k-am_sentiment_mr_test-acc_vs_subspace-eig-overlap_linx.pdf} &
%		\includegraphics[width=.2\linewidth]{figures/glove-wiki400k-am_sentiment_subj_test-acc_vs_subspace-eig-overlap_linx.pdf} &
%		\includegraphics[width=.2\linewidth]{figures/glove-wiki400k-am_sentiment_cr_test-acc_vs_subspace-eig-overlap_linx.pdf} &
%		\includegraphics[width=.2\linewidth]{figures/glove-wiki400k-am_sentiment_sst_test-acc_vs_subspace-eig-overlap_linx.pdf} &
%		\includegraphics[width=.2\linewidth]{figures/glove-wiki400k-am_sentiment_mpqa_test-acc_vs_subspace-eig-overlap_linx.pdf} \\
%		
%		EIG-DISTANCE & . & . & . & .\\
%		\includegraphics[width=.2\linewidth]{figures/glove-wiki400k-am_sentiment_mr_test-acc_vs_subspace-eig-distance_linx.pdf} &
%		\includegraphics[width=.2\linewidth]{figures/glove-wiki400k-am_sentiment_subj_test-acc_vs_subspace-eig-distance_linx.pdf} &
%		\includegraphics[width=.2\linewidth]{figures/glove-wiki400k-am_sentiment_cr_test-acc_vs_subspace-eig-distance_linx.pdf} &
%		\includegraphics[width=.2\linewidth]{figures/glove-wiki400k-am_sentiment_sst_test-acc_vs_subspace-eig-distance_linx.pdf} &
%		\includegraphics[width=.2\linewidth]{figures/glove-wiki400k-am_sentiment_mpqa_test-acc_vs_subspace-eig-distance_linx.pdf} \\
%		
%		
%		FROBENIUS ERROR & . & . & . & .\\
%		\includegraphics[width=.2\linewidth]{figures/glove-wiki400k-am_sentiment_mr_test-acc_vs_gram-large-dim-frob-error_linx.pdf} &
%		\includegraphics[width=.2\linewidth]{figures/glove-wiki400k-am_sentiment_subj_test-acc_vs_gram-large-dim-frob-error_linx.pdf} &
%		\includegraphics[width=.2\linewidth]{figures/glove-wiki400k-am_sentiment_cr_test-acc_vs_gram-large-dim-frob-error_linx.pdf} &
%		\includegraphics[width=.2\linewidth]{figures/glove-wiki400k-am_sentiment_sst_test-acc_vs_gram-large-dim-frob-error_linx.pdf} &
%		\includegraphics[width=.2\linewidth]{figures/glove-wiki400k-am_sentiment_mpqa_test-acc_vs_gram-large-dim-frob-error_linx.pdf} \\
%		
%		RECONSTRUCTION ERROR (FROB) & . & . & . & .\\
%		\includegraphics[width=.2\linewidth]{figures/glove-wiki400k-am_sentiment_mr_test-acc_vs_embed-frob-error_linx.pdf} &
%		\includegraphics[width=.2\linewidth]{figures/glove-wiki400k-am_sentiment_subj_test-acc_vs_embed-frob-error_linx.pdf} &
%		\includegraphics[width=.2\linewidth]{figures/glove-wiki400k-am_sentiment_cr_test-acc_vs_embed-frob-error_linx.pdf} &
%		\includegraphics[width=.2\linewidth]{figures/glove-wiki400k-am_sentiment_sst_test-acc_vs_embed-frob-error_linx.pdf} &
%		\includegraphics[width=.2\linewidth]{figures/glove-wiki400k-am_sentiment_mpqa_test-acc_vs_embed-frob-error_linx.pdf} \\
%		
%		%		DELTA1 (Lambda = sigma min/100) & . & . & . & .\\
%		%		\includegraphics[width=.2\linewidth]{figures/glove-wiki400k-am_sentiment_mr_test-acc_vs_gram-large-dim-delta1-0-trans_linx.pdf} &
%		%		\includegraphics[width=.2\linewidth]{figures/glove-wiki400k-am_sentiment_subj_test-acc_vs_gram-large-dim-delta1-0-trans_linx.pdf} &
%		%		\includegraphics[width=.2\linewidth]{figures/glove-wiki400k-am_sentiment_cr_test-acc_vs_gram-large-dim-delta1-0-trans_linx.pdf} &
%		%		\includegraphics[width=.2\linewidth]{figures/glove-wiki400k-am_sentiment_sst_test-acc_vs_gram-large-dim-delta1-0-trans_linx.pdf} &
%		%		\includegraphics[width=.2\linewidth]{figures/glove-wiki400k-am_sentiment_mpqa_test-acc_vs_gram-large-dim-delta1-0-trans_linx.pdf} \\
%		
%		DELTA1 (Lambda = sigma min) & . & . & . & .\\
%		\includegraphics[width=.2\linewidth]{figures/glove-wiki400k-am_sentiment_mr_test-acc_vs_gram-large-dim-delta1-2-trans_linx.pdf} &
%		\includegraphics[width=.2\linewidth]{figures/glove-wiki400k-am_sentiment_subj_test-acc_vs_gram-large-dim-delta1-2-trans_linx.pdf} &
%		\includegraphics[width=.2\linewidth]{figures/glove-wiki400k-am_sentiment_cr_test-acc_vs_gram-large-dim-delta1-2-trans_linx.pdf} &
%		\includegraphics[width=.2\linewidth]{figures/glove-wiki400k-am_sentiment_sst_test-acc_vs_gram-large-dim-delta1-2-trans_linx.pdf} &
%		\includegraphics[width=.2\linewidth]{figures/glove-wiki400k-am_sentiment_mpqa_test-acc_vs_gram-large-dim-delta1-2-trans_linx.pdf} \\
%		
%		DELTA1 (Lambda = sigma max) & . & . & . & .\\
%		\includegraphics[width=.2\linewidth]{figures/glove-wiki400k-am_sentiment_mr_test-acc_vs_gram-large-dim-delta1-6-trans_linx.pdf} &
%		\includegraphics[width=.2\linewidth]{figures/glove-wiki400k-am_sentiment_subj_test-acc_vs_gram-large-dim-delta1-6-trans_linx.pdf} &
%		\includegraphics[width=.2\linewidth]{figures/glove-wiki400k-am_sentiment_cr_test-acc_vs_gram-large-dim-delta1-6-trans_linx.pdf} &
%		\includegraphics[width=.2\linewidth]{figures/glove-wiki400k-am_sentiment_sst_test-acc_vs_gram-large-dim-delta1-6-trans_linx.pdf} &
%		\includegraphics[width=.2\linewidth]{figures/glove-wiki400k-am_sentiment_mpqa_test-acc_vs_gram-large-dim-delta1-6-trans_linx.pdf} \\	
%		\;\;\;\;\;(a) & \;\;\;\;\;\;(b) & \;\;\;\;\;\;(c) & \;\;\;\;\;\;(d) & \;\;\;\;\;\;(e)
%	\end{tabular}
%	\caption{glove-wiki400k-am: Performance vs. metrics, five sentiment tasks (MR, SUBJ, CR, SST, MPQA).
%	}
%	\label{fig:glove_wiki400k_am_sent_comparison_results}
%\end{figure*}
%
%\begin{figure*}
%	\centering
%	%	\begin{tabular}{c c c c}
%	\begin{tabular}{@{\hskip -0.0in}c@{\hskip -0.0in}c@{\hskip -0.0in}c@{\hskip -0.0in}c@{\hskip -0.0in}c@{\hskip -0.0in}}
%		EIG-OVERLAP & . & . & . & .\\
%		\includegraphics[width=.2\linewidth]{figures/fasttext1m_sentiment_mr_test-acc_vs_subspace-eig-overlap_linx.pdf} &
%		\includegraphics[width=.2\linewidth]{figures/fasttext1m_sentiment_subj_test-acc_vs_subspace-eig-overlap_linx.pdf} &
%		\includegraphics[width=.2\linewidth]{figures/fasttext1m_sentiment_cr_test-acc_vs_subspace-eig-overlap_linx.pdf} &
%		\includegraphics[width=.2\linewidth]{figures/fasttext1m_sentiment_sst_test-acc_vs_subspace-eig-overlap_linx.pdf} &
%		\includegraphics[width=.2\linewidth]{figures/fasttext1m_sentiment_mpqa_test-acc_vs_subspace-eig-overlap_linx.pdf} \\
%		
%		EIG-DISTANCE & . & . & . & .\\
%		\includegraphics[width=.2\linewidth]{figures/fasttext1m_sentiment_mr_test-acc_vs_subspace-eig-distance_linx.pdf} &
%		\includegraphics[width=.2\linewidth]{figures/fasttext1m_sentiment_subj_test-acc_vs_subspace-eig-distance_linx.pdf} &
%		\includegraphics[width=.2\linewidth]{figures/fasttext1m_sentiment_cr_test-acc_vs_subspace-eig-distance_linx.pdf} &
%		\includegraphics[width=.2\linewidth]{figures/fasttext1m_sentiment_sst_test-acc_vs_subspace-eig-distance_linx.pdf} &
%		\includegraphics[width=.2\linewidth]{figures/fasttext1m_sentiment_mpqa_test-acc_vs_subspace-eig-distance_linx.pdf} \\
%		
%		
%		FROBENIUS ERROR & . & . & . & .\\
%		\includegraphics[width=.2\linewidth]{figures/fasttext1m_sentiment_mr_test-acc_vs_gram-large-dim-frob-error_linx.pdf} &
%		\includegraphics[width=.2\linewidth]{figures/fasttext1m_sentiment_subj_test-acc_vs_gram-large-dim-frob-error_linx.pdf} &
%		\includegraphics[width=.2\linewidth]{figures/fasttext1m_sentiment_cr_test-acc_vs_gram-large-dim-frob-error_linx.pdf} &
%		\includegraphics[width=.2\linewidth]{figures/fasttext1m_sentiment_sst_test-acc_vs_gram-large-dim-frob-error_linx.pdf} &
%		\includegraphics[width=.2\linewidth]{figures/fasttext1m_sentiment_mpqa_test-acc_vs_gram-large-dim-frob-error_linx.pdf} \\
%		
%		RECONSTRUCTION ERROR (FROB) & . & . & . & .\\
%		\includegraphics[width=.2\linewidth]{figures/fasttext1m_sentiment_mr_test-acc_vs_embed-frob-error_linx.pdf} &
%		\includegraphics[width=.2\linewidth]{figures/fasttext1m_sentiment_subj_test-acc_vs_embed-frob-error_linx.pdf} &
%		\includegraphics[width=.2\linewidth]{figures/fasttext1m_sentiment_cr_test-acc_vs_embed-frob-error_linx.pdf} &
%		\includegraphics[width=.2\linewidth]{figures/fasttext1m_sentiment_sst_test-acc_vs_embed-frob-error_linx.pdf} &
%		\includegraphics[width=.2\linewidth]{figures/fasttext1m_sentiment_mpqa_test-acc_vs_embed-frob-error_linx.pdf} \\
%		%		DELTA1 (Lambda = sigma min/100) & . & . & . & .\\
%		%		\includegraphics[width=.2\linewidth]{figures/fasttext1m_sentiment_mr_test-acc_vs_gram-large-dim-delta1-0-trans_linx.pdf} &
%		%		\includegraphics[width=.2\linewidth]{figures/fasttext1m_sentiment_subj_test-acc_vs_gram-large-dim-delta1-0-trans_linx.pdf} &
%		%		\includegraphics[width=.2\linewidth]{figures/fasttext1m_sentiment_cr_test-acc_vs_gram-large-dim-delta1-0-trans_linx.pdf} &
%		%		\includegraphics[width=.2\linewidth]{figures/fasttext1m_sentiment_sst_test-acc_vs_gram-large-dim-delta1-0-trans_linx.pdf} &
%		%		\includegraphics[width=.2\linewidth]{figures/fasttext1m_sentiment_mpqa_test-acc_vs_gram-large-dim-delta1-0-trans_linx.pdf} \\
%		
%		DELTA1 (Lambda = sigma min) & . & . & . & .\\
%		\includegraphics[width=.2\linewidth]{figures/fasttext1m_sentiment_mr_test-acc_vs_gram-large-dim-delta1-2-trans_linx.pdf} &
%		\includegraphics[width=.2\linewidth]{figures/fasttext1m_sentiment_subj_test-acc_vs_gram-large-dim-delta1-2-trans_linx.pdf} &
%		\includegraphics[width=.2\linewidth]{figures/fasttext1m_sentiment_cr_test-acc_vs_gram-large-dim-delta1-2-trans_linx.pdf} &
%		\includegraphics[width=.2\linewidth]{figures/fasttext1m_sentiment_sst_test-acc_vs_gram-large-dim-delta1-2-trans_linx.pdf} &
%		\includegraphics[width=.2\linewidth]{figures/fasttext1m_sentiment_mpqa_test-acc_vs_gram-large-dim-delta1-2-trans_linx.pdf} \\
%		
%		DELTA1 (Lambda = sigma max) & . & . & . & .\\
%		\includegraphics[width=.2\linewidth]{figures/fasttext1m_sentiment_mr_test-acc_vs_gram-large-dim-delta1-6-trans_linx.pdf} &
%		\includegraphics[width=.2\linewidth]{figures/fasttext1m_sentiment_subj_test-acc_vs_gram-large-dim-delta1-6-trans_linx.pdf} &
%		\includegraphics[width=.2\linewidth]{figures/fasttext1m_sentiment_cr_test-acc_vs_gram-large-dim-delta1-6-trans_linx.pdf} &
%		\includegraphics[width=.2\linewidth]{figures/fasttext1m_sentiment_sst_test-acc_vs_gram-large-dim-delta1-6-trans_linx.pdf} &
%		\includegraphics[width=.2\linewidth]{figures/fasttext1m_sentiment_mpqa_test-acc_vs_gram-large-dim-delta1-6-trans_linx.pdf} \\	
%		\;\;\;\;\;(a) & \;\;\;\;\;\;(b) & \;\;\;\;\;\;(c) & \;\;\;\;\;\;(d) & \;\;\;\;\;\;(e)
%	\end{tabular}
%	\caption{fasttext1m: Performance vs. metrics, five sentiment tasks (MR, SUBJ, CR, SST, MPQA).
%	}
%	\label{fig:fastext1m_sent_comparison_results}
%\end{figure*}

\section{Theoretical Results}
\label{app:theory}
\subsection{NEW THEORY}
Assume $K = USU^T \in \RR^{n \times n}$ (rank $d$), $\tK = VRV^T\in \RR^{n \times n}$ (rank $k$).
Let $\{U_1,\ldots,U_n\}$ be the eigenvectors of $K$ in decreasing eigenvalue order, with the first $d$ having non-zero eigenvalue.
Similarly, let $\{V_1,\ldots,V_n\}$ be the eigenvectors of $\tK$ in decreasing eigenvalue order, with the first $k$ having non-zero eigenvalue.
Let $U = [U_1,\ldots U_d]$, $U_{\perp} = [U_{d+1},\ldots,U_n]$, $V = [V_1,\ldots V_k]$, $V_{\perp} = [V_{k+1},\ldots,V_n]$.

\begin{eqnarray*}
	R(K) &=& \frac{\lambda^2}{n}y^T (K+\lambda I)^{-2} y + \frac{\sigma^2}{n} \tr\bigg(K^2(K+\lambda I)^{-2}\bigg) \\
	n\cdot R(K) &=& \sum_{i=1}^d \Big(\frac{\lambda}{\sigma_i + \lambda}\Big)^2(U_i^T y)^2 + \sum_{i=d+1}^n (U_i^T y)^2 + \sigma^2 \sum_{i=1}^d \Big(\frac{\sigma_i}{\sigma_i + \lambda}\Big)^2 \\
\end{eqnarray*}
If we assume that we are in the noiseless linear regression setting ($\sigma=0$, $\lambda=0$), instead of the noisy linear ridge regression setting, this simplifies:
\begin{eqnarray*}
	n\cdot R(K) &=& \sum_{i=d+1}^n (U_i^T y)^2 \\
	&=& y^T U_{\perp} U_{\perp}^T y \\
	&=& y^T (I - UU^T) y \\
	&=& \|y\|^2 - y^T UU^T y \\
	n\cdot R(\tK) &=& \|y\|^2 - y^T VV^T y \\
	n(R(\tK) - R(K)) &=& \Big(\|y\|^2 - y^T VV^T y\Big) - \Big(\|y\|^2 - y^T UU^T y\Big)\\
	&=& y^T (UU^T - VV^T) y \\
	&\leq& \|y\|^2 \|UU^T - VV^T\|_2 \\
	&\leq& \|y\|^2 \|UU^T - VV^T\|_F \\
	&\leq& \|y\|^2 \sqrt{d + k - 2 \|U^T V\|_F^2} \quad \text{(by Lemma\ref{lemma1})}\\
\end{eqnarray*}

ANOTHER VERSION\\
Here we consider a random $y \in \RR^n$ vector with identity covariance.
\begin{eqnarray*}
	\expect{y}{n(R(\tK) - R(K))} &=& \expect{y}{\Big(\|y\|^2 - y^T VV^T y\Big) - \Big(\|y\|^2 - y^T UU^T y\Big)}\\
	&=& \expect{y}{y^T (UU^T - VV^T) y} \\
	&=& \expect{y}{y^T UU^T y - y^TVV^T y} \\
	&=& \|U\|_F^2  - \expect{y}{y^TVV^T y} \\
	&\leq& d - \|V^T U\|_F^2 \quad \text{by derivation below.} \\
	\expect{y}{n(R(K) - R(\tK))} &\leq& k - \|V^T U\|_F^2 \\
	\expect{y}{y^T VV^T y} &=& \expect{y}{\|V^T y\|_2^2} \\
	&=& \expect{y}{\|VV^T y\|_2^2} \\
	&\geq&  \expect{y}{\|VV^T UU^T y\|_2^2}  \quad\text{(projecting onto $U$ first makes $y$ magnitude smaller)}\\
	&=& \|VV^T UU^T\|_F^2 \quad \text{(by Lemma\ref{lemma2})}\\
	&=& \tr\Big(UU^TVV^TVV^TUU^T \Big) \\
	&=& \tr\Big((U^TV) V^TV (V^TU) U^TU \Big) \\
	&=& \tr \Big( (V^T U)^T (V^T U)\Big) \\
	&=& \|V^T U\|_F^2
\end{eqnarray*}

ONE MORE VERSION
\begin{eqnarray*}
	n(R(\tK) - R(K)) &=& \Big(\|y\|^2 - y^T VV^T y\Big) - \Big(\|y\|^2 - y^T UU^T y\Big)\\
	&=& y^T (UU^T - VV^T) y \\
	&=& y^T UU^T y - y^TVV^T y \\
	&=& \text{add line here.} \\
	y^T VV^T y &=& \|V^T y\|_2^2 \\
	&=& \|VV^T y\|_2^2 \\
	&\geq&  \|VV^T UU^T y\|_2^2  \quad\text{(projecting onto $U$ first makes $y$ magnitude smaller)}\\
	&=&  \|(V^T U) (U^T y)\|_2^2 \\
	&\geq& \sigma_{min}(V^T U)^2 \|U^T y\|^2 \\
%	&=& \tr\Big(y^T UU^TVV^TVV^TUU^T y \Big) \\
%	&=& \tr\Big(UU^TVV^TVV^TUU^T yy^T \Big) \\
%	&=& \tr\Big(U (U^TV) (V^TU) U^T yy^T \Big) \\
%	&=& \tr\Big((U^TV) (V^TU) (U^T y)(U^T y)^T \Big) \\
%	&\geq& \text{NOT SURE HOW TO LOWER BOUND}
%	&=& \tr\Big((U^TV) V^TV (V^TU) U^TU \Big) \\
%	&=& \tr \Big( (V^T U)^T (V^T U)\Big) \\
%	&=& \|V^T U\|_F^2
\end{eqnarray*}
So the worst case analysis cares about the minimum singular value of $V^T U$. \todo{But this will be 0 if rank of $V$ is smaller than rank of $U$, right??}

ONE MORE:\\
Tri says we can use the fact that the singular values of $UU^T-VV^T$ are the same as those of $U^{\perp} U^{\perp T} VV^T$ (but repeated twice).  And then use sin(theta) theorem to argue that $\|U^{\perp} U^{\perp T} VV^T\|$ is small


\begin{lemma}
\label{lemma1}
\begin{eqnarray*}
	\|UU^T - VV^T \|_F^2 &=& \tr\Big( (UU^T - VV^T)^T (UU^T - VV^T)  \Big)\\
	&=& \tr(UU^TUU^T) + \tr(VV^T VV^T) - \tr(VV^T UU^T) - \tr(UU^T VV^T) \\
	&=& d + k - 2 \tr((U^T V)^T (U^T V)) \\
	&=& d + k - 2 \|U^T V\|_F^2
\end{eqnarray*}
\end{lemma}

\begin{lemma}
\label{lemma2}
Let $A \in \RR^{n\times d}$ be an arbitrary matrix. If a random vector $x \in \RR^d$ is drawn from a distribution with identiy covariance matrix, then $\|A\|_F^2 = \expect{x}{\|Ax\|_2^2}$.
\end{lemma}
\begin{proof}
\begin{eqnarray*}
\expect{x}{\|Ax\|_2^2} &=& \expect{x}{(Ax)^T (Ax)} \\
&=& \expect{x}{x^T A^T A x} \\
&=& \expect{x}{\tr \Big(x^T A^T A x\Big)} \\
&=& \expect{x}{\tr \Big(A^T A x x^T\Big)} \\
&=& \tr \Big(A^T A \cdot \expect{x}{x x^T}\Big) \\
&=& \tr \Big(A^T A\Big) \\
&=& \|A\|_F^2 \\
\end{eqnarray*}
\end{proof}

Let $\{a_i\}_{i=1}^r$ be an orthonormal basis of $U \backslash V$, $\{b_j\}_{i=1}^s$ be an orthonormal basis of $V / U$, and $\{c_k\}_{i=1}^t$ be an orthonormal basis of $U \cap V$.
\begin{eqnarray*}
	\|UU^T - VV^T \|_F^2 &=& \sum_{i=1}^r \|(UU^T - VV^T)a_i\|^2 + \sum_{j=1}^s \|(UU^T - VV^T)b_j\|^2 + \sum_{k=1}^t \|(UU^T - VV^T)c_k\|^2 \\
	&=& r + s
\end{eqnarray*}

\subsection{Theory Summary}
We have observed that word embedding matrices have pretty flat spectra, with large smallest non-zero singular values.  Here, we show that this implies that one can use a large regularizer $\lambda$ while attaining similar generalization performance to smaller $\lambda$.  We then show that large $\lambda$ implies small relative spectral distance between the full-precision embedding matrix and the low-precision embedding matrix with high probability.  We know from prior work that small relative spectral distance gives tight generalization bounds.

%High-level logic:
%\begin{enumerate}
%	\item If the optimal regularizer $\lambda^* \in [0,a\sm]$ for some $a \in [0,1]$, we show that we can actually attain similar generalization performance if we use $\lambda = a\sm$ instead of $\lambda^*$.  Using a large regularizer $\lambda = a\sm$ allows us to with high probability have small relative spectral distance between $ZZ^T$ and $(Z+C)(Z+C)^T$ (with respect to this $\lambda$).  This gives us a bound on the generalization performance of $(Z+C)(Z+C)^T$ using this $\lambda$, relative to (an upper bound on) the generalization performance of $ZZ^T$ with this $\lambda$.  Because we know that training with $a\sm$ on $ZZ^T$ gives performance close to optimal, this bounds the performance of the quantized embeddings with $\lambda = a\sm$ in terms of the best possible generalization performance of the full-precision embeddings.
%	\item If the optimal $\lambda^*$ is larger than $\sm$, then we have small relative spectral distance between $ZZ^T$ and $(Z+C)(Z+C)^T$ (with respect to this $\lambda^*$), and this directly bounds the generalization performance of the quantized features in terms of the best possible full-precision performance.
%\end{enumerate}
	
%Here, we show that this implies small relative spectral distance between the Gram matrices of the quantized vs.\ non-quantized embeddings.  This then implies that learning a linear regression model over the quantized embeddings should perform similarly to the model training on the full-precision embeddings (by the generalization bounds in LP-RFF paper), for very low-precision $b$.

%\textbf{Open Issue}: Using Tropp Style bounds to show that this result holds with high-probability for a random quantization of the embeddings (as opposed to showing it holds for the \textit{expectation} of the Gram matrix).

%$$w^* = (X^T X + \lambda I)^{-1}X^T (y+\eps)$$
%
%$$\expect{\eps}{\|y - Xw^*\|^2}$$


\subsection{Notation and Definitions}
We consider the setting of fixed design ridge regression over a fixed data matrix $X \in \RR^{n\times d}$ (in our case, a word embedding matrix). Let $y_i = \by_i + \eps_i \in \RR$ be the observed noisy label for the $i^{th}$ word $z_i$, where the noise satisfies $\expect{}{\eps_i} = 0$ and $\var{}{\eps_i} = \sigma^2$.  Let $K = XX^T$ be the Gram matrix corresponding to this data.  Let $X+C$ be a random quantization of $Z$ to $b$ bits, and let $\tK = (X+C)(X+C)^T$. Assume $\expect{}{C_{ij}} = 0$, and let $\delta_b^2/d \geq \expect{}{C_{ij}^2}$ be an upper bound on the variance of the entry-wise random quantization noise (and thus $0 \preceq \expect{}{CC^T} \preceq \delta_b^2 I_n$).  $\|X\|$ will denote spectral norm of $X$, $\|X\|_F$ will denote Frobenius norm.

We will assume $X_{ij} \in [-\frac{1}{\sqrt{d}},\frac{1}{\sqrt{d}}]$.  If we uniformly quantize this interval using $b$ bits, the size of each sub-interval is $\frac{2}{\sqrt{d}(2^b-1)}$.  Thus, using the fact that a bounded random variable in an interval of length $r$ has variance at most $r^2/4$, we get that $\expect{}{C_{ij}^2} \leq \Big(\frac{2}{\sqrt{d}(2^b-1)}\Big)^2\cdot \frac{1}{4} = \frac{1}{d(2^b-1)^2} = \delta_b^2/d$, for $\delta_b^2 \defeq 1/(2^b-1)^2$.

 
%
%From the LP-RFF work, recall the definition of relative spectral distance:
%$$D_{\lambda}(K,\tK) = \min\bigg\{\Delta \in \RR^+ \;\bigg|\; \frac{1}{1+\Delta}(K+\lambda I) \preceq \tK+\lambda I \preceq (1+\Delta)(K+\lambda I)\bigg\}$$

%It is easy to see that $\expect{}{(Z+C)(Z+C)^T} = K + D$, where $D$ is a diagonal matrix satisfying $0\preceq D\preceq d \cdot \delta_b^2 I$ (we also show this in the LP-RFF work).  

\subsection{Results}
\begin{enumerate}
	\item \textbf{Theorem 1}: Large regularizer doesn't impact generalization much if spectrum decays slowly.
	\item \textbf{Theorem 2}: Large regularizer + low-precision = small ($\Delta_1,\Delta_2$) between full-precision and low-precision Gram matrices.  We know from prior work that this implies good generalization bounds.
%	\item \textbf{Theorem 3}: We directly bound the generalization performance of the models trained on the low-precision embeddings relative to the full-precision embeddings, showing that when the product $2^b(\sigma_d  + \lambda)$ is large, the degradataion in generalization performance is small.  Here, $\sigma_d$ denotes the smallest eigenvalue of the covariance matrix $X^T X$, $\lambda$ is the regularization parameter, and $b$ is the number of bits per entry of the low-precision embeddings.
\end{enumerate}

\subsubsection{Theorem 1}
\begin{theorem}
Let $X$ be a data matrix, and $\by$ be the corresponding vector of labels. Let $\sm$ be the smallest eigenvalue of $X^T X$, and let $\lambda_1, \lambda_2$ be two scalars such that $0 \leq \lambda_1 \leq \lambda_2 \leq a\cdot \sm$, for some $a \in [0,1]$. Letting $\cR_{\lambda}(K)$ denote the expected loss when training with regularizer $\lambda$, Gram matrix $K = XX^T$, and label noise $\sigma^2$, we get that:
\begin{equation}
\frac{R_{\lambda_2}(XX^T) - R_{\lambda_1}(XX^T)}{\|y\|^2/n} \leq a^2
\label{eq1}
\end{equation}
\end{theorem}
\begin{proof}
Let $K = U\Sigma U^T$ be the eigendecomposition of $K = XX^T$. Because $K$ is a rank $d$ matrix, we know the eigenvalues $\sigma_i=0$ for $i > d$.  We know from previous results [Avron '17, Aloui '15] that:
\begin{eqnarray*}
\cR_{\lambda}(K) &=& \frac{1}{n}\lambda^2\by^T(K+\lambda I)^{-2} \by + \frac{1}{n}\sigma^2 \tr\bigg(K^2(K+\lambda I)^{-2}\bigg) \\
&=& \frac{1}{n}\sum_{i=1}^n (U_i^T \by)^2\frac{\lambda^2}{(\sigma_i + \lambda)^2} + \frac{\sigma^2}{n}\sum_{i=1}^n \frac{\sigma_i^2}{(\sigma_i + \lambda)^2} \\
&=& \frac{1}{n}\sum_{i=1}^d (U_i^T \by)^2\frac{\lambda^2}{(\sigma_i + \lambda)^2} +
\frac{1}{n}\sum_{i=d+1}^n (U_i^T \by)^2 +
 \frac{\sigma^2}{n}\sum_{i=1}^d \frac{\sigma_i^2}{(\sigma_i + \lambda)^2} \\
\cR_{\lambda_2}(K) - \cR_{\lambda_1}(K) 
&=& \frac{1}{n}\sum_{i=1}^d (U_i^T \by)^2\bigg(\frac{\lambda_2^2}{(\sigma_i + \lambda_2)^2} - \frac{\lambda_1^2}{(\sigma_i + \lambda_1)^2}\bigg) +
\frac{\sigma^2}{n}\sum_{i=1}^d \bigg(\frac{\sigma_i^2}{(\sigma_i + \lambda_2)^2} - \frac{\sigma_i^2}{(\sigma_i + \lambda_1)^2} \bigg)\\
&\leq& \frac{1}{n}\sum_{i=1}^d (U_i^T \by)^2\frac{\lambda_2^2}{(\sigma_i + \lambda_2)^2}\\
&\leq& \frac{1}{n}\sum_{i=1}^d (U_i^T \by)^2\frac{a^2 \sm^2}{\sm^2}\\
&\leq& \frac{a^2}{n}\|y\|^2
\end{eqnarray*}
\end{proof}
\textbf{Remark}: We can understand the division by $\|y\|^2/n$ as a way to normalize the above difference in generalization performance; in particular, $\|y\|^2/n$ is the test error obtained by the model which always guesses $0$.

Also, note that in the case where $\sigma = 0$, $\lambda_1 = 0$, $\lambda_2 = a \sigma_{min}$, $\sigma_{max} = c\sigma_{min}$, and $\sum_{i=1}^d (U_i^T \by)^2 = \|y\|^2$, we can also lower bound Equation~\eqref{eq1} by $\frac{a^2}{(c+a)^2}$.

\subsubsection{Theorem 2}

\begin{theorem}
	\label{thm2}
	Let $X \in \RR^{n\times d}$ be a data matrix with corresponding (linear) kernel matrix $K = XX^T$; let $X+C$ denote a $b$-bit quantization of $X$, with $\tK = (X+C)(X+C)^T$ the kernel matrix of the quantized data matrix. Here, $C$ denotes the quantization noise, with $\expect{}{C_{ij}} = 0$ and $\var{}{C_{ij}} \leq \delta_b^2/d \;\;\forall i,j$, where $b$ is the number of bits used per feature.
	Then for any $\Delta_1 \geq 0, \Delta_2 \geq \delta^2_b/\lambda$,
	\begin{eqnarray}
	\Prob\Big[(1 - \Delta_1) (K + \lambda I_n) \preceq \tK + \lambda I_n \preceq (1 + \Delta_2) (K + \lambda I_n)
	\Big] 
	\geq \\ 1 - 
	n \exp \bigg(\frac{-\Delta_1^2}{2dL^2 + (2L/3)\Delta_1}\bigg) -
	n \exp \bigg(\frac{-(\Delta_2-\delta_b^2/\lambda)^2}{2dL^2 + (2L/3)(\Delta_2-\delta_b^2/\lambda)}\bigg),
	\end{eqnarray}
	for $L \defeq 5 \cdot \frac{2^b \cdot \delta_b^2}{\lambda}\cdot \frac{n}{d}$.
\end{theorem}


%\noindent\textbf{Remark:} If we approximate $\delta_b^2 \approx 2^{-2b}$, and let $5n/d \eqdef c$, then $L \approx c /(2^b \lambda)$, and the RHS of this bound becomes
%\begin{eqnarray*}
%1 - 
%n \exp \bigg(\frac{-\Delta_1^2 /L}{2dL + (2/3)\Delta_1}\bigg) -
%n \exp \bigg(\frac{-(\Delta_2-\delta_b^2/\lambda)^2 /L}{2dL + (2/3)(\Delta_2-\delta_b^2/\lambda)}\bigg) \\
%= 1 - 
%n \exp \bigg(\frac{-\Delta_1^2 (2^b \lambda)}{2dc^2/(2^b \lambda) + (2c/3)\Delta_1}\bigg) -
%n \exp \bigg(\frac{-(\Delta_2-1/(2^{2b}\lambda))^2 (2^b \lambda)}{2dc^2/(2^b \lambda) + (2c/3)(\Delta_2-1/(2^{2b}\lambda))}\bigg)
%% OLD VERSION
%%= 1 - 2n \exp \Bigg(\frac{-\Big(\Delta-1/(2^{2b}\lambda)\Big)^2 \cdot 2^b \lambda}{2dc^2/(2^b\lambda) + (2c/3)\Big(\Delta-1/(2^{2b}\lambda)\Big)}\Bigg).
%\end{eqnarray*}
%As $2^b \lambda \rightarrow \infty$, and letting $\Delta_1=\Delta_2\eqdef \Delta$ for simplicity, the dominant terms become
%\begin{eqnarray*}
%1 - 2n \exp \Bigg(\frac{-\Delta^2 \cdot 2^b \lambda}{(2c/3)\Delta}\Bigg) = 1 - 2n \exp \Bigg(\frac{-\Delta \cdot 2^b \lambda}{(2c/3)}\Bigg),
%\end{eqnarray*}
%This makes it clear that as $2^b \lambda \rightarrow \infty$, this probability goes to 1, for any $\Delta > 0$.  \textbf{Note that this is NOT the case for the bound in the LP-RFF paper!}

\begin{proof}
We conjugate the desired inequality with $B \defeq (K + \lambda I_n)^{-1/2}$ (\ie, 
multiply by $B$ on the left and right), noting that semidefinite ordering is
preserved by conjugation:
\begin{align*}
&(1 - \Delta_1) (K + \lambda I_n) \preceq \tK + \lambda I_n \preceq (1 + \Delta_2) (K + \lambda I_n) \\
\iff\ &(1 - \Delta_1) I_n \preceq B (\tK + \lambda I_n) B \preceq (1 + \Delta_2) I_n \\
\iff\ &-\Delta_1 I_n \preceq B (\tK + \lambda I_n) B - I_n \preceq \Delta_2 I_n \\
\iff\ &-\Delta_1 I_n \preceq B (\tK + \lambda I_n - K - \lambda I_n) B \preceq \Delta_2 I_n \\
\iff\ &-\Delta_1 I_n \preceq B (\tK - K) B \preceq \Delta_2 I_n.
\end{align*}

Letting $D=\expect{}{CC^T}$, and recalling  $0 \preceq D \preceq \delta_b^2 I_n$, 
by Lemma~\ref{lem1} we have that 
$$-\Delta_1 I_n \preceq B (\tK - K - D) B \preceq
(\Delta_2 - \delta^2_b/\lambda) I_n 
\Longrightarrow -\Delta_1 I_n \preceq B (\tK - K) B \preceq
\Delta_2 I_n,$$
and thus
$$\Prob\Big[ -\Delta_1 I_n \preceq B (\tK - K) B \preceq
\Delta_2 I_n \Big] \geq  \Prob\Big[ -\Delta_1 I_n \preceq B (\tK - K - D) B \preceq
(\Delta_2 - \delta^2_b/\lambda) I_n \Big].$$
	
We thus proceed to bound the RHS of this inequality, using 
Lemma~\ref{lem:quantized_concentration_two_sided}.
This allows us to conclude that for any $\Delta_1 \geq 0$, $\Delta_2 \geq \delta_b^2/\lambda$,
\begin{align*}
&\Prob\bigg[- \Delta_1 I_n \preceq B\Big((X + C)(X + C)^T - (XX^T+D)\Big)B \preceq (\Delta_2 - \delta^2_b/\lambda) I_n\bigg] \\
\geq\ &1 - n \left[ \exp \left( \frac{-\Delta_1^2/2}{dL^2 +
	L\Delta_1/3)} \right) + \exp \left(\frac{-(\Delta_2 - \delta^2_b/\lambda)^2/2}{dL^2 + L(\Delta_2 - \delta^2_b/\lambda)/3)} \right)  \right],
\end{align*}
for $L = 5 \cdot \frac{2^b \cdot \delta_b^2}{\lambda}\cdot  \frac{n}{d}$.

Thus, combining all of the above results, we get that
\begin{eqnarray*}
&&\Prob\Big[(1 - \Delta_1) (K + \lambda I_n) \preceq \tK + \lambda I_n \preceq (1 + \Delta_2) (K + \lambda I_n)
\Big] \\
&=& \Prob\Big[ -\Delta_1 I_n \preceq B (\tK - K) B \preceq
\Delta_2 I_n \Big] \\
&\geq& \Prob\Big[ -\Delta_1 I_n \preceq B (\tK - K - D) B \preceq
(\Delta_2 - \delta^2_b/\lambda) I_n \Big]\\
&\geq& 1 - n \left[ \exp \left( \frac{-\Delta_1^2/2}{dL^2 +
	L\Delta_1/3)} \right) + \exp \left(\frac{-(\Delta_2 - \delta^2_b/\lambda)^2/2}{dL^2 + L(\Delta_2 - \delta^2_b/\lambda)/3)} \right)  \right]
\end{eqnarray*}
\end{proof}


\begin{corollary}
	If $\Delta_1 \geq \frac{\log(n/\rho)L}{3}\Big(1+\sqrt{1+\frac{18d}{\log(n/\rho)}}\Big) \approx \frac{5n}{2^b \lambda}\sqrt{\frac{2\log(n/\rho)}{d}}$,
	then $\Prob\big[(1 - \Delta_1) (K + \lambda I_n) \preceq \tK + \lambda I_n \big] \geq  1 - \rho$. 
	Similarly, if $\Delta_2 \geq \frac{\delta_b^2}{\lambda} +  \frac{\log(n/\rho)L}{3}\Big(1+\sqrt{1+\frac{18d}{\log(n/\rho)}}\Big) \approx \frac{1}{2^{2b}\lambda} + \frac{5n}{2^b \lambda}\sqrt{\frac{2\log(n/\rho)}{d}}$,
	then $\Prob\big[\tK + \lambda I_n \preceq (1 + \Delta_2) (K + \lambda I_n)\big] \geq  1 - \rho$. 
\end{corollary}

\begin{proof}
Letting $\Delta_2 \rightarrow \infty$ in Theorem~\ref{thm2}, we see that 
$\Prob\big[(1 - \Delta_1) (K + \lambda I_n) \preceq \tK + \lambda I_n \big] \geq 1 - n \exp \bigg(\frac{-\Delta_1^2}{2dL^2 + (2L/3)\Delta_1}\bigg)$.
Lower-bounding the RHS by $1-\rho$, and solving for $\Delta_1$, we get the following set of equivalent statements (assuming $\Delta_1 \geq 0$, and $a\defeq \log(n/\rho)$):

\begin{eqnarray*}
1 - n \exp \bigg(\frac{-\Delta_1^2}{2dL^2 + (2L/3)\Delta_1}\bigg) &\geq& 1-\rho \\
\Longleftrightarrow \rho &\geq& n \exp \bigg(\frac{-\Delta_1^2}{2dL^2 + (2L/3)\Delta_1}\bigg) \\
\Longleftrightarrow \log(n/\rho) &\leq& \frac{\Delta_1^2}{2dL^2 + (2L/3)\Delta_1} \\
\Longleftrightarrow 0 &\leq&  \Delta_1^2 - (2aL/3)\Delta_1 - 2adL^2 \\
\Longleftrightarrow \Delta_1 &\geq&  \frac{1}{2}\bigg(\frac{2aL}{3} + \sqrt{\Big(\frac{2aL}{3}\Big)^2 + 8adL^2}\bigg) \\
\Longleftrightarrow \Delta_1 &\geq& \frac{aL}{3}\bigg(1 + \sqrt{1 + \frac{18d}{a}}\bigg)
\end{eqnarray*}
Now, using $L = 5 \cdot \frac{2^b \cdot \delta_b^2}{\lambda}\cdot  \frac{n}{d} \approx \frac{5n/d}{2^b \lambda}$ (because $\delta_b^2 = (2^b-1)^{-2} \approx 2^{-2b}$), $1 + \sqrt{1 + \frac{18d}{a}} \approx \sqrt{\frac{18d}{a}}$ (assuming $18d \gg a$), we get
\begin{eqnarray*}
	\Delta_1 &\gtrsim& \frac{\log(n/\rho)}{3}\cdot \frac{5n/d}{2^b \lambda}\cdot \sqrt{\frac{18d}{\log(n/\rho)}} \\
	&=& \frac{5n}{2^b \lambda}\sqrt{\frac{2\log(n/\rho)}{d}}.
\end{eqnarray*}
This completes the first part of the proof.
To prove the second part, we repeat the exact same steps with $\widehat{\Delta_2} = \Delta_2 - \delta_b^2/\lambda$, this time letting $\Delta_1 \rightarrow \infty$ in Theorem~\ref{thm2}.

\end{proof}

\subsection{Lemmas}
\begin{lemma}
	\label{lem1}
	Letting $K=XX^T$, $\tK = (X+C)(X+C)^T$, $B = (K+\lambda I_n)^{-1/2}$, $D = \expect{}{CC^T}$, with $0 \preceq D \preceq \delta_b^2 I_n$, it follows that:
	\begin{align*}
-\Delta_1 I_n &\preceq B (\tK - K - D) B \preceq
(\Delta_2 - \delta^2_b/\lambda) I_n \\
\Longrightarrow -\Delta_1 I_n &\preceq B (\tK - K) B \preceq
\Delta_2 I_n
	\end{align*}
\end{lemma}
\begin{proof}
Simply add $BDB$ to all sides, and notice that $0 \preceq BDB \preceq (\delta_b^2/\lambda) I_n$ (follows because $\|BDB\| \leq \|B^2\| \|D\|$, $0 \preceq D \preceq \delta_b^2 I_n$ and $B^2 = (K+\lambda I_n)^{-1}$).
\begin{eqnarray*}
-\Delta_1 I_n + BDB &\preceq& B (\tK - K - D) B  + BDB \preceq
(\Delta_2 - \delta^2_b/\lambda) I_n  + BDB \\
\Longrightarrow -\Delta_1 I_n \preceq  -\Delta_1 I_n + BDB &\preceq& B (\tK - K) B \preceq
\Delta_2 I_n - (\delta_b^2/\lambda) I_n  + BDB \preceq \Delta_2 I_n
\end{eqnarray*}
\end{proof}


\begin{lemma}
	Let $K=XX^T$ be a (linear) kernel matrix for $X \in\RR^{n \times d}$, and $\tK = (X+C)(X+C)^T$ be a $b$-bit quantization of $K$ with expectation $K+D$.
	Letting $B\defeq (K+\lambda I_n)^{-1/2}$ and $L \defeq 5 \cdot \frac{2^b \cdot \delta_b^2}{\lambda}\cdot  \frac{n}{d}$, then for any $t_1, t_2 \geq 0$,
	\begin{align*}
	&\Prob\bigg[- t_1 I_n \preceq B\Big((X + C)(X + C)^T - (XX^T+D)\Big)B \preceq t_2 I_n\bigg] \\
	\geq\ &1 - n \left[ \exp \left( \frac{-t_1^2/2}{dL^2 +
		Lt_1/3)} \right) + \exp \left(\frac{-t_2^2/2}{dL^2 + Lt_2/3)} \right)  \right].
	\end{align*}
	\label{lem:quantized_concentration_two_sided}
\end{lemma}

\begin{proof}
	Using the notation from Theorem~\ref{thm:intdim-bernstein-herm},
	we consider $S_k = B\Big((x_k + c_k)(x_k + c_k)^T  - x_k x_k^T - D_k\Big)B$, for $B\defeq (K+\lambda I_n)^{-1/2}$, $D_k = \expect{}{c_k c_k^T}$, and $x_k$ and $c_k$ denoting the $k^{th}$ columns of $X$ and $C$ respectively.
	It is easy to see that $Z\defeq \sum_{k=1}^d S_k = B((X+C)(X+C)^T - XX^T - D)B = B(\tK - K - D)B$, and that $\expect{}{S_k} = 0$.
	Thus, to apply Theorem~\ref{thm:intdim-bernstein-herm}, we simply need to find upper bounds $L,v(Z)$ such that $\lambda_{max}(S_k) \leq L$ and  $\|\sum_k \expect{}{S_k^2}\| \leq v(Z)$.
	By Lemma~\ref{upper_bounds}, $L \defeq 5 \cdot \frac{2^b \cdot \delta_b^2}{\lambda}\cdot  \frac{n}{d}$ and $v(Z) = dL^2$.
	
	Applying Theorem~\ref{thm:intdim-bernstein-herm} with $Z$, for any $t_2 \geq 0$,
	we have
	\begin{equation*}
	\Prob\bigg[\lambda_{\max}(B\Big((X + C)(X + C)^T - (XX^T+D)\Big)B) \succeq t_2 I_n \bigg] \leq n
	\exp \left( \frac{-t_2^2/2}{dL^2 + Lt_2/3)} \right).
	\end{equation*}
	Similarly, applying Theorem~\ref{thm:intdim-bernstein-herm} with $-Z$ and
	using the fact that $\lambda_{\max}(-Z) = -\lambda_{\min}(Z)$, for any $t_1 \geq 0$,
	we have
	\begin{equation*}
	\Prob\bigg[\lambda_{\min}(B\Big((X + C)(X + C)^T - (XX^T+D)\Big)B) \preceq -t_1 I_n \bigg] \leq n
	\exp \left( \frac{-t_1^2/2}{dL^2 + Lt_1/3)} \right).
	\end{equation*}
	Combining the two bounds with the union bound yields the desired inequality.
\end{proof}


\begin{lemma}
	\label{upper_bounds}
	Let $S_k = B\Big((x_k + c_k)(x_k + c_k)^T  - x_k x_k^T - D_k\Big)B$, with $B=(K+\lambda I_n)^{-1/2}$ and $D_k=\expect{}{c_k c_k^T}$.  It follows that
	$$\|S_k\| \leq 5 \cdot \frac{2^b \cdot \delta_b^2}{\lambda}\cdot  \frac{n}{d} \eqdef L,$$
	and
	$$\Big\|\sum_{k=1}^d \expect{}{S_k^2}\Big\| \leq dL^2.$$
\end{lemma}
\begin{proof}
\begin{eqnarray*}
\|S_k\| &=& \|B\Big((x_k + c_k)(x_k + c_k)^T  - x_k x_k^T - D_k\Big)B\| \\
&=& \|B\Big(x_kc_k^T + c_kx_k^T + c_k c_k^T - D_k\Big)B\| \\
&\leq& \|B^2\|\Big(\|x_kc_k^T\| + \|c_kx_k^T\| + \|c_k c_k^T\| + \|D_k\|\Big) \\
&=& \|B^2\|\Big(2\|x_k\|\|c_k\| + \|c_k\|^2 + \|D_k\|\Big) 
\end{eqnarray*}
To bound $\|x_k\|$ and $\|c_k\|$ we use the facts that $x_k$ is a vector of length $n$, with each entry bounded in magnitude by $1/\sqrt{d}$, and that $c_k$ is a vector of length $n$ with each entry bounded by $\frac{2}{\sqrt{d}(2^b-1)}$.  Thus, $\|x_k\| \leq \sqrt{n/d}$, and $\|c_k\| \leq \sqrt{4n/(d(2^b-1)^2)} = \sqrt{4\delta_b^2n/d} = 2\delta_b\sqrt{n/d}$.  We will also use the fact that $\|B^2\| = \|(K+\lambda I_n)^{-1}\| \leq 1/\lambda$.  Lastly, we note that $0 \preceq D_k = \expect{}{c_k c_k^T} = diag(\expect{}{c_{k1}^2},\ldots,\expect{}{c_{kn}^2}) \preceq (\delta_b^2/d) I_n$, and thus $\|D_k\| \leq \delta_b^2/d$.

We now continue the above chain of inequalities:
\begin{eqnarray*}
\|S_k\| &\leq& (1/\lambda)\Big(2\sqrt{n/d}\cdot 2\delta_b\sqrt{n/d} + 4\delta_b^2(n/d) + \delta_b^2/d\Big)\\
&=& (1/\lambda)\Big(4\delta_b(n/d) + 4\delta_b^2(n/d) + \delta_b^2/d\Big) \\
&=& (1/\lambda)\Big(4\delta_b^2(2^b-1)(n/d) + (4n + 1)(\delta_b^2/d)\Big) \\
&\leq& (1/\lambda)\Big(5\delta_b^2(2^b-1)(n/d) + 5n(\delta_b^2/d)\Big) \\
&=& (1/\lambda)\Big(5n(\delta_b^2/d)\Big((2^b-1)+1\Big)\Big) \\
&=& 5 \cdot \frac{2^b \cdot \delta_b^2}{\lambda}\cdot  \frac{n}{d} \eqdef L
\end{eqnarray*}
This implies that $\|S_k^2\| \leq L^2$.  It is now easy to see that 
$\Big\|\sum_{k=1}^d \expect{}{S_k^2}\Big\| \leq \sum_{k=1}^d \big\|\expect{}{S_k^2}\big\| \leq dL^2 \eqdef v(Z)$, completing the proof of the lemma.


%Now it's time to bound $\E[S_i^2]$.  We will use
%\begin{eqnarray*}
%	\E[S_i^2] &=& \frac{1}{m^2}\Big(\E\big[(u_i u_i^T)^2\big] - \E\big[u_iu_i^T\big]^2\Big) \preceq \frac{1}{m^2} \E[(u_i u_i^T)^2]
%	= \frac{1}{m^2} \E[u_i u_i^T u_i u_i^T] \\
%	&=& \frac{1}{m^2} \E[\norm{u_i}^2 u_i
%	u_i^T] \preceq \frac{2n \norm{B}^2}{m^2} \E[u_i u_i^T].
%\end{eqnarray*}
%Thus
%\begin{equation*}
%\sum_{i=1}^{m} \E[S_i^2] \preceq \frac{2n \norm{B}^2}{m} \E[v_1 v_1^T] = \frac{2n
%	\norm{B}^2}{m} B(K + D) B^T \preceq \frac{2n \norm{B}^2}{m} B(K + \delta^2_b I_n)B^T
%= LM/m.
%\end{equation*}



\end{proof}


\begin{theorem}[Matrix Bernstein: Hermitian Case (Theorem 6.6.1 Tropp)] \label{thm:intdim-bernstein-herm}
	Consider a finite sequence $\{ S_k \}$ of random Hermitian matrices in $\RR^{n\times n}$, and assume that
	\begin{equation*}
	\E [S_k] = 0
	\quad\text{and}\quad
	\lambda_{\max}(S_k) \leq L
	\quad\text{for each index $k$.}
	\end{equation*}
	Introduce the random matrix
	\begin{equation*}
	Z = \sum\nolimits_k S_k.
	\end{equation*}
	Let $v(Z)$ be the matrix variance statistic of the sum:
	\begin{equation*}
	v(Z) = \|\E [Z^2]\| = \|\sum_k \E [S_k^2]\|.
	\end{equation*}
	Then, for $t \geq 0$,
	\begin{equation} \label{eqn:intdim-bernstein-tail}
	\Prob \left( \lambda_{\max}(Z) \geq t \right)
	\leq n \cdot \exp\left( \frac{-t^2/2}{v(Y) + Lt/3} \right).
	\end{equation}
	\label{thm:bernstein}
\end{theorem}
	
%\begin{lemma}
%	\label{lem2}
%	Letting $R = (X^T X + \lambda I_d)^{-1/2}$, $D = \expect{}{C^T C}$, with $0 \preceq D \preceq \frac{n/d}{(2^b-1)^2} I_d$, it follows that:
%	\begin{align*}
%	-\bigg(t - \frac{n/d}{(2^b-1)^2(\sigma_d + \lambda)}\bigg) I_d &\preceq R (\tX^T \tX - X^T X - D) R \preceq
%	\bigg(t - \frac{n/d}{(2^b-1)^2(\sigma_d + \lambda)}\bigg) I_d \\
%	\Longrightarrow -t I_d &\preceq R (\tX^T \tX - X^T X) R \preceq t I_d
%	\end{align*}
%\end{lemma}
%\begin{proof}
%	Simply add $RDR$ to all sides, and notice that $\|RDR\| \leq \|R^2\|\|D\| \leq  \frac{(n/d)\delta_b^2}{\sigma_d+\lambda}$. Thus
%	\begin{align*}
%-\bigg(t - \frac{n/d}{(2^b-1)^2(\sigma_d + \lambda)}\bigg) I_d + RDR &\preceq R (\tX^T \tX - X^T X - D) R + RDR \preceq
%\bigg(t - \frac{n/d}{(2^b-1)^2(\sigma_d + \lambda)}\bigg) I_d +RDR \\
%\Longrightarrow -t I_d \preceq -t I_d + \frac{n/d}{(2^b-1)^2(\sigma_d + \lambda)}I_d + RDR &\preceq R (\tX^T \tX - X^T X) R \preceq
%t I_d - \frac{n/d}{(2^b-1)^2(\sigma_d + \lambda)} I_d + RDR \preceq t I_d
%\end{align*}
%\end{proof}
%\begin{lemma}
%	\label{upper_bounds2}
%	Let $S_k = R\Big((x_k + c_k)(x_k + c_k)^T  - x_k x_k^T - D_k\Big)R$, with $R=(X^T X+\lambda I_d)^{-1/2}$ and $D_k = \expect{}{c_kc_k^T}$. Importantly, unlike in Lemma~\ref{upper_bounds}, here $x_k$ and $c_k$ will be vectors in $\RR^d$ representing the $k^{th}$ word embedding and its corresponding quantization noise (\textbf{apologies for overloading notation!}).  
%	It follows that
%		$$\|S_k\| \leq \frac{9}{(2^b-1)(\sigma_d + \lambda)} \eqdef L,$$
%		and
%		$$\Big\|\sum_{k=1}^n \expect{}{S_k^2}\Big\| \leq nL^2.$$
%	\end{lemma}
%	\begin{proof}
%Exactly like in Lemma~\ref{upper_bounds} we can show that
%		\begin{eqnarray*}
%			\|S_k\| &\leq& \|R^2\|\Big(2\|x_k\|\|c_k\| + \|c_k\|^2 + \|D_k\|\Big) 
%		\end{eqnarray*}
%		To bound $\|x_k\|$ and $\|c_k\|$ we use the facts that $x_k$ is a vector of length $d$, with each entry bounded in magnitude by $1/\sqrt{d}$, and that $c_k$ is a vector of length $d$ with each entry bounded by $\frac{2}{\sqrt{d}(2^b-1)}$.  Thus, $\|x_k\| \leq 1$, and $\|c_k\| \leq 2/(2^b-1)$.  We will also use the fact that $\|R^2\| = \|(X^T X +\lambda I_d)^{-1}\| \leq \frac{1}{\sigma_d + \lambda}$. Lastly, we note that $0 \preceq D_k = \expect{}{c_k c_k^T} = diag(\expect{}{c_{k1}^2},\ldots,\expect{}{c_{kd}^2}) \preceq (\delta_b^2/d) I_d$, and thus $\|D_k\| \leq \delta_b^2/d = \frac{1}{d(2^b-1)^2}$.
%		
%		We now continue the above chain of inequalities:
%		\begin{eqnarray*}
%			\|S_k\| &\leq&\frac{1}{\sigma_d + \lambda}\Big(\frac{4}{2^b-1} + \frac{4}{(2^b-1)^2} + \frac{1}{d(2^b-1)^2}\Big)\\
%			&\leq&\frac{1}{\sigma_d + \lambda}\Big(\frac{8 + 1/d}{2^b-1}\Big)\\
%			&\leq&\frac{9}{(\sigma_d + \lambda)(2^b-1)} \eqdef L\\
%		\end{eqnarray*}
%		This implies that $\|S_k^2\| \leq L^2$.  It is now easy to see that 
%		$\Big\|\sum_{k=1}^n \expect{}{S_k^2}\Big\| \leq \sum_{k=1}^n \big\|\expect{}{S_k^2}\big\| \leq nL^2$, completing the proof of the lemma.
%	\end{proof}

%\subsection{Theorem 3}
%\begin{theorem}
%	Let $X\in \RR^{n\times d}$ be a data matrix with $|X_{ij}| \leq \frac{1}{\sqrt{d}}$, and $\tX\defeq X+C$ denote a random $b$-bit quantization of $X$ with $\expect{}{C_{ij}}=0$, $\expect{}{C_{ij}^2} \leq \frac{1}{(2^b-1)^2}$, and $|C_{ij}| \leq \frac{2}{\sqrt{d}(2^b-1)} \; \forall i,j$.  
%	Let $\sigma_d$ be the smallest eigenvalue of $X^T X$, and $\tsigma_d$ be the smallest eigenvalue of $\tX^T \tX$.
%	Let $y \in \RR^n$ be the corresponding vector of ``clean'' labels, where $y_i + \eps_i$ are the noisy observed labels; assume $\expect{}{\eps_i} = 0$, $\expect{}{\eps_i^2} = \sigma^2$, and $|\eps_i| \leq a \; \forall i$.  Let $\cR_{\lambda}(X)$ denote the expected generalization error when training with regularizer $\lambda$, data matrix $X$, and label noise $\sigma^2$.  Let $f(t) = \bigg(\frac{8n}{(2^b-1)(\tsigma_d + \lambda)} +\frac{t}{1-t}\cdot \frac{n}{\sigma_d+\lambda}\bigg)\cdot \big(\|y\| + a\sqrt{n}\big)$ and let $L = \frac{9}{(2^b-1)(\sigma_d + \lambda)}$.  It follows that for any $t\geq \frac{n/d}{(2^b-1)^2(\sigma_d + \lambda)}$
%	
%	$$R_{\lambda}(\tX) - R_{\lambda}(X) \leq f(t)\Big(2\sqrt{R_{\lambda}(X)}+f(t)\Big)$$
%	with probability at least 
%	
%	$$1 - 2d \exp \left(\frac{-\Big(t-\frac{n/d}{(2^b-1)^2(\sigma_d + \lambda)}\Big)^2}{2nL^2 + (2L/3)\Big(t-\frac{n/d}{(2^b-1)^2(\sigma_d + \lambda)}\Big)} \right)$$
%\end{theorem}
%\noindent\textbf{Remark:} If we approximate $(2^b-1)^2 \approx 2^{2b}$, $2^b-1\approx 2^b$, $L \approx \frac{9}{2^b(\sigma_d + \lambda)}$, and the RHS of this bound becomes
%\begin{eqnarray*}
%	\approx 1 - 2d \exp \left(\frac{-\Big(t-\frac{n/d}{2^{2b}(\sigma_d + \lambda)}\Big)^2\frac{1}{L}}{2nL + (2/3)\Big(t-\frac{n/d}{2^{2b}(\sigma_d + \lambda)}\Big)} \right) &=& 1 - 2d \exp \left(\frac{-\Big(t-\frac{n/d}{2^{2b}(\sigma_d + \lambda)}\Big)^2\frac{2^b(\sigma_d + \lambda)}{9}}{\frac{18n}{2^b(\sigma_d + \lambda)} + (2/3)\Big(t-\frac{n/d}{2^{2b}(\sigma_d + \lambda)}\Big)} \right)
%\end{eqnarray*}
%As $2^b (\sigma_d +\lambda) \rightarrow \infty$, the dominant terms become
%\begin{eqnarray*}
%	1 - 2d \exp \left(\frac{-t^2 \cdot \frac{2^b(\sigma_d + \lambda)}{9}}{(2t/3)} \right) = 1 - 2d \exp \left(-  \frac{2^b(\sigma_d + \lambda)}{6}\cdot t \right).
%\end{eqnarray*}
%This makes it clear that as $2^b (\sigma_d + \lambda) \rightarrow \infty$, this probability goes to 1, for any $t > 0$.
%
%\avner{I need to update $f(t)$ to not depend on $\tsigma_d$!  Of course we can simply upper bound $f(t)$ by setting $\tsigma_d = 0$, but it would be nicer if we were able to replace $\tsigma_d + \lambda$ with $\sigma_d + \lambda$ instead.}
%
%\begin{proof}
%	The generalization error in fixed design kernel ridge regression, with a linear kernel function, is $\expect{\eps}{\|y-X(X^T X + \lambda I)^{-1} X^T (y+\eps)\|^2}$.  We will denote this by $R_{\lambda}(X)$.  We would like to bound the difference between $R_{\lambda}(X)$ and $R_{\lambda}(\tX)$, where $\tX = X+C$ is a quantized version of $X$.  Let $r_{\eps,\lambda}(X) = \|y-X(X^T X + \lambda I)^{-1} X^T (y+\eps)\|$; note that we will simply write $r(X)$ for simplicity, suppressing the subscripts. For brevity we will use $A \defeq X^T X + \lambda I$, and $\tA \defeq \tX^T \tX + \lambda I$.  Our approach will be to bound $R_{\lambda}(\tX) - R_{\lambda}(X)$ in terms of $r(X) - r(\tX)$, as follows:
%	\begin{eqnarray*}
%		R_{\lambda}(\tX) - R_{\lambda}(X) &=& \expect{\eps}{\|y-\tX\tA^{-1} \tX^T (y+\eps)\|^2-\|y-XA^{-1} X^T (y+\eps)\|^2 } \\
%		&=& \expect{\eps}{r(\tX)^2 - r(X)^2} \\
%		&=& \expect{\eps}{\Big(r(\tX) - r(X)\Big)\Big(r(\tX)+r(X)\Big)}
%	\end{eqnarray*}
%	Thus, if we upper bound the magnitude of $r(\tX) - r(X)$ and $r(\tX) + r(X)$, we will have an upper bound on $R_{\lambda}(\tX) - R_{\lambda}(X)$. The proof will follow the following steps:
%	\begin{itemize}
%		\item \textbf{Step 1}: We begin by bounding $r(\tX) - r(X)$ under the assumption that $(1-t)A \preceq \tA \preceq (1+t)A$.
%		\item \textbf{Step 2}: Given such a bound $T \geq r(\tX) - r(X)$, we can bound $r(\tX) + r(X) = 2r(x) + (r(\tX) - r(X)) \leq 2r(X) + T$.  This gives an upper bound
%		\begin{eqnarray*}
%			R_{\lambda}(\tX) - R_{\lambda}(X) &\leq& \expect{\eps}{T\Big(2r(X)+T\Big)}\\
%			&=& T\Big(2\expect{\eps}{r(X)}+T\Big) \quad \text{(Assuming $T$ doesn't depend on $\eps$.)}\\
%			&\leq& T\Big(2\sqrt{R_{\lambda}(X)}+T\Big) \quad \text{(By Jensen's ineqality.)}
%		\end{eqnarray*}
%		\item \textbf{Step 3}: We lower bound the probability of the event  $(1-t)A \preceq \tA \preceq (1+t)A$ occurring.
%	\end{itemize}
%	
%	\noindent We now begin the work of bounding $r(\tX) - r(X)$.
%	\begin{eqnarray*}
%		r(\tX) - r(X) &=& \|y-\tX\tA^{-1} \tX^T (y+\eps)\| - \|y-X A^{-1} X^T (y+\eps)\| \\
%		&\leq& \Big\|\Big(y-\tX\tA^{-1} \tX^T (y+\eps)\Big) - \Big(y - X A^{-1} X^T (y+\eps)\Big)\Big\| \\
%		&=& \|\tX\tA^{-1} \tX^T (y+\eps) - X A^{-1} X^T(y+\eps)\| \\
%		&\leq& \big\|\tX\tA^{-1} \tX^T - X A^{-1} X^T\big\|\cdot \|y + \eps\| \\
%		&=& \big\|(X+C)\tA^{-1} (X+C)^T - X A^{-1} X^T\big\|\cdot \|y + \eps\| \\
%		&=& \big\|C \tA^{-1} X^T + X \tA^{-1} C^T + C \tA^{-1} C^T + X \tA^{-1} X^T - X A^{-1} X^T\big\|\cdot \|y + \eps\| \\
%		&\leq& \bigg(\|C\| \|\tA^{-1}\|\Big(2\|X\| + \|C\|\Big) + \big\|X \tA^{-1} X^T - X A^{-1} X^T\big\|\bigg)\cdot \big(\|y\| + \|\eps\|\big) 
%	\end{eqnarray*}
%	We now proceed to bound the individual elements in this expression. We use the facts that $X \in R^{n \times d}$ with $|X_{ij}| \leq 1/\sqrt{d}$,
%	$C \in R^{n \times d}$ with $|C_{ij}|\leq \frac{2}{\sqrt{d}(2^b-1)}$, and $\eps \in \RR^n$ with $|\eps_i| \leq a$.  We let $\sigma_i$ and $\tsigma_i$ denote the $i^{th}$ largest eigenvalues of $X^T X$ and $\tX^T \tX$ respectively.
%	\begin{eqnarray*}
%		\|C\| &\leq& \|C\|_F \leq \sqrt{nd \cdot \frac{4}{d(2^b-1)^2}} = \frac{2\sqrt{n}}{2^b-1} \\
%		\|X\| &\leq& \|X\|_F \leq \sqrt{n}\\
%		\|\tA^{-1}\| &=& \|(\tX^T\tX + \lambda I)^{-1}\|  = \frac{1}{\tsigma_d + \lambda} \\
%		\|\eps\| &\leq& a\sqrt{n}
%	\end{eqnarray*}
%	The trickiest term to bound is $\big\|X \tA^{-1} X^T - X A^{-1} X^T\big\|$.  To do this, we assume $(1-t)A \preceq \tA \preceq (1+t)A$.
%	\begin{eqnarray*}
%		(1-t)A &\preceq& \tA \preceq (1+t)A\\
%		\Longleftrightarrow \frac{1}{1+t}A^{-1} &\preceq& \tA^{-1} \preceq \frac{1}{1-t}A^{-1} \\
%		\Longrightarrow \frac{1}{1+t} X A^{-1}X^T &\preceq& X \tA^{-1}X^T  \preceq \frac{1}{1-t}X A^{-1}X^T \\
%		\Longleftrightarrow \Big(\frac{1}{1+t} - 1\Big)X A^{-1}X^T &\preceq& X \tA^{-1}X^T - XA^{-1}X^T \preceq \Big(\frac{1}{1-t}-1\Big)X A^{-1}X^T \\
%		\Longleftrightarrow \frac{-t}{1+t}X A^{-1}X^T &\preceq& X \tA^{-1}X^T - XA^{-1}X^T \preceq \frac{t}{1-t}X A^{-1}X^T \\
%		\Longrightarrow \frac{-t}{1+t}\|XA^{-1}X^T\| I_n &\preceq&  X \tA^{-1}X^T - XA^{-1}X^T \preceq \frac{t}{1-t}\|X A^{-1}X^T\|I_n \\
%		\Longrightarrow  \|X \tA^{-1}X^T - XA^{-1}X^T\| &\leq&  \max\Big(\frac{t}{1+t},\frac{t}{1-t}\Big)\|X A^{-1}X^T\|\\
%		&\leq&  \frac{t}{1-t}\|X\|^2 \|A^{-1}\|\\
%		\Longrightarrow  \|X \tA^{-1}X^T - XA^{-1}X^T\| &\leq&  \frac{t}{1-t}\cdot \frac{n}{\sigma_d+\lambda}
%	\end{eqnarray*}
%	
%	Combining all the above, we see that if $(1-t)A \preceq \tA \preceq (1+t)A$, then
%	\begin{eqnarray*}
%		r(\tX) - r(X) &\leq& \bigg(\|C\| \|\tA^{-1}\|\Big(2\|X\| + \|C\|\Big) + \big\|X \tA^{-1} X^T - X A^{-1} X^T\big\|\bigg)\cdot \big(\|y\| + \|\eps\|\big) \\
%		&\leq& \bigg(\frac{2\sqrt{n}}{2^b-1}\cdot \frac{1}{\tsigma_d + \lambda}\Big(2\sqrt{n} +\frac{2\sqrt{n}}{2^b-1}\Big) +\frac{t}{1-t}\cdot \frac{n}{\sigma_d+\lambda}\bigg)\cdot \big(\|y\| + a\sqrt{n}\big) \\
%		&\leq& \bigg(\frac{2\sqrt{n}}{2^b-1}\cdot \frac{1}{\tsigma_d + \lambda}\cdot 4\sqrt{n} +\frac{t}{1-t}\cdot \frac{n}{\sigma_d+\lambda}\bigg)\cdot \big(\|y\| + a\sqrt{n}\big) \\
%		&=& \bigg(\frac{8n}{(2^b-1)(\tsigma_d + \lambda)} +\frac{t}{1-t}\cdot \frac{n}{\sigma_d+\lambda}\bigg)\cdot \big(\|y\| + a\sqrt{n}\big) 
%	\end{eqnarray*}
%	
%	This (basically) completes steps 1 and 2, though there is one final nuance:
%	\begin{itemize}
%		\item The upper bound includes the term $\tsigma_d$, which is random as it is the smallest eigenvalue of $\tX^T\tX$. There are two options for dealing with this: (1) simple use the fact that $\tsigma_d \geq 0$. (2) Use matrix concentration results to upper bound the probability of $\tsigma_d$ being below some value \textbf{TODO: DEAL WITH THIS}.
%	\end{itemize}
%	
%	Now, we proceed to lower bound the probability that $(1-t)A \preceq \tA \preceq (1+t)A$.  This proof will be similar to the proof of Theorem~\ref{thm2}. We will once again use the matrix Bernstein inequality. In a manner analogous to the proof of Theorem ~\ref{thm2}, we conjugate the desired inequality with $R \defeq (X^T X + \lambda I_d)^{-1/2} = A^{-1/2}$ noting that semidefinite ordering is preserved by conjugation:
%	\begin{align*}
%	&(1 - t) A \preceq \tA \preceq (1 + t) A \\
%	\iff\ &(1 - t) I_d \preceq R \tA R \preceq (1 + t) I_d \\
%	\iff\ &-t I_d \preceq R \tA R - I_d \preceq t I_d \\
%	\iff\ &-t I_d \preceq R (\tX^T \tX + \lambda I_d - X^T X - \lambda I_d) R \preceq t I_d \\
%	\iff\ &-t I_d \preceq R (\tX^T \tX - X^T X) R \preceq t I_d.
%	\end{align*}
%	
%	Letting $D=\expect{}{C^TC}$, by Lemma~\ref{lem2} we have that 
%	$$-\Big(t - \frac{n/d}{(2^b-1)^2(\sigma_d + \lambda)}\Big) I_d \preceq R (\tX^T \tX - X^T X - D) R \preceq
%	\Big(t - \frac{n/d}{(2^b-1)^2(\sigma_d + \lambda)}\Big) I_d 
%	\Longrightarrow -t I_d \preceq R (\tX^T \tX - X^T X) R \preceq
%	t I_d$$
%	and thus
%	$$\Prob\Big[ \|R (\tX^T \tX - X^T X) R\| \geq t\Big] \leq  \Prob\bigg[ \|R (\tX^T \tX - X^T X -D) R\| \geq t-\frac{n/d}{(2^b-1)^2(\sigma_d + \lambda)}\bigg]  $$
%	
%	We thus proceed to bound the RHS of this inequality, using the matrix Bernstein inequality (Theorem 6.1.1 in Tropp).  We let $S_k = R\Big((x_k + c_k)(x_k + c_k)^T  - x_k x_k^T - D_k\Big)R$, with $R=(X^T X+\lambda I_d)^{-1/2}$ and $D_k = \expect{}{c_kc_k^T}$. Importantly, unlike in Lemma~\ref{upper_bounds}, here $x_k$ and $c_k$ will be vectors in $\RR^d$ representing the $k^{th}$ word embedding and its corresponding quantization noise (\textbf{apologies for overloading notation!}).  As we show in Lemma~\ref{upper_bounds2}
%	It follows that
%	$$\|S_k\| \leq \frac{9}{(2^b-1)(\sigma_d + \lambda)} \eqdef L,$$
%	and
%	$$\Big\|\sum_{k=1}^n \expect{}{S_k^2}\Big\| \leq nL^2.$$
%	
%	We note that $Z\defeq \sum_{k=1}^n S_k = R\Big(\tX^T \tX - X^T X - D\Big)R$.  We now use matrix Bernstein to conclude that $\forall t \geq \frac{n/d}{(2^b-1)^2(\sigma_d + \lambda)}$, 
%	\begin{eqnarray*}
%		\Prob\Big[ (1 - t) A \preceq \tA \preceq (1 + t) A\Big] &=& 
%		1- \Prob\Big[\big\|R (\tX^T \tX - X^T X) R\big\| \geq t\Big]\\ 
%		&\geq& 1- \Prob\bigg[ \big\|R (\tX^T \tX - X^T X -D) R\big\| \geq t-\frac{n/d}{(2^b-1)^2(\sigma_d + \lambda)}\bigg] \\
%		&\geq& 1- 2d \exp \left(\frac{-\Big(t-\frac{n/d}{(2^b-1)^2(\sigma_d + \lambda)}\Big)^2}{2nL^2 + (2L/3)\Big(t-\frac{n/d}{(2^b-1)^2(\sigma_d + \lambda)}\Big)} \right)
%	\end{eqnarray*}
%\end{proof}


%%%%%%%%%%%%%%%%%%%%%%%%%%%%%%%%%%%%%%%%%%%%%%%%%%%%%%%%%%%%%%%%%%%%%%%%%%%%%%%
%%%%%%%%%%%%%%%%%%%%%%%%%%%%%%%%%%%%%%%%%%%%%%%%%%%%%%%%%%%%%%%%%%%%%%%%%%%%%%%



\end{document}


% This document was modified from the file originally made available by
% Pat Langley and Andrea Danyluk for ICML-2K. This version was created
% by Iain Murray in 2018. It was modified from a version from Dan Roy in
% 2017, which was based on a version from Lise Getoor and Tobias
% Scheffer, which was slightly modified from the 2010 version by
% Thorsten Joachims & Johannes Fuernkranz, slightly modified from the
% 2009 version by Kiri Wagstaff and Sam Roweis's 2008 version, which is
% slightly modified from Prasad Tadepalli's 2007 version which is a
% lightly changed version of the previous year's version by Andrew
% Moore, which was in turn edited from those of Kristian Kersting and
% Codrina Lauth. Alex Smola contributed to the algorithmic style files.
