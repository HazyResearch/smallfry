%%%%%%%% ICML 2019 EXAMPLE LATEX SUBMISSION FILE %%%%%%%%%%%%%%%%%

\documentclass{article}

% Recommended, but optional, packages for figures and better typesetting:
\usepackage{microtype}
\usepackage{graphicx}
\usepackage{subfigure}
\usepackage{booktabs} % for professional tables

% hyperref makes hyperlinks in the resulting PDF.
% If your build breaks (sometimes temporarily if a hyperlink spans a page)
% please comment out the following usepackage line and replace
% \usepackage{icml2019} with \usepackage[nohyperref]{icml2019} above.
\usepackage{hyperref}

% Attempt to make hyperref and algorithmic work together better:
\newcommand{\theHalgorithm}{\arabic{algorithm}}

% Use the following line for the initial blind version submitted for review:
\usepackage{icml2019}

% If accepted, instead use the following line for the camera-ready submission:
%\usepackage[accepted]{icml2019}

% The \icmltitle you define below is probably too long as a header.
% Therefore, a short form for the running title is supplied here:
\icmltitlerunning{Compressing Word Embeddings with Uniform Quantization}

%Added by Avner
\usepackage{amsmath}

%\usepackage{graphicx}
%\usepackage{amsmath,amsfonts,amssymb,amsopn,amsbsy,amsthm}
%\usepackage{dsfont,bm,bbm,times,url,verbatim,epstopdf,xspace}
%\usepackage[capitalize]{cleveref}
%\usepackage{placeins}
%\usepackage{amssymb,amsthm}
%\usepackage{bbm,epstopdf}
%\usepackage[group-separator={,}]{siunitx}
%\usepackage[export]{adjustbox}
%\usepackage{hhline}
%\usepackage[hypertexnames=false]{hyperref}
% for footnotes without markers
%\newcommand\blfootnote[1]{%
%  \begingroup
%  \renewcommand\thefootnote{}\footnote{#1}%
%  \addtocounter{footnote}{-1}%
%  \endgroup
%}

%\newcommand{\vfigsp}{\vspace{-0.5em}}
%\newcommand{\vflistsp}{\vspace{-0.25em}}
\newcommand{\vfigsp}{}
\newcommand{\vflistsp}{}

% New commands added from Nystrom notes.
\newcommand{\Nystrom}{Nystr\"{o}m }
\newcommand{\NystromNS}{Nystr\"{o}m} % NS means ``no space''
\newcommand{\NystromCaps}{NYSTR\"{O}M }
\newcommand{\NystromCapsNS}{NYSTR\"{O}M} % NS means ``no space''
\newcommand{\Ap}{A^\perp}
\newcommand{\mua}{\mu_A}
\newcommand{\muap}{\mu_{\Ap}}
\newcommand{\bphi}{\bar{\phi}}
\newcommand{\by}{\bar{y}}
\newcommand{\bS}{\bar{S}}
\newcommand{\bT}{\bar{T}}
%\newcommand{\bw}{\bar{w}}
%\newcommand{\bv}{\bar{v}}
%\newcommand{\bF}{\bar{F}}
%\newcommand{\bg}{\bar{g}}
%\newcommand{\be}{\bar{\eta}}
%\newcommand{\br}{\bar{\rho}}
%\newcommand{\bU}{\bar{U}}
%\newcommand{\bV}{\bar{V}}
%\newcommand{\bLam}{\bar{\Lambda}}
%\newcommand{\bL}{\bar{L}}
\newcommand{\hx}{\hat{x}}
\newcommand{\hb}{\hat{b}}
%\newcommand{\hX}{\hat{X}}
%\newcommand{\hY}{\hat{Y}}
\newcommand{\hp}{\hat{\phi}}
\newcommand{\hK}{\hat{K}}
%\newcommand{\tp}{\tilde{\phi}}
\newcommand{\tk}{\tilde{k}}
%\newcommand{\tC}{\tilde{C}}
\newcommand{\eps}{\epsilon}
\newcommand{\teps}{\tilde{\epsilon}}
\newcommand{\tS}{\tilde{S}}
\newcommand{\tK}{\tilde{K}}
\newcommand{\tZ}{\tilde{Z}}
\newcommand{\tA}{\tilde{A}}
\newcommand{\tf}{\tilde{f}}
\newcommand{\tz}{\tilde{z}}
\newcommand{\tsigma}{\tilde{\sigma}}
\newcommand{\tgamma}{\tilde{\gamma}}
\newcommand{\tlambda}{\tilde{\lambda}}
\newcommand{\hcR}{\widehat{\cR}}
\newcommand{\id}{I}
\newcommand{\sq}{\sqrt{2}}
\newcommand{\ulq}{\underline{q}}
\newcommand{\olq}{\overline{q}}
\newcommand*{\QED}{\hfill\ensuremath{\square}}
\DeclareMathOperator*{\argmin}{arg\,min}
\DeclareMathOperator*{\argmax}{arg\,max}
\DeclareMathOperator*{\rank}{rank}

\newcommand*\conj[1]{\overline{#1}}

% note: I removed the package amsthm because it defines "proof", which is already defined in jmlr2e package
\newcommand{\ie}{i.e.}
\newcommand{\eg}{e.g.}
\newcommand{\etal}{et al.\ }
\newcommand{\etalNS}{et al.}

\def\ddefloop#1{\ifx\ddefloop#1\else\ddef{#1}\expandafter\ddefloop\fi}
%
%% \bbA, \bbB, ...
\def\ddef#1{\expandafter\def\csname bb#1\endcsname{\ensuremath{\mathbb{#1}}}}
\ddefloop ABCDEFGHIJKLMNOPQRSTUVWXYZ\ddefloop
%
%% \bfA, \bfB, ...
%\def\ddef#1{\expandafter\def\csname bf#1\endcsname{\ensuremath{\mathbf{#1}}}}
%\ddefloop ABCDEFGHIJKLMNOPQRSTUVWXYZabcdefghijklmnopqrstuvwxyz\ddefloop
%
%% \bfalpha, \bfbeta, ...,  \bfGamma, \bfDelta, ...,
%\def\ddef#1{\expandafter\def\csname bf#1\endcsname{\ensuremath{\pmb{\csname #1\endcsname}}}}
%\ddefloop {alpha}{beta}{gamma}{delta}{epsilon}{varepsilon}{zeta}{eta}{theta}{vartheta}{iota}{kappa}{lambda}{mu}{nu}{xi}{pi}{varpi}{rho}{varrho}{sigma}{varsigma}{tau}{upsilon}{phi}{varphi}{chi}{psi}{omega}{Gamma}{Delta}{Theta}{Lambda}{Xi}{Pi}{Sigma}{varSigma}{Upsilon}{Phi}{Psi}{Omega}{ell}\ddefloop
%
%% \cA, \cB, ...
\def\ddef#1{\expandafter\def\csname c#1\endcsname{\ensuremath{\mathcal{#1}}}}
\ddefloop ABCDEFGHIJKLMNOPQRSTUVWXYZ\ddefloop
%
%% \mbf0, \mbf1, ...
%\newcommand\mbf{\ensuremath{\mathbf}}
%
%\DeclareMathOperator*{\argmin}{arg\,min}
%\DeclareMathOperator*{\argmax}{arg\,max}
%
%\newcommand\parens[1]{(#1)}
\newcommand\norm[1]{\|#1\|}
%\newcommand\braces[1]{\{#1\}}
%\newcommand\brackets[1]{[#1]}
%\newcommand\ceil[1]{\lceil#1\rceil}
%\newcommand\abs[1]{|#1|}
%\newcommand\ind[1]{\ensuremath{\mathds{1}\{#1\}}}
\newcommand\dotp[1]{\langle #1 \rangle}
%\newcommand\Parens[1]{\left(#1\right)}
%\newcommand\Norm[1]{\left\|#1\right\|}
%\newcommand\Braces[1]{\left\{#1\right\}}
%\newcommand\Brackets[1]{\left[#1\right]}
%\newcommand\Ceil[1]{\left\lceil#1\right\rceil}
%\newcommand\Abs[1]{\left|#1\right|}
%\newcommand\Ind[1]{\mathds{1}\left\{#1\right\}}
\newcommand\Dotp[1]{\left\langle#1\right\rangle}
%
\newcommand{\RR}{\ensuremath{\bbR}} %real numbers
\newcommand{\Nat}{\ensuremath{\bbN}} %natural numbers 
\newcommand{\CC}{\ensuremath{\bbC}} %complex numbers
%
%\newcommand{\risk}{\ensuremath{R}} % risk
%\newcommand{\loss}{\ensuremath{\ell}} % loss
%\newcommand{\logloss}{\ensuremath{\ell_{\operatorname{log}}}} % logistic loss
%
%\newcommand{\rfm}{\ensuremath{z}} % (random) feature map
%\newcommand{\kfm}{\ensuremath{\phi}} % kernel feature map
%\newcommand{\kernel}{\ensuremath{K}} % kernel function
%
%%\setlength{\marginparwidth}{25mm}
%\usepackage[textsize=tiny]{todonotes}
%%\usepackage[disable]{todonotes}
%\newcommand{\djh}[2][]{\todo[color=red!20!white,#1]{DH: #2}}
%\newcommand{\avner}[2][]{\todo[color=red!20!white,#1]{AM: #2}}
\newcommand\new[1]{\textcolor{blue}{#1}}
\newcommand\todo[1]{\textcolor{red}{#1}}
\newcommand\avner[1]{\textcolor{red}{AM: #1}}
\newcommand\tri[1]{\textcolor{red}{TD: #1}}
\newcommand\jian[1]{\textcolor{red}{JZ: #1}}
%
%\newcommand{\numtr}{\ensuremath{n_{\operatorname{tr}}}}
%\newcommand{\numho}{\ensuremath{n_{\operatorname{ho}}}}
%\newcommand{\numte}{\ensuremath{n_{\operatorname{te}}}}
%
%\usepackage{tabularx}
%\newcolumntype{Y}{>{\centering\arraybackslash}X}
%\usepackage{multirow}
%
%\usepackage{algorithm}
%\usepackage{algorithmic}
%\usepackage{pdflscape}
%
%% define vector and matrix symbols
%\newcommand{\vct}[1]{\boldsymbol{#1}} % vector
%\newcommand{\mat}[1]{\boldsymbol{#1}} % matrix
%\newcommand{\cst}[1]{\mathsf{#1}}  % constant
%
%%%%% Special math symbols
%\newcommand{\field}[1]{\mathbb{#1}}
%\newcommand{\R}{\field{R}} % real domain
%\newcommand{\C}{\field{C}} % complex domain
%\newcommand{\F}{\field{F}} % functional domain
%%\newcommand{\T}{^{\top}\!\!} % transpose
%\newcommand{\T}{^{\textrm T}} % transpose
%
%%% operator in linear algebra, functional analysis
%\newcommand{\inner}[2]{#1\cdot #2}
%%\newcommand{\norm}[1]{\|#1\|}
%\newcommand{\twonorm}[1]{\left\|#1\right\|_2^2}
%\newcommand{\onenorm}[1]{\|#1\|_1}
%\newcommand{\Map}[1]{\mathcal{#1}}  % operator in functions, maps such as M: domain1 --> domain 2
%
%% operator in probability: expectation, covariance, 
\newcommand{\E}{\mathbb{E}} % Expectation symbol (ADDED BY TRI)
\newcommand{\Prob}{\mathbb{P}}
\newcommand{\ProbOpr}[1]{\mathbb{#1}}
%% independence
%\newcommand\independent{\protect\mathpalette{\protect\independenT}{\perp}}
%\def\independenT#1#2{\mathrel{\rlap{$#1#2$}\mkern2mu{#1#2}}}
%%\newcommand{\ind}[2]{{#1} \independent{#2}}
%\newcommand{\cind}[3]{{#1} \independent{#2}\,|\,#3} % conditional independence
\newcommand{\expect}[2]{%
\ifthenelse{\equal{#2}{}}{\ProbOpr{E}_{#1}}
{\ifthenelse{\equal{#1}{}}{\ProbOpr{E}\left[#2\right]}{\ProbOpr{E}_{#1}\left[#2\right]}}} % Expectation: syntax: E{1}{2} = E_1[2], E{}{2}=E[2], E{1}{} = E_1
\newcommand{\var}[2]{%
\ifthenelse{\equal{#2}{}}{\ProbOpr{VAR}_{#1}}
{\ifthenelse{\equal{#1}{}}{\ProbOpr{VAR}\left[#2\right]}{\ProbOpr{VAR}_{#1}\left[#2\right]}}} 
% Expectation: syntax: V{1}{2} = V_1[2], V{}{2}=V[2], V{1}{} = V_1
%\newcommand{\cndexp}[2]{\ProbOpr{E}\,[ #1\,|\,#2\,]}  % conditional expectation
%
%% operator in optimization
%%\DeclareMathOperator{\argmax}{arg\,max}
%%\DeclareMathOperator{\argmin}{arg\,min}
%
%% special functions
%\newcommand{\trace}[1]{\operatornamewithlimits{tr}\left\{#1\right\}}
\newcommand{\diag}{\operatornamewithlimits{diag}}
%\newcommand{\sign}{\operatornamewithlimits{sign}}
%\newcommand{\const}{\operatornamewithlimits{const}}
%
%% special display
%\newcommand{\parde}[2]{\frac{\partial #1}{\partial  #2}}
%
%
%% environment
\newtheorem{thm}{Theorem}
\newtheorem{theorem}{Theorem}
\newtheorem{definition}{Definition}
\newtheorem{lemma}[theorem]{Lemma}
\newtheorem{conjecture}[theorem]{Conjecture}
\newtheorem{proposition}[theorem]{Proposition}
\newtheorem{claim}{Claim}
\newtheorem{corollary}{Corollary}[theorem]

\newtheorem{innercustomthm}{Theorem}
\newenvironment{customthm}[1]
{\renewcommand\theinnercustomthm{#1}\innercustomthm}
{\endinnercustomthm}

\newtheorem{innercustomprop}{Proposition}
\newenvironment{customprop}[1]
{\renewcommand\theinnercustomprop{#1}\innercustomprop}
{\endinnercustomprop}


%\newtheorem{thm}{Theorem}
%\newtheorem{theorem}{Theorem}
%\newtheorem{claimcor}{Corollary}[claim]
%\newtheorem{lemma}{Lemma}[theorem]



%% DPP symbols
%\newcommand{\Xt}{{\mathcal{X}_{(t)}}} % X_t
%\newcommand{\Yt}{{Y_{(t)}}} % Y_t
%\newcommand{\Ystart}{{Y^{\star}_{(t)}}} % Y*_t
%\newcommand{\ground}{{\mathcal{Y}}} % ground set
%\newcommand{\groundX}{{\mathcal{X}}} % ground set
%
%
%
%% shorthand
%\newcommand{\vtheta}{\vct{\theta}}
%\newcommand{\vmu}{\vct{\mu}}
%\newcommand{\valpha}{\vct{\alpha}}
%\newcommand{\vc}{\vct{c}}
%\newcommand{\vp}{\vct{p}}
%\newcommand{\vq}{\vct{q}}
%\newcommand{\vx}{{\vct{x}}}
%\newcommand{\vy}{\vct{y}}
%\newcommand{\vz}{{\vct{z}}}
%\newcommand{\vu}{\vct{u}}
%\newcommand{\vo}{{\vct{o}}}
%\newcommand{\va}{\vct{a}}
%\newcommand{\vb}{\vct{b}}
%\newcommand{\vr}{\vct{r}}
%\newcommand{\vt}{\vct{t}}
%\newcommand{\vs}{\vct{s}}
%\newcommand{\vv}{\vct{v}}
%\newcommand{\vw}{\vct{w}}
%\newcommand{\ones}{\vct{1}}
%\newcommand{\mU}{\mat{U}}
%\newcommand{\mA}{\mat{A}}
%\newcommand{\mB}{\mat{B}}
%\newcommand{\mC}{\mat{C}}
%\newcommand{\mW}{\mat{W}}
%\newcommand{\mH}{\mat{H}}
%\newcommand{\mS}{\mat{S}}
%\newcommand{\mJ}{\mat{J}}
%\newcommand{\mM}{\mat{M}}
%\newcommand{\mT}{\mat{T}}
%\newcommand{\mZ}{\mat{Z}}
%\newcommand{\mO}{\mat{O}}
%\newcommand{\mY}{\mat{Y}}
%%\newcommand{\cN}{\cst{N}}
%%\newcommand{\cQ}{\cst{Q}}
%%\newcommand{\cD}{\cst{D}}
%%\newcommand{\cL}{\cst{L}}
%%\newcommand{\cK}{\cst{K}}
%%\newcommand{\cH}{\cst{H}}
%\newcommand{\sQ}{\mathcal{Q}}
%\newcommand{\sS}{\mathcal{S}}
%\newcommand{\mL}{\mat{L}}
%\newcommand{\mI}{\mat{I}}
%\newcommand{\mK}{\mat{K}}
%\newcommand{\mSigma}{\mat{\Sigma}}
%\newcommand{\sF}{\mathcal{F}}
%\newcommand{\vzero}{\vct{0}}
%\newcommand{\vf}{\vct{f}}
%\newcommand{\vF}{\vct{F}}
%\newcommand{\vP}{\vct{P}}
%
%
%\newcommand{\vh}{\vct{h}}
%\newcommand{\vg}{\vct{g}}
%\newcommand{\vphi}{\vct{\phi}}
%\newcommand{\vpsi}{\vct{\psi}}
%\newcommand{\vdelta}{\vct{\delta}}
%\newcommand{\vomega}{\vct{\omega}}
%\newcommand{\mOmega}{\mat{\Omega}}
%
%% 
%\newcommand{\nx}{\tilde{x}}
%\newcommand{\vnx}{{\tilde{\vx}}}
%\newcommand{\vnz}{{\tilde{\vz}}}
%%\newcommand{\deltavx}{(\vnx-\vx)}
%\newcommand{\deltavx}{\delta_\vx}
%\newcommand{\vmx}{\bar{\vx}}
%\newcommand{\vmz}{\bar{\vz}}
%\newcommand{\sigmax}{\mSigma_{\vx}}
%\newcommand{\sigmaz}{\mSigma_{\vz}}
%\newcommand{\no}{\tilde{o}}
%\newcommand{\vno}{{\tilde{\vo}}}
%\newcommand{\nell}{\tilde{\ell}}
%\newcommand{\jacob}{\mat{J}}
%\newcommand{\hess}{\mat{H}}
%\newcommand{\mloss}{\hat{\ell}}
%\newcommand{\vDD}{\vct{\Delta}}
%%
%\newcommand{\eat}[1]{}
%
%\newcommand{\FS}[1]{{\color{blue}FS:#1}}
%
%\newcommand\id{\ensuremath{\mathbbm{1}}}
%\newcommand*{\QED}{\hfill\ensuremath{\square}}
\newcommand{\defeq}{:=}
\newcommand{\eqdef}{=:}
%\newcommand{\var}{{\rm Var}} % Variance
\providecommand{\tr}{\mathop{\rm tr}}
\newcommand\numberthis{\addtocounter{equation}{1}\tag{\theequation}}
\begin{document}
	
\twocolumn[
\icmltitle{Compressing Word Embeddings with Uniform Quantization}

% It is OKAY to include author information, even for blind
% submissions: the style file will automatically remove it for you
% unless you've provided the [accepted] option to the icml2019
% package.

% List of affiliations: The first argument should be a (short)
% identifier you will use later to specify author affiliations
% Academic affiliations should list Department, University, City, Region, Country
% Industry affiliations should list Company, City, Region, Country

% You can specify symbols, otherwise they are numbered in order.
% Ideally, you should not use this facility. Affiliations will be numbered
% in order of appearance and this is the preferred way.
\icmlsetsymbol{equal}{*}

\begin{icmlauthorlist}
	\icmlauthor{Random Name}{equal,stan}
\end{icmlauthorlist}

\icmlaffiliation{stan}{Department of Computer Science, Stanford University, Stanford, California, USA}

\icmlcorrespondingauthor{Random Name}{random@stanford.edu}

% You may provide any keywords that you
% find helpful for describing your paper; these are used to populate
% the "keywords" metadata in the PDF but will not be shown in the document
\icmlkeywords{Machine Learning, ICML}

\vskip 0.3in
]

% this must go after the closing bracket ] following \twocolumn[ ...

% This command actually creates the footnote in the first column
% listing the affiliations and the copyright notice.
% The command takes one argument, which is text to display at the start of the footnote.
% The \icmlEqualContribution command is standard text for equal contribution.
% Remove it (just {}) if you do not need this facility.

%\printAffiliationsAndNotice{}  % leave blank if no need to mention equal contribution
\printAffiliationsAndNotice{\icmlEqualContribution} % otherwise use the standard text.


\begin{abstract}
%Compressing word embeddings is critical to their deployment on memory-constrained devices.
We study the principles governing the downstream performance of compressed word embeddings.
We empirically demonstrate that existing metrics for evaluating compression performance align poorly with downstream performance, and propose a new metric---the \textit{eigenspace overlap metric}---which aligns much better.
%This metric measures the degree of overlap between the subspaces spanned by the eigenvectors of the nonzero eigenvalues of the Gram matrices of the compressed and uncompressed embeddings.
To explain this metric's influence on downstream performance, we prove bounds on the generalization performance of compressed embeddings in terms of this metric in the linear regression setting.
We then use this metric to better understand the empirical success of a simple compression based on uniform quantization, by proving explicit bounds on its eigenspace overlap.

%In this paper, we present a deeper understanding on the generalization performance of compressed word embeddings in downstream tasks.
%We begin by empirically comparing different embedding compression methods. Surprisingly, we observe simple uniform quantization can compete with state-of-the-art methods, and can match the generalization performance of uncompressed embeddings at high compression rate.
%To explain this performance comparison, 
%we propose a new compression quality metric \textit{eigenspace overlap}, which empirically correlates strongly with downstream task performance while conventional metrics fail to so. Built on this metric, we analyze the generalization bound for compressed embedding in downstream tasks, and use it to explain why uniformly quantized embeddings can attain strong generalization performance. Our analysis reveals that uniformly quantized embeddings can achieve strong eigenspace overlap with respect to the uncompressed counterparts with high probability. Thus it can match the performance of uncompressed embedding with high compression rate.


%We explicitly relate this metric with the downstream performance of embeddings by 
%
%We show that this metric is closely connected to downstream performance by proving bounds on the generalization performance of compressed embeddings in the linear regression setting in terms of this metric.
%
%
%
%To explain the downstream performance of compressed embeddings using this metric,
%
%To explain the impact of eigenspace overlap on downstream performance,
%
%We directly connect this metric with the downstream performance
%
%show that we can bound the generalization performance of compressed embeddings in the linear regression setting in terms of this metric.

%We then show that we can use this metric to better understand the performance of specific compression methods by proving explicit bounds on the eigenspace overlap of a simple compression method based on uniform quantization.




%Compressing word embeddings is critical to their deployment on memory-constrained devices such as mobile phones.
%State-of-the-art embedding compression methods attains strong empirical performance by learning compressed representations using deep architectures.
%In this paper, we work on a simple and fast embedding compression method based on uniform quantization, which can match the empirical performance of state-of-the-art compression methods across different compression rates and tasks.
%Theoretically, we analyze how the precision and dimensionality of the quantized word embeddings jointly impact their generalization performance in the context of linear ridge regression.
%Our analysis reveals that one can attain better performance under memory constraints by using lower-precision embeddings whose dimension approaches but does not exceed the optimal embedding dimension.

%First, we perform an empirical analysis of the downstream performance of several compression methods. 
%Surprisingly, we find that simple baselines, including one based on uniform quantization, can compete with the state-of-the-art neural compression algorithms, and that their success cannot be explained with existing metrics.


%We then use this metric to provide explicit bounds on the generalization performance of embeddings compressed with uniform quantization.

%To explain these empirical results, we propose a new metric---the eigenspace overlap metric---which we show better explains the downstream performance of embeddings.
%
%We investigate how to design simple and fast compression methods for word embeddings with strong downstream performance.
%To better reveal the impact of compression on downstream performance, we first propose a new metric---the \textit{eigenspace overlap metric}---for evaluating the quality of a compressed embedding.
%We prove generalization bounds in terms of this metric, and demonstrate that it aligns much better with the embedding's downstream performance than existing metrics.
%For a compressed embedding to have high eigenspace overlap, its rank must be similar to that of the uncompressed embedding.
%This motivates our design of a simple and fast compression method which preserves the embedding rank by uniformly quantizing the entries of the original embedding matrix.
%Theoretically, we show that uniform quantization has a small impact on the eigenspace overlap when \todo{$<$include condition here$>$}.
%Empirically, we demonstrate that our compression method is orders of magnitude faster than existing methods, and can match their downstream performance \todo{$<$include numbers here$>$}.


%Compressing word embeddings is critical to their deployment on memory-constrained devices such as mobile phones.
%Although numerous sophisticated compression techniques have been proposed, they are generally quite computationally expensive, and their performance is poorly understood.
%To better explain the performance of existing techniques, we propose a new eigenspace overlap metric which measure the 


% OLD VERSION
%Compressing word embeddings is critical to their deployment on memory-constrained devices such as mobile phones.
%State-of-the-art embedding compression methods attain strong empirical performance by learning compressed representations using deep architectures.
%In this paper, we work on a simple and fast embedding compression method based on uniform quantization that matches the empirical performance of state-of-the-art compression methods across different compression rates and tasks.
%To shed light on the settings in which this compression method is guaranteed to perform well, we analyze the impact of compression on generalization performance.
%This analysis reveals that if the number of bits per embedding entry is chosen appropriately, strong generalization performance is guaranteed.

%
%To shed light on the settings in which this compression method is guaranteed to perform well, we analyze the
%
%To lever
%
%To shed light on the settings in which this compression method is guaranteed to perform well, we present generalization bounds leveraging a notion of spectral approximation between matrices.
%
%This analysis that 
%
%To shed light on settings where uniformly quantized embeddings attain strong performance in downstream tasks, we theoretically analyze how quantization precision impact generalization performance in the context of linear ridge regression.
%Our analysis reveals that when quantization error is small relative to regularization, uniformly quantized embeddings can achieve similar performance as the uncompressed embeddings in downstream tasks.

%This method first determines a threshold value to clip extreme values in the embedding matrix using simple golden section search; it then uniformly quantizes the clipped embedding into fixed-point representation. 
%Without computationally expensive training, the uniform quantization approach can empirically match the performance of state-of-the-art compression methods across different compression rates.
%approach and solution
%Contribution 1: empirically match 
%Contribution 2: 

\end{abstract}

\section{Introduction}
\label{sec:intro}
\subsection{Jian Version}
In recent years, \textit{word embeddings} \citep{word2vec13,glove14,fasttext18} have brought large improvements to a wide range of applications in natural language processing (NLP) \citep{collins16,drqa17}.
By encoding words as low-dimensional dense vectors, word embeddings allow NLP tasks to be framed as pattern recognition problems over dense continuous spaces, which can then be tackled using the powerful machinery of neural networks.
However, these word embeddings can occupy a very large amount of memory, making it impractical to deploy them to memory-constrained environments like smart phones.
To this end, it is of great importance to decrease the amount of memory occupied by the embeddings for massive deployment.


Designing simple embedding compression methods with both strong empirical performance and provable guarantees is challenging.
For example, the current state-of-the-art methods for compressing word embeddings are deep learning-based approaches, such as deep compositional code \citep{dccl17} and K-way discrete code \citep{chen2018learning}. In these methods, the compression are achieved by training a neural network architectures to represent each word as a sum of vectors from learned dictionaries.


In this work, we revisit a compression method based on simple uniform quantization, and analyze the impact of quantization on generalization performance in the case of linear regression models.
The quantization-based approach has two parts: By using golden section search, we find the optimal threshold at which to clip the extreme values in the word embedding matrix. We then uniformly quantize the clipped embeddings within the clipped interval. Based on these two steps, this simple approach can achieve fast compression without computationally expensive training or extensive hyperparameter tuning.

Empirically, we demonstrate that uniform quantization can match the performance of the more complex baselines across a variety of tasks, embedding types, and compression rates.
On the machine reading comprehension task, we evaluate different embeddings with DrQA model \cite{drqa17} using the Stanford Question Answering Dataset (SQuAD) \citep{squad16}. Uniform quantization can attain 32x compression for GloVe and fastText embeddings, while on average attaining F1 scores within 0.45\% and \todo{XX\%} absolute of the full-precision embeddings, respectively. To evaluate on more NLP applications with various model architecture, we also show that quantized embeddings matche the generalization performance of embeddings compressed with state-of-the-art deep learning-based methods, across 7 sentiment analysis, 2 word similarity and 2 analogy tasks. 

We theoretically show that the quantization has negligible effect on generalization in important schemes when quantization error is relatively small comparing to regularization strength. Specifically we work in the case of linear regression, which is a simplified model of regression and ranking-based NLP tasks such as sentiment analysis. Our analysis reveals that when the spectrum of the word embedding matrix decays slowly (which we empirically observe to be true), strong regularization has minimally impact on generalization performance. Thus uniformly quantized embeddings can have similar generalization performance to that of uncompressed full precision embeddings.
\todo{Discuss how clipping fits into theory.?}



\subsection{Avner Version}
In recent years, \textit{word embeddings} \citep{word2vec13,glove14,fasttext18} have brought large improvements to a wide range of applications in natural language processing (NLP) \citep{collins16,drqa17}.
By encoding words as low-dimensional dense vectors, word embeddings allow NLP tasks to be framed as pattern recognition problems over dense continuous spaces, which can then be tackled using the powerful machinery of neural networks.
However, these word embeddings can occupy a very large amount of memory, making it impractical to deploy them to memory-constrained environments like smart phones.
Our goal in this work is to dramatically decrease the amount of memory occupied by the embeddings, while provably retaining strong performance on downstream tasks.

Designing a compression scheme which is capable of retaining strong performance at low memory budgets, while also being amenable to theoretical analysis, is a challenging problem.
There have been numerous important works proposing compression schemes for word embeddings and studying their empirical performance, for example using deep dictionary learning or sparse representations \citep{sparse16,andrews16,dccl17}.
Although these methods can attain large reductions in memory usage while performing well on tasks such as language modeling \citep{mikolov10} and machine translation \citep{bahdanau15}, to the best of our knowledge there exists no analysis describing the effect compression has on generalization performance for these methods.

%Designing powerful and principled compression schemes is challenging because it is unclear a priori how to model how compression affects performance.
%This is particularly true when the compression schemes being considered are complex, and when large non-convex models are employed for the downstream tasks.
%For example, the current state-of-the-art method for compressing word embeddings, called deep compositional code learning (DCCL), uses a deep architecture to represent each word as a sum of vectors from learned dictionaries \citep{dccl17}.

In this work, we show that a simple compression method based on uniform quantization can compete with the above-mentioned methods, and we bound this method's impact on the generalization performance of linear regression models.
This method has two parts: first, it determines the optimal threshold at which to clip the extreme values in the word embedding matrix.
It then uniformly quantizes the clipped embeddings within the clipped interval.
This algorithm is easy to implement, has no hyperparameters, and runs in seconds on a single CPU core.

Empirically, we demonstrate that this method can match the performance of the state-of-the-art compression schemes on question answering, sentiment analysis, and word analogy and similarity tasks, for both GloVe \citep{glove14} and fastText \citep{fasttext18} embeddings.
On the Stanford Question Answering Dataset (SQuAD) \citep{squad16}, for example, it attains a compression ratio of 32x with an average F1 score only 0.47\% absolute below the full-precision GloVe embeddings, while the deep dictionary learning method of \citet{dccl17} is 0.43\% below.

Theoretically, we show in the context of linear ridge regression that quantization has a negligible effect on generalization performance when the quantization noise is small relative to the regularization parameter.
Furthermore, we show that when the spectrum of the word embedding matrix decays slowly (which we empirically observe to be true), a large regularizer, and thus low-precision, will perform similarly to the full-precision embeddings.
Importantly, our analysis is not specific to word embeddings; it directly extends to any setting in which a linear model is being learned over a uniformly quantized representation.
%\todo{Discuss how clipping fits into theory. Motivate why regression matters (as opposed to classification) for NLP.}
%Our theoretical analysis builds on recent work analyzing the generalization performance of low-precision representations in the context of kernel ridge regression \citep{lprff18}.

Our main contributions:
\begin{itemize}
	\item We show that a simple compression scheme based on uniform quantization can compete with state-of-the-art word embedding compression methods on a variety of tasks.
	\item We prove that this compression method's impact on the generalization performance of linear ridge regression models is minimal when the quantization noise is small relative to the regularization parameter.
\end{itemize}

The rest of this paper is organized as follows:
In Section~\ref{sec:relwork} we review related work, including existing methods for compressing word embeddings.
In Section~\ref{sec:uniform} we present our compression method.
We discuss our large-scale experiments in Section~\ref{sec:experiments}, and the theoretical analysis for our method in Section~\ref{sec:theory}.
We conclude in Section~\ref{sec:conclusion}.


\section{Uniform Quantization}
\label{sec:uniform}

\section{Experiments}
\label{sec:experiments}

\section{Theory}
\label{sec:theory}

\section{Related Work}
\label{sec:relwork}

\section{Conclusion}
\label{sec:conclusion}

% Acknowledgements should only appear in the accepted version.
%\section*{Acknowledgements}
%
%\textbf{Do not} include acknowledgements in the initial version of
%the paper submitted for blind review.

%\subsection{Figures}
% EXAMPLE FIGURE
%\begin{figure}[ht]
%\vskip 0.2in
%\begin{center}
%\centerline{\includegraphics[width=\columnwidth]{icml_numpapers}}
%\caption{Historical locations and number of accepted papers for International
%Machine Learning Conferences (ICML 1993 -- ICML 2008) and International
%Workshops on Machine Learning (ML 1988 -- ML 1992). At the time this figure was
%produced, the number of accepted papers for ICML 2008 was unknown and instead
%estimated.}
%\label{icml-historical}
%\end{center}
%\vskip -0.2in
%\end{figure}


%\subsection{Algorithms}
% EXAMPLE ALGORITHM
%\begin{algorithm}[tb]
%   \caption{Bubble Sort}
%   \label{alg:example}
%\begin{algorithmic}
%   \STATE {\bfseries Input:} data $x_i$, size $m$
%   \REPEAT
%   \STATE Initialize $noChange = true$.
%   \FOR{$i=1$ {\bfseries to} $m-1$}
%   \IF{$x_i > x_{i+1}$}
%   \STATE Swap $x_i$ and $x_{i+1}$
%   \STATE $noChange = false$
%   \ENDIF
%   \ENDFOR
%   \UNTIL{$noChange$ is $true$}
%\end{algorithmic}
%\end{algorithm}

%\subsection{Tables}
% EXAMPLE TABLE
%\begin{table}[t]
%\caption{Example table.}
%\label{sample-table}
%\vskip 0.15in
%\begin{center}
%\begin{small}
%\begin{sc}
%\begin{tabular}{lcccr}
%\toprule
%Data set & Naive & Flexible & Better? \\
%\midrule
%Cleveland & 83.3$\pm$ 0.6& 80.0$\pm$ 0.6& $\times$\\
%Glass2    & 61.9$\pm$ 1.4& 83.8$\pm$ 0.7& $\surd$ \\
%\bottomrule
%\end{tabular}
%\end{sc}
%\end{small}
%\end{center}
%\vskip -0.1in
%\end{table}


\bibliography{ref}
\bibliographystyle{icml2019}


\end{document}


% This document was modified from the file originally made available by
% Pat Langley and Andrea Danyluk for ICML-2K. This version was created
% by Iain Murray in 2018. It was modified from a version from Dan Roy in
% 2017, which was based on a version from Lise Getoor and Tobias
% Scheffer, which was slightly modified from the 2010 version by
% Thorsten Joachims & Johannes Fuernkranz, slightly modified from the
% 2009 version by Kiri Wagstaff and Sam Roweis's 2008 version, which is
% slightly modified from Prasad Tadepalli's 2007 version which is a
% lightly changed version of the previous year's version by Andrew
% Moore, which was in turn edited from those of Kristian Kersting and
% Codrina Lauth. Alex Smola contributed to the algorithmic style files.
