



















To further demonstrate how the eigenspace overlap metric can be used to better understand the performance of compressed embeddings, in this section we show that we can upper bound the eigenspace overlap for uniformly quantized embeddings.
Given the above theoretical and empirical connections between eigenspace overlap and generalization performance, these bounds help explain why the uniformly quantized embeddings perform so well.
To prove these bounds, we leverage the classic Davis-Kahan $sin(\Theta)$ theorem from matrix perturbation analysis \citep{sintheta70}.
Because we know exactly what the noise structure of the uniform quantization method is, we can use this knowledge to bound how much the eigenspace of the compressed embeddings can differ from the uncompressed embeddings.

We now present the result:
\begin{theorem}
	Let $X \in \RR^{n\times d}$ be a bounded embedding matrix with $X_{ij} \in [-\frac{1}{\sqrt{d}},\frac{1}{\sqrt{d}}]$ with largest and smallest singular values $\sigma_{max}$ and $\sigma_{min}$.
	Then the eigenspace overlap of the corresponding $b$-bit uniformly quantized embedding matrix can be lower bounded as follows:
	\begin{eqnarray*}
		\eigover(X,\tX) &\geq& d - \Bigg(\frac{4\sqrt{n}}{2^b-1} \cdot \frac{\sigma_{max} + \frac{\sqrt{n}}{2^b-1} }{\sqrt{\sigma_{min}}} \Bigg)^2
	\end{eqnarray*}
\end{theorem}
We can further simplify this expression using the fact that $\sigma_{max} = \|X\|_2 \leq \sqrt{n}$; using this fact, we get the following corollary:
\begin{corollary}
If $b \geq \log_2\bigg(\frac{8n}{\sqrt{\rho \, d\, \sigma_{min}}} + 1\bigg)$, then the $b$-bit uniformly quantized embedding matrix $\tX$ satisfies $\eigover(X,\tX) \geq (1-\rho)d$.
\end{corollary}


\subsection{Theory validation}
	\begin{itemize}
		\item Impact of quantization on overlap 
			\begin{itemize}
 				\item Exp 1: overlap vs precision for different dimensionality. Expectation: overlap increases with higher precision.
				\item Exp2: overlap vs dimensionality for different precision. Expectation: overlap increases with dimensionality. This explains that under fix memory budget, using lower bits quantization can be beneficial
			\end{itemize}
		\item The impact of clipping on eigen-subspace overlap
			\begin{itemize}
				\item Simulation based experiments on subspace overlap as a function of different clipping threshold and precision. 
				\item The way we introduce this in: in our main theorem, we assume the dynamic range is $O(1/\sqrt{d})$ as a consequence of the automatic clipping. We want to show here this is the case in practice and then discuss the specific way clipping influence eigenspace overlap.
			\end{itemize}
		\item Loop-back discussion on the large scale empirical experiments in Section~\ref{subsec:hard_explain}
	\end{itemize}
